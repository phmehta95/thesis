\chapter{Super-Kamiokande Gadolinium Upgrade}\label{chp:superkgdupgrade}


\section{Motivation behind Super-Kamiokande Gadolinium Upgrade}

In order to be able to possibly observe the supernova relic neutrino (SRN) flux, also known as diffuse supernova neutrino background (DSNB) flux it was proposed that Gadolinium (Gd) should be added to the the water in Super-Kamiokande. Natural isotopes of gadolinium have large thermal neutron capture cross sections. As a result of this, when neutrons are captured on them there is a cascade of gamma rays that occurs, with an energy totalling ~8 MeV, whereas neutron capture that occurs on hydrogen produces a single 2.2 MeV gamma ray. Two such natural isotopes, Gd-155 and Gd-157 have thermal neutron capture cross sections of 60740 barns and 253700 barns respectively, while the thermal neutron capture cross section of hydrogen is just 0.329 barns. The Super-Kamiokande with Gadolinium experiment, formerly known as GADZOOKS! (Gadolinium Antineutrino Detector Zealously Outperforming Old Kamiokande, Super!) was proposed in 2003, which stated the intention of enriching Water Cherenkov detectors with water soluble gadolinium salt.