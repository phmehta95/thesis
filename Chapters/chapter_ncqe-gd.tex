\chapter{NCQE-Gd}
\label{chp:ncqegd}

Both MC and data neutrino events undergo a reconstruction phase, where the low-energy fitter BONSAI is applied to the event. This reconstruction is carried out using timing and cable information, however charge information is omitted. The ouput of BONSAI gives information which will be used in the reduction phase of the data and allow for the selection of the NCQE sample. BONSAI is a low-energy fitter used in the few-MeV to tens of MeV energy range, therefore suitable for this analysis. It's directional reconstruction uses the fitted vertex and maximises for direction while it's energy reconstruction of the event is based upon the number of hit PMTs. Equation \ref{bonsai} shows how the vertex reconstruction is formulated, where $w_{i}$ is the Gaussian hit weight of the PMT, $t_{i}$ is the hit time of the PMT and $x_{i}$ is the position of the hit PMT.
\newline
\begin{equation}
\label{bonsai}
g(\vec{v})=\sum_{i=1}^{N} w_{i} e^{\left.-0.5\left(t_{i}-\left|\overrightarrow{x_{i}}-\vec{v}\right| / c\right) / \sigma\right)^{2}}
\end{equation}
The following quantities comprise the BONSAI output, the first two being helpful spectator variables and the latter five constituting parameters which are used in the reduction phase of the analysis, from which the neutrino NCQE sample is determined.\\

\underline{Neutrino vertex direction}\\

This vector points towards the direction which is an average over all the Cherenkov cone axes which are produced, due to there being multiple leptons induced in the interaction.\\



\underline{Neutrino vertex position}\\
In NCQE events the vector of the Cherenkov cone produced by an outgoing charged lepton, gives the point at which the Oxygen molecule involved in an NCQE event excites and de-excites due to an interaction with a neutrino and it is this interaction point which is the neutrino vertex position.
\newline

 \underline{Reconstructed energy}\\
 In line with the standard SK low energy analysis definition, this energy is simply the reconstructed energy with the 0.511MeV electron mass omitted. The range for Erec in this variable is 3.49 MeV to 29.49 MeV.
\newline

 \underline{Dwall}\\
 This variable gives the minimum distance of the neutrino vertex position from the closest wall of the Super-Kamiokande detector.
\newline

 \underline{Effwall}\\
 Thus variable gives the distance between the neutrino vertex posiiton and the closest wall, but moving back from the vertex position along the neutrino vertex direction vector.
\newline

 \underline{Vertex direction and goodness coefficient}\\
 The coefficient $ovaQ$ (defined in Equation \ref{ovaq}) describes the quality of the vertex reconstruction. It consists of two parameters $g^2_{vtx}$ and $g^2_{dir}$ where the former describes the goodness of the vertex which is based on PMT hit timings, and increases the sharper an event is in time. The latter is the directional goodness and measures the azimuthal uniformity in the ring pattern produced by the Cherenkov cone, which decreass the more uniform an event is in space. As a result of this, $ovaQ$ increases the more uniform and sharp in time an event is.
\newline

 \underline{Cherenkov angle $theta_{C}$}\\
 In neutral current events, this value determines the opening angle of the Cherenkov cone and unlike for charged current events does not 
\newline

\subsection{True neutron tagging information}
\subsection{Primary selection criteria}
\subsection{Secondary selection criteria}
When the neutron vertex is found by this method, 14 variables which describe different aspects of the neutron candidate are calculated. For each of the neutron candidates the vector of these variables are computed and fed into the neural network and this produces an output value which is between 0 and 1. These variables relate to different features regarding categorising hits from neutron capture on Gd or H, including the number of the hits from neutron capture, the isotropy of these hits, the Cherenkov angles of these hits and the position of the neutron vertex in the detector when capture occurs. A description of these variables are given as follows:
\newline


\underline{N10nvx}
\newline
This is the number of hits in the 10ns sliding window of the neutron candidate
\newline


\underline{N300S}
\newline
Excluding the number of hits in the 10ns sliding window (N10nvx), this is the number of hits in the extended window of 300ns.
\newline


\underline{NcS}
\newline
This variable is defined as:
\begin{equation}
\label{ncs}
    NcS = N10nvx - Nclushit
\end{equation}

    Where $Nclushit$ is the number of clusterised hits: if hit \textit{i} and \textit{j} are hits on PMTs, then for hit \textit{i} and hit \textit{j} the hit vector $\hat{r}_i$ can be written as:

\begin{equation}
\label{hit}
    \hat{r}_{i}=\frac{\overrightarrow{P M T_{i}}-\overrightarrow{V T X_{n}}}{\left\|\overrightarrow{P M T_{i}}-\overrightarrow{V T X_{n}}\right\|}
\end{equation}

where the angle at the point of the neutron capture vertex between $\hat{r}_{i}$ and $\hat{r}_{j}$ of the PMT hits is defined as:

\begin{equation}
\theta_{i j}=\arccos \left(\hat{p}_{i} \cdot \hat{p}_{j}\right)
\end{equation}

where the hits are defined as clustered if $\theta_{ij}$ is less than 14


\underline{llrca}
\newline
This variable is the log likelihood ratio calculated using triplets of hits from N10nvx that make up a rudimentary Cherenkov cone, from which the opening angle $\theta$ is calculated. Two PDFs ($\theta_{Ci}$) and ($\theta_{Ci}$) are calculated from each $\theta_{Ci}$ where p\_s and p\_b are the probability density functions of $\theta_{C}$ depending on whether the hits come from a true neutron capture on Gd or H or a false neutron capture which makes up the background. The log likelihood ratio variable is computed using Equation {\ref{llrca}}.

\begin{equation}
\label{llrca}
\newline
  llrca =\sum_{i \in\{ { triplets }\}} \log \left(\frac{f_{B}\left(\theta_{C i}\right)}{f_{S}\left(\theta_{C i}\right)}\right)
\end{equation}


\underline{beta-n}
\newline
These variables (where n = 1,2,3,4,5) are defined using Legendre polynomials, shown in Equation \ref{beta} \cite{snopaper}, which gives the isotropy of the Cherenkov hits.

\begin{equation}
\label{beta}
 beta- n=\frac{2}{N 10 {nvx}(N 10 {nvx}-1)} \sum_{i \neq j}  { Legendre }_{n}\left(\cos \theta_{i j}\right)
\end{equation}

where $Legendre_n$ gives the Legendre polynomial of order $n$ and $theta_{ij}$ is the angle between hit PMTs relative to the neutron capture vertex.



\underline{ndwall}
\newline
This parameter, similar to dwall, gives the shortest distance of the neutron capture vertex from the wall of the Super-Kamiokande tank.
\newline



\underline{ntowall}
\newline
This variable (similar to effwall), gives the distance of the neutron capture vertex from the wall, however, unlike ndwall it gives the direction of the neutron capture specifically along the direction of the centre of the hits. The direction ($\overrightarrow{R}$) is given by:

\begin{equation}
\overrightarrow{\operatorname{dir}}=\sum_{i=1}^{N 10 n v x} \hat{p}_{i}
\end{equation}





