\chapter{The Tokai-to-Kamioka experiment}
\epigraph{(sarcastically) ``Nice experiment, guys.''}{Arnold Rimmer, Red Dwarf: S5E5 Demons and Angels (1992) }
\label{chp:t2kdetector}

\section{The Tokai-to-Kamioka experiment}
The Tokai-to-Kamioka (T2K) experiment \cite{KANEYUKI2005178} is a long baseline neutrino oscillation experiment based in Japan and its purpose is to study neutrino oscillations: specifically a precision measurement of the neutrino oscillation parameters $\Delta m_{23}^{2}$ and $\sin ^{2} \theta_{23}$ and to improve the measurement of the leptonic CP violating phase $\delta_{CP}$, which is mentioned in Chapter 1. The experiment produces a beam of intense muon neutrinos at J-PARC (Japan Proton Accelerator Research Complex \cite{nagamiya2012introduction}) in Tokai, which is located on the east coast of Japan in Ibaraki Prefecture. The muon neutrino beam travels 295 km west towards the far detector Super-Kamiokande (see Chapter 3.)  The neutrino beam is measured by other detectors such as ND280 \cite{kudenko2009near} and INGRID \cite{abe2012measurements} prior to reaching Super-Kamiokande which is important with regards to measuring the neutrino oscillation parameters. Both ND280 and Super-Kamiokande are off-axis detectors, meaning that they are located 2.5$\degree$ off axis from the centre of the neutrino beam. This results in the peak of the energy of the muon neutrinos to be 0.6 GeV, maximising the neutrino oscillation on the 295 km baseline. This chapter explains production of the muon neutrino beam from the J-PARC proton beam line and the near detector complex.


\subsection{Neutrino beam production}

\subsubsection{J-PARC proton beam production}


A sequence of three accelerators results in the production of a proton beam from the J-PARC accelerator complex, these are the LINAC (linear accelerator), the RCS (rapid cyclic synchrotron) and the main ring synchrotron (MR) \cite{hasegawa2010status}, \cite{sato2018high}. A negative hydrogen ion is accelerated to a kinetic energy of 400 MeV by a linear accelerator (LINAC), from which a beam of protons is created by converting the negative hydrogen ion beam using charge-stripping foils. This proton beam is then accelerated to a kinetic energy of 3 GeV by the RCS, and about 5\% of the bunches produced from this process are passed to the MR where the proton beam will be accelerated up to 30 GeV \cite{hasegawa2017performance}. The rest of the proton bunches are passed to the neutron and muon beamline in the Material and Life Science Facility \cite{higemoto2017materials}. 

\subsubsection{Neutrino beam production}

The neutrino beam is produced using the primary and secondary beamline as shown in Figure \ref{fig:nubeamline}. The primary beamline takes the proton beam from the MR and directs it towards Kamioka. The beam is then transferred through a succession of beam monitors which measure facets of the proton beam including the beam profile, intensity and position. The beam monitor which is closest to the graphite target measures the ``Protons-On-Target'' (POT), a value used to determine the neutrino beam flux. The secondary beamline takes the proton bunches and passes them through a target station, the decay volume and the beam dump \cite{sekiguchi2008neutrino}. After interacting with the target station, the proton bunches are collimated through a 1.7 m graphite rod where the collimated hole the proton bunches pass through are 30 mm in diameter. Beam profile reconstruction occurs in the OTR (Optical Transition Radiation) monitor, made up of titanium alloy foil placed at a 45 degree angle in order to intercept the beam. As the proton beam enters the foil, the visible light that is produced escapes through a collection of mirrors and is then captured by a charge injection device camera, which creates the beam profile \cite{bhadra2013optical}. 
\newline
After beam profiling, the proton beam then impacts upon a graphite rod target which is 91.4 cm long and 2.6 cm in diameter - this collision produces secondary hadrons, including pions which are focused by three magnetic horns (shown in Figures \ref{fig:horn1}, \ref{fig:horn2} and \ref{fig:horn3}). These magnetic horns can be used to produce either a muon neutrino or muon antineutrino beam depending on the polarity of the 250kA current they are pulsed with. If a +250 kA current is used, the positive pions and kaons produced can go on to make muon neutrino beams wheareas if a -250 kA current is used negative pions and kaons can decay to create muon antineutrino beams (both are shown in Equation \ref{eq:nubeam}). The +250 kA mode is called Forward Horn Current (FHC) mode and the -250 kA mode is called Reversed Horn Current (RHC) mode \cite{sekiguchi2008t2k}. The analysis in this thesis will occur in FHC mode only. 


\begin{figure}
    \centering
     \begin{subfigure}[b]{0.33\linewidth}
      \includegraphics[width=\linewidth]{Figures/horn1.PNG}
      \caption{T2K Horn 1.}
      \label{fig:horn1} 
     \end{subfigure}
     \begin{subfigure}[b]{0.33\linewidth}
       \includegraphics[width=\linewidth]{Figures/horn2.PNG}
        \caption{T2K Horn 2.} 
     \label{fig:horn2}
      \end{subfigure} 
      \begin{subfigure}[b]{0.33\linewidth}
        \includegraphics[width=\linewidth]{Figures/horn3.PNG}
         \caption{T2K Horn 3.} 
      \label{fig:horn3}
       \end{subfigure} 
\end{figure}


\begin{figure}
    \centering
    \includegraphics[width=0.7\textwidth]{Figures/nubeamline.png}
    \caption{Schematic of the neutrino beam line.}
        \label{fig:nubeamline}
\end{figure}

\begin{equation}
\begin{array}{lll}
\pi^{+} & \longrightarrow \mu^{+}+\nu_{\mu} & \text { FHC } \\
\pi^{-} & \longrightarrow \mu^{-}+\overline{\nu_{\mu}} & \text { RHC }
\end{array}
\label{eq:nubeam}
\end{equation}

A 75 ton volume beam dump made of graphite and iron stops the particles, specifically the protons, secondary hadrons and mesons which have a momentum below 5 GeV/c, and therefore end up being absorbed by the beam dump. A muon monitor is placed after the beam dump in order to directly measure the the beam intensity and beam direction \cite{matsuoka2010design}. Muons can be used to monitor the beam properties because along with the neutrinos these are the main particles produced from pion decay. After the muon monitor, a nucleon emulsion plate detector measures the flux and momentum of the muons. 

\subsection{Near detectors}

\subsubsection{INGRID detector}

The INGRID (Interactive Neutrino GRID) detector is a neutrino detector which unlike ND280 is placed on-axis instead of off-axis. This allows it to directly monitor the direction and the intensity of the neutrino beam by measuring the interactions of the neutrinos with the alternating iron plates that make it up \cite{otani2008design}. INGRID is also placed 280 m from the graphite target and consists of 14 modules placed in a cross formation, with the centre of the cross placed at the centre of the neutrino beam. The INGRID modules are comprised of nine iron plates alternating with 11 tracking scintillator planes, which are themselves surrounded by scintillator plates the purpose of which is to reject interactions that occur outside the module. A schematic of the INGRID cross is shown in \ref{fig:ingridcross} and a schematic of the modules is shown in \ref{fig:ingridmodule}. 

\begin{figure}
    \centering
    \includegraphics[width=0.7\textwidth]{Figures/ingridcross.png}
    \caption[Schematic of the INGRID cross.]{Schematic of the INGRID cross taken from \cite{t2k_collaboration_t2k_2013}.}
    \label{fig:ingridcross}
\end{figure}
\begin{figure}
    \centering
    \includegraphics[width=0.6\textwidth]{Figures/ingridmodule.png}
    \caption[Individual INGRID module schematic.]{Individual INGRID module schematic taken from \cite{t2k_collaboration_t2k_2013}.}
    \label{fig:ingridmodule}
\end{figure}

An additional module, called the proton module, was added to more precisely measure the neutrino interaction cross-section with the T2K on-axis neutrino beam. This module is used to distinguish the quasi-elastic interaction channel in order to compare it with Monte Carlo simulations of the beamline and neutrino interactions. The Proton Module is made of scintillator planes (no alternating iron plates) and is contained by veto planes. The Proton Module was placed in the centre of the INGRID cross at the intersection of the vertical and horizontal modules \cite{kikawa2012development}. Figure \ref{fig:ingridevent} shows what a standard neutrino event looks like on the INGRID module. A neutrino enters from the left and after interacting with the scintillator cells (shown in green) produces hits (shown in red), with the relative size of each circle corresponding to the observed signal in that cell. The blue cells show the position of the veto scintillators, while the gray planes show the iron plates.   


\begin{figure}
    \centering
    \includegraphics[width=0.7\textwidth]{Figures/ingridevent.png}
    \caption[INGRID event display showing a typical INGRID event.]{INGRID event display showing a typical INGRID event, taken from \cite{t2k_collaboration_t2k_2011}.}
    \label{fig:ingridevent}
\end{figure}

By reconstructing the profile of the beam in x and y directions using the number of neutrino events in seven horizontal and vertical INGRID modules, and fitting the profiles with a Gaussian, the centre of the beam can be defined as the peak of the Gaussian fit. The beam direction can then be reconstructed as the direction from the proton beam target to the reconstructed beam centre. Figure \ref{fig:INGRID_centre} shows these neutrino beam centres from the centre of INGRID, for both the x and y directions, measured in April 2010 \cite{Abe_2012}.


\begin{figure}
    \centering
    \includegraphics[width=0.7\textwidth]{Figures/INGRID_centre.png}
    \caption{Neutrino beam profiles for x (left) and y (right) directions.}
    \label{fig:INGRID_centre}
\end{figure}



\subsubsection{ND280}

ND280 is a near detector which sits 280 m from the target. It is an off-axis detector, meaning that just like Super-Kamiokande, it is placed 2.5$\degree$ off-axis from the center of the beam. This stems from the relationship between the energy of the neutrinos produced from the decay of the pions, a relationship shown in Equation \ref{eq:piondecay}. 

\begin{equation}
    E_{\nu}=\frac{m_{\pi}^{2}-m_{\mu}^{2}}{2\left(E_{\pi}-\sqrt{E_{\pi}^{2}-m_{\pi}^{2}} \cos \theta\right)}
\label{eq:piondecay}
\end{equation}

where $E_{\pi}$ is the energy of the parent pion and $\theta$ is the scattering angle between the direction of the outgoing neutrino and the direction of the parent pion's momentum, $m_{\pi}$ is the parent pion mass and $m_{\mu}$ is the mass of the outgoing muon. Figure \ref{fig:energyangle} shows the neutrino energy from pion decay plotted against the energy of the parent pion for a range of off-axis angles. 

\begin{figure}
\centering
\includegraphics[width=0.7\textwidth]{Figures/energy_angle.PNG}
\caption[Energy of the neutrino plotted against the energy of the parent pion for multiple different off-axis angles.]{Energy of the neutrino plotted against the energy of the parent pion for multiple different off-axis angles, taken from \cite{t2k_collaboration_t2k_2013}.}
\label{fig:energyangle}
\end{figure}

As the off-axis angle increases, the intensity of the neutrino beam decreases, and therefore picking an off-axis angle of 2.5$\degree$ at which to place the ND280 detector complex strikes a good balance between keeping a high beam intensity while ensuring a peak energy of 0.6 GeV in order to have the neutrino oscillation be maximised at 295km. The relationship between muon neutrino oscillation probability and muon neutrino energy is shown at the top in Figure \ref{fig:nuprobosc}, and at the bottom the muon neutrino flux 295 km away from the graphite target can be seen for three different off-axis angles.

\begin{figure}
    \centering
    \includegraphics[width=0.7\textwidth]{Figures/nuprobosc.png}
    \caption[The probability of survival of muon neutrinos (top plot) and neutrino beam flux at the 295km far detector (bottom).]{The probability of survival of muon neutrinos (top plot) and neutrino beam flux at the 295km far detector (bottom). Taken from \cite{t2k_collaboration_t2k_2013}.}
    \label{fig:nuprobosc}
\end{figure}

Figure \ref{fig:ND280_schematic} shows a schematic of the ND280 detector complex. It has five main goals: firstly, to measure the cross sections of muon neutrino interactions, so neutrino-nucleus interaction models can reduce their systematic uncertainties \cite{abe2015measurement}. Secondly, measuring the component of the neutrino beam which is made of electron neutrinos, hence being able to better constrain the background to electron neutrino appearance at the far detector. Thirdly, to predict the event rate at Super-Kamiokande by measuring the energy spectrum of the muon neutrinos produced. It also aims to study non quasi-elastic processes which produce pions below the Super Kamiokande Cherenkov threshold, and to measure the neutral current (NC) pion-nought production rates. 

\begin{figure}
    \centering
    \includegraphics[width=0.7\textwidth]{Figures/nd280_complex.png}
    \caption[Near detector ND280 schematic.]{Near detector ND280 schematic taken from \cite{t2k_collaboration_t2k_2011}.}
    \label{fig:ND280_schematic}
\end{figure}

ND280 is made of a neutral pion detector (P0D), three Time Projection Chambers (TPCs) and two Fine Grained Detectors (FGDs). These are enclosed within Electromagentic Calorimeters (ECals) and a Side Muon Range Detector (SMRD). These detectors are magnetised using a magnet originally used in the UA1 detector at CERN. 

The neutral pion detector is important due to its ability to detect a process that can imitate the signal event of electron neutrino appearance at Super-Kamiokande. Neutral pions are produced during the neutral current interactions on water ($\nu_{\mu} + N \rightarrow \nu_{\mu} + N^{'} +\pi^{0} + X$) and the purpose of $P0D$ is to measure the cross-section of this interaction. 
\newline 
 There are also two electromagnetic calorimeters which remove events entering the detector from the outside. The SMRD (Side Muon Range Detector) is used as a way to measure the momenta of muons which escape the detector complex at a large angle relative to the direction of the beam. 
\newline
There are three time projection chambers placed downstream of the neutral pion detector. These are used for reconstruction of charged particle tracks, particle identification and determination of the momentum of the particle \cite{abgrall2011time}. There are two fine grained detectors (FGDs) in ND280, which have two purposes. Firstly to yield a mass for the neutrino interaction, and secondly, to track the charged particles issuing from the neutrino interaction vertex \cite{amaudruz2012t2k}. Because the far detector, Super-Kamiokande, is a Water Cherenkov detector, the tracker in the ND280 detector needs to measure the neutrino interaction rates with water. This is done by including six water target modules in the second FGD in order to determine neutrino interaction cross-sections with water.
\newline
The ECal is an electromagnetic calorimeter which surrounds the P0D, the TPCs and the FGDs \cite{allan2013electromagnetic}. It consists of plastic scintillator layers sandwiched between lead absorber sheets. Its purpose is to aid in the full event reconstruction through detecting photons and measuring their energy and direction, along with detecting charged particles and getting information which helps in their identification (seperating out electrons, muons and pions.)
