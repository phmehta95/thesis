\chapter{The Tokai-to-Kamioka experiment}
\label{chp:t2kdetector}

The Tokai-to-Kamioka (T2K) experiment is a long baseline neutrino oscillation experiment based in Japan and its purpose is to study neutrino oscillations: specifically a precision measurement of the neutrino oscillation parameters $\Delta m_{23}^{2}$ and $\sin ^{2} \theta_{23}$ and to increase the measurement to the leptonic CP violating phase $\delta_{CP}$, which are mentioned in Chapter 1. The experiment produces a beam of intense muon neutrinos at J-PARC (Japan Proton Accelerator Research Complex) in Tokai, which is located on the far east coast of Japan in Ibaraki Prefecture. The muon neutrino beam travels 295 km west towards the far detector Super-Kamiokande (see Chapter 3.)  These neutrinos are detected by other detectors such as ND280 and INGRID, before they oscillate and reach Super-Kamiokande which is important with regards to measuring the neutrino oscillation parameters. ND280 and Super-Kamiokande are off-axis detectors, meaning that they are placed 2.5$\degree$ off axis to the centre of the neutrino beam - this allows for the peak of the energy of the muon neutrinos to be at 0.6 GeV, meaning that the neutrino oscillation on the 295 km baseline is maximised, and the reduced spread in muon neutrino energy means that these detectors are far less susceptible to potential backgrounds. This chapter explains production of the muon neutrino beam from the JPARC proton beam line and the near detector complex.

\subsection{Neutrino beam production}

\subsubsection{JPARC proton beam production}

The production of a proton beam from JPARC is due to three accelerators, the LINAC (linear accelerator), an RCS (rapid cyclic synchrotron) and the main ring synchrotron (MR). A negative hydrogen ion is accelerated to a kinetic energy of 400 MeV, from which a beam of protons is created by converting the negative hydrogen ion beam using charge-stripping foils. This proton beam is then accelerated to a kinetic energy of 3 GeV by the RCS, and about 5\% of the bunches produced from this process are passed to the MR where the proton beam will be accelerated up to 30 GeV. 

\subsubsection{Neutrino beam production}

The neutrino beam is produced using a primary and a secondary beamline as shown in Figure \ref{fig:nubeamline}. The primary beamline involves taking the proton beam from the MR and targeting it towards the direction of Kamioka and then transferring it through a succession of beam monitors which measure facets of the neutrino beam including the beam profile, intensity and position. The beam monitor which is closest to the graphite target measures the "Protons-On-Target" (POT), a value used to determine the neutrino beam flux. The secondary beamline involves taking the proton bunches and passing them through a target station, the decay volume and the beam dump. After interacting with the target station, the proton bunches are collimated through a 1.7 m graphite rod where the collimated hole the proton bunches pass through are 30 mm in diameter. Beam profile reconstruction occurs in the OTR (Optical Transition Radiation) monitor, made up of titanium alloy foil placed at a 45 degree angle in order to intercept the beam. As the proton beam enters the foil, the visible light that is produced escapes through a collection of mirrors and is then captured by a charge injection device camera, which creates the beam profile. After beam profiling, the proton beam then impacts upon a graphite rod target which is 91.4 cm long and 2.6 cm in diameter - this collision produces secondary hadrons, including pions which are focused by three magnetic horns. These magnetic horns can be used to produce either a muon neutrino or muon antineutrino beam depending on the polarity of the 250kA current they are pulsed with. If a +250 kA current is used, the positive pions and kaons produced can go on to make muon neutrino beams wheareas if a -250 kA current is used negative pions and kaons can decay to create muon antineutrino beams (both are shown in Equation \ref{eq:nubeam}). The +250 kA mode is called Forward Horn Current (FHC) mode and the -250 kA mode is called Reversed Horn Current (RHC) mode, and the analysis in this thesis will occur in FHC mode only. 

\begin{equation}
\begin{array}{lll}
\pi^{+} & \longrightarrow \mu^{+}+\nu_{\mu} & \text { FHC } \\
\pi^{-} & \longrightarrow \mu^{-}+\overline{\nu_{\mu}} & \text { RHC }
\end{array}
\label{eq:nubeam}
\end{equation}

A 75 ton volume beam dump made of graphite and iron stops the particles, specifically the protons, secondary hadrons and mesons which have a momentum below 5 GeV/c end up being absorbed by the beam dump. A muon monitor is placed after the beam dump in order to directly measure the the beam intensity and beam direction - muons can be used to monitor the beam properties because along with the neutrinos these are the main particles produced from the pion decay. After the muon monitor, a nucleon emulsion plate detector measures the flux and momentum of the muons. 



