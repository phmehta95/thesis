\chapter{Introduction}
\label{chp:intro}

\section{Introduction}

\subsection{Neutrino Physics}

There are a plethora of physics phenomenon in which neutrinos are involved, including beta decay, cosmic rays, and supernovae. As part of the Standard Model, they are descibed as being Dirac fermions with no electric charge with three flavours: the electron neutrino, the muon neutrino and the tau neutrino, corresponding with their associated leptons: the electron, muon and tauon. Prior to the discovery of neutrino oscillations, it was believed that neutrinos were massless, but they in fact have small but non zero masses (<1eV). The next subsections of this chapter will discuss a brief history of neutrino physics including the discovery of neutrino oscillations, the manner in which neutrinos interact with nuclei in the Super-Kamiokande detector, and the motivation behind an NCQE neutron tagging analysis.

\subsection{History of Neutrino Physics}

To correct a violation of energy conservation discovered in beta-decay, Wolfgang Pauli put forward the idea of a neutrino (Italian for "little neutral one") as a solution. In 1934, Enrico Fermi's theory of beta decay stated that a neutron could decay to a proton, electron and an antielectron neutrino and in 1956, Clyde Cowan and Frederick Reines directly confirmed the existence of neutrinos, by detetecting the electron antineutrino originating from inverse beta decay produced from a nuclear reactor. In inverse beta decay (shown in Equation \ref{eq:IBD_eq}), an electron antineutrino interacts with a proton to produce a neutron and positrons. These positrons pair-annihilate with electrons and produce two 0.5 MeV gamma rays which go in opposite directions. Scintillator material was placed in a tank of water, which was used to detect the gamma photons, and the scintillator light produced flashes of visible light which were detected by photomultiplier tubes. Cadmium chloride was used to detect the coincident neutron, which after exciting and de-exciting, emits a gamma ray within 5 microseconds after the pair-annihilation gammas are detected. In 1962, Ledermen, Schwartz and Steinberger detecting the muon neutrino, and in 2000 the DONUT collaboration at Fermilab detected the existence of the tau neutrino. 

In the 1960s, the Homestake experiment made the first measurement of solar electron neutrino flux. These are produced by nuclear fusion using the proton-proton chain reaction, where four protons are converted into neutrinos, alpha particles, positrons and energy. The experiment used a perchloroethlene-based detector, placed 1,478 metres underground in the Homestake Gold Mine in South Dakota. When an electron neutrino interacts with 37Cl in the perchloroethlene, the 37Cl becomes a radioactive isotope of 37Ar which are extracted by bubbling helium through the tank, and then counted in order to determine how many neutrinos had been captured. The theoretical solar neutrino flux calculated by John Bachall was about three times as much as Raymond Davis' results from the experiment: with Bachall's calculations predicting a solar neutrino flux of $7.9 \pm 2.6$ SNU, whereas the Homestake experimental results showed a flux of $2.1 \pm 0.3$ SNU. These results were consistent with those from the Kamiokande, SAGE and GALLEX experiments.
\newline
The solution to this problem came from the Super-Kamiokande and Sudbury Neutrino Observatory experiments. In 1998, Super Kamiokande showed evidence of neutrino oscillation: where muon neutrinos produced by cosmic rays in the upper atmosphere changed into tau neutrinos within the Earth, pointing to the fact that the deficit in the solar neutrino flux oberved at Homestake, SAGE and GALLEX had to do with neutrino oscillation. In 2001 SNO observed the flux of electron neutrinos but also the flux of all flavours of neutrinos, and found that the fraction of electron neutrinos was found to be 34\%, perfectly concordant with the prediction.

\subsection{Neutrino Oscillation}
In 1957 Bruno Pontecorvo postulated that neutrinos could transition from neutrinos to antineutrinos and vice versa (similarly to how two kinds of neutral kaons ($\bar{K_{0}}$, and $K_{0}$ were found to oscillate.) Neutrino flavour oscillation theory was then developed by Maki, Nakagawa and Sakata in 1962. The PMNS matrix (Pontecorvo-Maki-Nakagawa-Sakata matrix), the neutrino analogue of the Cabbibo-Kobayashi-Masakawa quark mixing matrix. 
Equation \ref{eq:neutrino_osc} shows the relationship between the mass and flavour eigenstates for a neutrino with a definite flavour of $\alpha$ and a definite mass of $m_{i}$.

$$
\begin{aligned}
&\left|\nu_{\alpha}\right\rangle=\sum_{i} U_{\alpha i}^{*}\left|\nu_{i}\right\rangle \\
&\left|\nu_{i}\right\rangle=\sum_{\alpha} U_{\alpha i}\left|\nu_{\alpha}\right\rangle
\end{aligned}
\label{eq:neutrino_osc}
$$


In Equation \ref{eq:neutrino_osc}, the terms $U_{\alpha i}^{*}$ and $U_{\alpha i}$ are the complex conjugate and normal PMNS matrix. Equation \ref{eq:PMNS_matrix} shows the 3x3 form of the PMNS matrix, where $c_{ij} = cos {\theta_{ij}}$ and $s_{ij} = sin {\theta_{ij}}$.

$$
\begin{aligned}
&U=\left(\begin{array}{ccc}
1 & 0 & 0 \\
0 & c_{23} & s_{23} \\
0 & -s_{23} & c_{23}
\end{array}\right)\left(\begin{array}{ccc}
c_{13} & 0 & s_{13} e^{-i \delta_{\mathrm{CP}}} \\
0 & 1 & 0 \\
-s_{13} e^{i \delta_{\mathrm{CP}}} & 0 & c_{13}
\end{array}\right)\left(\begin{array}{ccc}
c_{12} & s_{12} & 0 \\
-s_{12} & c_{12} & 0 \\
0 & 0 & 1
\end{array}\right)\\
&=\left(\begin{array}{ccc}
c_{12} s_{13} & s_{12} c_{13} & s_{13} e^{-i \delta_{\mathrm{CP}}} \\
-s_{12} c_{23}-c_{12} s_{13} s_{23} e^{i \delta_{\mathrm{CP}}} & c_{12} c_{23}-s_{12} s_{13} s_{23} e^{i \delta_{\mathrm{CP}}} & c_{13} s_{23} \\
s_{12} s_{23}-c_{12} s_{13} c_{23} e^{i \delta_{\mathrm{CP}}} & c_{12} s_{23}-s_{12} s_{13} c_{23} e^{i \delta_{\mathrm{CP}}} & c_{13} c_{23}
\end{array}\right),
\end{aligned}
\label{eq:PMNS_matrix}
$$

In Equation \ref{eq:PMNS_matrix}, if the sin $\delta_{CP}$ terms are not equal to 0, it means that there will be imaginary terms in the matrix, which will contribute to CP violation. 


\subsection{Neutrino-nucleus interactions in Super-Kamiokande Gd}

\subsection{Supernova Relic Neutrinos}

\subsection{Analysis Motivation}




