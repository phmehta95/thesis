\chapter{Introduction}
\label{chp:intro}

\section{Introduction}

\subsection{Neutrino Physics}

There are a plethora of physics phenomenon in which neutrinos are involved, including beta decay, cosmic rays, and supernovae. As part of the Standard Model, they are descibed as being Dirac fermions with no electric charge with three flavours: the electron neutrino, the muon neutrino and the tau neutrino, corresponding with their associated leptons: the electron, muon and tauon. Prior to the discovery of neutrino oscillations, it was believed that neutrinos were massless, but they in fact have small but non zero masses (<1eV). The next subsections of this chapter will discuss a brief history of neutrino physics including the discovery of neutrino oscillations, the manner in which neutrinos interact with nuclei in the Super-Kamiokande detector, and the motivation behind an NCQE neutron tagging analysis.

\subsubsection{History of Neutrino Physics}

To correct a violation of energy conservation discovered in beta-decay, Wolfgang Pauli put forward the idea of a neutrino (Italian for "little neutral one") as a solution. In 1934, Enrico Fermi's theory of beta decay stated that a neutron could decay to a proton, electron and an antielectron neutrino and in 1956, Clyde Cowan and Frederick Reines directly confirmed the existence of neutrinos, by detetecting the electron antineutrino originating from inverse beta decay produced from a nuclear reactor. In inverse beta decay 

\subsubsection{Neutrino-nucleus interactions in Super-Kamiokande Gd}

\subsubsection{Supernova Relic Neutrinos}

\subsection{Analysis Motivation}




