\chapter{Introduction}
\label{chp:intro}

\section{Introduction}

\subsection{Neutrino Physics}

There are a plethora of physics phenomenon in which neutrinos are involved, including beta decay, cosmic rays, and supernovae. As part of the Standard Model, they are descibed as being Dirac fermions with no electric charge with three flavours: the electron neutrino, the muon neutrino and the tau neutrino, corresponding with their associated leptons: the electron, muon and tauon. Prior to the discovery of neutrino oscillations, it was believed that neutrinos were massless, but they in fact have small but non zero masses ($\le$ 1eV). The next subsections of this chapter will discuss a brief history of neutrino physics including the discovery of neutrino oscillations, the manner in which neutrinos interact with nuclei in the Super-Kamiokande detector, and the motivation behind an NCQE neutron tagging analysis.

\subsection{History of Neutrino Physics}

To correct a violation of energy conservation discovered in beta-decay, Wolfgang Pauli put forward the idea of a neutrino (Italian for "little neutral one") as a solution \cite{brown_idea_1978}. In 1934, Enrico Fermi's theory of beta decay stated that a neutron could decay to a proton, electron and an antielectron neutrino and in 1956 \cite{luca_electroweak}, Clyde Cowan and Frederick Reines directly confirmed the existence of neutrinos \cite{cowan_reines}, by detetecting the electron antineutrino originating from inverse beta decay produced from a nuclear reactor. In inverse beta decay an electron antineutrino interacts with a proton to produce a neutron and positrons. These positrons pair-annihilate with electrons and produce two 0.5 MeV gamma rays which go in opposite directions. Scintillator material was placed in a tank of water, which was used to detect the gamma photons, and the scintillator light produced flashes of visible light which were detected by photomultiplier tubes. Cadmium chloride was used to detect the coincident neutron, which after exciting and de-exciting, emits a gamma ray within 5 microseconds after the pair-annihilation gammas are detected. In 1962, Ledermen, Schwartz and Steinberger detecting the muon neutrino, and in 2000 the DONUT collaboration at Fermilab detected the existence of the tau neutrino \cite{lederman_schwartz}.

In the 1960s, the Homestake experiment made the first measurement of solar electron neutrino flux \cite{homestake_davis}. These are produced by nuclear fusion using the proton-proton chain reaction, where four protons are converted into neutrinos, alpha particles, positrons and energy. The experiment used a perchloroethlene-based detector, placed 1,478 metres underground in the Homestake Gold Mine in South Dakota. When an electron neutrino interacts with 37Cl in the perchloroethlene, the 37Cl becomes a radioactive isotope of 37Ar which are extracted by bubbling helium through the tank, and then counted in order to determine how many neutrinos had been captured. The theoretical solar neutrino flux calculated by John Bachall was about three times as much as Raymond Davis' results from the experiment: with Bachall's calculations predicting a solar neutrino flux of $7.9 \pm 2.6$ SNU, whereas the Homestake experimental results showed a flux of $2.1 \pm 0.3$ SNU. These results were consistent with those from the Kamiokande, SAGE and GALLEX experiments.
\newline
The solution to this problem came from the Super-Kamiokande and Sudbury Neutrino Observatory experiments. In 1998, Super Kamiokande showed evidence of neutrino oscillation \cite{fukuda_1998}: where muon neutrinos produced by cosmic rays in the upper atmosphere changed into tau neutrinos within the Earth, pointing to the fact that the deficit in the solar neutrino flux oberved at Homestake, SAGE and GALLEX had to do with neutrino oscillation. In 2001 SNO observed the flux of electron neutrinos but also the flux of all flavours of neutrinos, and found that the fraction of electron neutrinos was found to be 34\%, perfectly concordant with the prediction \cite{sno_2001}.

\subsection{Neutrino Oscillation}
In 1957 Bruno Pontecorvo postulated that neutrinos could transition from neutrinos to antineutrinos and vice versa (similarly to how two kinds of neutral kaons $\bar{K_{0}}$, and $K_{0}$ were found to oscillate) \cite{Pontecorvo:1957cp}. Neutrino flavour oscillation theory was then developed by Maki, Nakagawa and Sakata in 1962. The PMNS matrix (Pontecorvo-Maki-Nakagawa-Sakata matrix), the neutrino analogue of the Cabbibo-Kobayashi-Masakawa quark mixing matrix \cite{maki_pmns}. 
Equation \ref{eq:neutrino_osc} shows the relationship between the mass and flavour eigenstates for a neutrino with a definite flavour of $\alpha$ and a definite mass of $m_{i}$.

$$
\begin{aligned}
&\left|\nu_{\alpha}\right\rangle=\sum_{i} U_{\alpha i}^{*}\left|\nu_{i}\right\rangle \\
&\left|\nu_{i}\right\rangle=\sum_{\alpha} U_{\alpha i}\left|\nu_{\alpha}\right\rangle
\end{aligned}
\label{eq:neutrino_osc}
$$


In Equation \ref{eq:neutrino_osc}, the terms $U_{\alpha i}^{*}$ and $U_{\alpha i}$ are the complex conjugate and normal PMNS matrix. Equation \ref{eq:PMNS_matrix} shows the 3x3 form of the PMNS matrix, where $c_{ij} = cos {\theta_{ij}}$ and $s_{ij} = sin {\theta_{ij}}$.

$$
\begin{aligned}
&U=\left(\begin{array}{ccc}
1 & 0 & 0 \\
0 & c_{23} & s_{23} \\
0 & -s_{23} & c_{23}
\end{array}\right)\left(\begin{array}{ccc}
c_{13} & 0 & s_{13} e^{-i \delta_{\mathrm{CP}}} \\
0 & 1 & 0 \\
-s_{13} e^{i \delta_{\mathrm{CP}}} & 0 & c_{13}
\end{array}\right)\left(\begin{array}{ccc}
c_{12} & s_{12} & 0 \\
-s_{12} & c_{12} & 0 \\
0 & 0 & 1
\end{array}\right)\\
&=\left(\begin{array}{ccc}
c_{12} s_{13} & s_{12} c_{13} & s_{13} e^{-i \delta_{\mathrm{CP}}} \\
-s_{12} c_{23}-c_{12} s_{13} s_{23} e^{i \delta_{\mathrm{CP}}} & c_{12} c_{23}-s_{12} s_{13} s_{23} e^{i \delta_{\mathrm{CP}}} & c_{13} s_{23} \\
s_{12} s_{23}-c_{12} s_{13} c_{23} e^{i \delta_{\mathrm{CP}}} & c_{12} s_{23}-s_{12} s_{13} c_{23} e^{i \delta_{\mathrm{CP}}} & c_{13} c_{23}
\end{array}\right),
\end{aligned}
\label{eq:PMNS_matrix}
$$

In Equation \ref{eq:PMNS_matrix}, if the sin $\delta_{CP}$ terms are not equal to 0, it means that there will be imaginary terms in the matrix, which will contribute to CP (charge-parity) violation. The angles $\theta_{12}$, $\theta_{23}$ and $\theta_{13}$ are mixing angles. An additional 3 x 3 matrix term is present if neutrinos are proven to be their own antiparticle (Majorana particles): it has not been proven whether neutrinos are Majorana or not. Equation \ref{eq:maj_nu} shows this extra term, where the two Majorana CP-violating phases are given as ($\alpha_{21}$, $\alpha_{31}$). 

$$
\left(\begin{array}{ccc}
1 & 0 & 0 \\
0 & e^{i \lambda_{21}} & 0 \\
0 & 0 & e^{i \lambda_{31}}
\end{array}\right)
\label{eq:maj_nu}
$$


\subsection{Neutrino-nucleus interactions in Super-Kamiokande Gd}
Understanding neutrino interaction modes, and understanding neutrino nucleus interaction modes, in particular the neutral current quasielastic reaction is key to understanding this analysis. 

There are two main types of neutrino interaction: charged-current (CC) and neutral current (NC). The former occurs when a W $\pm$ boson is used in a nuclear exchange, and the latter occurs when a $Z^{0}$ is used (see Figure \ref{fig:CC_NC}).

\begin{figure}
    \includegraphics[width=\textwidth]{Figures/CC_NC.png}
    \caption{Feynman diagrams of charged-current (left) and neutral-current (right) neutrino interactions}
    \label{fig:CC_NC}
\end{figure}

There are multiple types of neutrino-nucleus interactions that occur which can be either charged current or neutral current interactions, or both. 

One such interaction, and often the simplest interaction is elastic scattering ($$
\nu+e^{-} \rightarrow \nu+e^{-}
$$), which occurs when a neutrino scatters off an electron with a virtual vector boson being exchanged. This type of scattering is used in the detection of low energy neutrinos, primarily those from the sun. Figure \ref{fig:elastic_scattering} shows the Feynman diagrams for this kind of interaction.

\begin{figure}
    \includegraphics[width=\textwidth]{Figures/elastic_scattering.png}
    \caption{Feynman diagrams of neutral-current (left) and charged-current (right) neutrino-electron scattering}
    \label{fig:elastic_scattering}
\end{figure}

Single mesons can also be produced via neutrino-nucleon reactions: these are mostly pions, but also some kaons and eta particles can also be produced. Here a neutrino with a high enough energy, interacts with and excites a nucleon, producing a resonant baryon which decays to a nucleon and a single pion (shown in Equation \ref{eq:resonant_pion}), where $N$ and $N'$ are nucleons.

$$
\begin{gathered}
\nu_{l}+N \rightarrow l+N^{*} \\
N^{*} \rightarrow \pi+N^{\prime}
\end{gathered}
\label{eq:resonant_pion}
$$

The resonant baryon produced during the reaction is usually a $\Delta(1232)$ resonance. 

Single pion final sates can also be priduced by a neutrino which scatters an entire nucleus (X), shown in Equation \ref{eq:single_pion_CC} for the charged current reactions and Equation \ref{eq:single_pion_NC} for neutral current reactions. 

\begin{equation}
\nu_{l}+X \rightarrow l^{-}+X+\pi^{+}, \quad \bar{\nu}_{l}+X \rightarrow l^{+}+X+\pi^{-}
\label{eq:single_pion_CC}
\end{equation}

\begin{equation}
\nu_{l}+X \rightarrow \nu_{l}+X+\pi^{0}, \quad \bar{\nu}_{l}+X \rightarrow \bar{\nu}_{l}+X+\pi^{0}
\label{eq:single_pion_NC}
\end{equation}

At higher energies (above 1 GeV), neutrino interactions can also produce kaons in the final state, due to the higher energies being able to produce strange quarks. 

Deep inelastic scattering is a type of neutrino interaction where the neutrino scatters off a quark inside the proton or neutron involved in the exchange, via a W (CC) or Z (NC) boson (Equation \ref{eq:DIS_eq}), shown in Figure \ref{fig:CC_DIS}.

$$
\begin{aligned}
&\nu_{l}+N \rightarrow l^{-}+X, \quad \bar{\nu}_{l}+N \rightarrow l^{+}+X(\mathrm{CC}) \\
&\nu_{l}+N \rightarrow \nu_{l}+X, \quad \bar{\nu}_{l}+N \rightarrow \bar{\nu}_{l}+X(\mathrm{NC})
\end{aligned}
\label{eq:DIS_eq}
$$

\begin{figure}
    \includegraphics[width=\textwidth]{Figures/CC_DIS.png}
    \caption{Feynman diagram for a charged current deep inelastic scattering interaction with an incoming muon neutrino}
    \label{fig:CC_DIS}
\end{figure}


The next two neutrino-nucleus interactions explained here are of particular relevance to the analysis in this thesis: inverse beta decay (IBD) and quasi-elastic scattering. 

Inverse beta decay is the reaction by which Cowan and Reines first detected electron antineutrinos: it is important at low energies: from the minimum energy for the reaction to take place ($E_{\nu}$ = 1.806 MeV) to tens of MeV. Diffuse Supernova Neutrino Background and low energy antineutrinos produced from nuclear reactors can be detected via this process. Figure \ref{fig:IBD_feynman} shows the Feynman diagram for this reaction. The neutron produced by this reaction is integral to the motivation behind the Gadolinium-doping upgrade to Super-Kamiokande, which will be explained in the next section.

\begin{figure}
    \includegraphics[width=\textwidth]{Figures/IBD_feynman.png}
    \caption{Feynman diagram for the inverse beta decay reaction}
    \label{fig:IBD_feynman}
\end{figure}

Finally, we get to the type of interaction investigated in this thesis: quasi-elastic scattering. This makes up the majority of the neutrino-nucleus interaction cross-sections at the energy range from 100 MeV to ~2 Gev, and therefore vital for the study of neutrinos from long baseline neutrino experiments and also low energy atmospheric neutrinos. Equation \ref{eq:QE_reaction} shows the equations for both the charge current (CCQE) and neutral current (NCQE) version of this interaction where the incoming neutrino scatters off a nucleon.

$$
\begin{aligned}
\mathrm{CC}: \nu(k)+n(p) & \rightarrow l^{-}\left(k^{\prime}\right)+p\left(p^{\prime}\right) \\
\bar{\nu}(k)+p(p) & \rightarrow l^{+}\left(k^{\prime}\right)+n\left(p^{\prime}\right) \\
\mathrm{NC}: \nu(k)+N(p) & \rightarrow \quad \nu\left(k^{\prime}\right)+N\left(p^{\prime}\right) \\
\bar{\nu}(k)+N(p) & \rightarrow \bar{\nu}\left(k^{\prime}\right)+N\left(p^{\prime}\right)
\end{aligned}
\label{eq:QE_reaction}
$$

\begin{figure}
    \includegraphics[width=\textwidth]{Figures/QE_feynman.png}
    \caption{Feynman diagram for a quasi elastic scattering interaction off a nucleon}
    \label{fig:QE_reaction}
\end{figure}


\subsection{Supernova Relic Neutrinos}

A key feature of the analysis presented in this thesis is that it is an investigation into the significant background of the signal for SRN. It is therefore important to state and understand the process behind the production of supernova relic neutrinos in order to get a firm handle on the motivation behind this analysis. 

\subsubsection{Supernovae Classification}
Supernovae occur when a star with a mass around eight times the mass of our sun exploads, and in a galaxy these occur only a few times in a century. Supernovae are classified into different types: Type 1a, Type 1b, Type 1c anf Type II. The classification of supernovae are determined by looking at the spectral lines in the light emitted from these supernovae. Table \ref{table:supernova_classification} shows how these supernovae are classed and which spectral line elements are associated with each class. 

\begin{center}
\begin{tabular}{||c c||} 
    \hline
    Supernova Classification & Element lines present in spectra \\ 
    \hline \hline
    Type 1a & No hydrogen, silicon  \\ 
    \hline
    Type 1b & No hydrogen, no silicon, helium  \\
    \hline
    Type 1c & No hydrogen, no silicon, no helium  \\
    \hline
    Type II & Hydrogen  \\
    \hline \hline
\end{tabular}
\label{table:supernova_classification}
\end{center}

The kinetic energy of a supernova is ~$10^{44}$ J, and 99 \% of the energy from core-collapse supernovae (CCSN) are released in the the form of neutrinos. Unline Type 1a supernovae which are usually thermonuclear supernovae, Type 1b, Type 1c and Type II are core-collapse supernovae, from which more neutrinos are emitted which is why these types of supernovae are of more interest. 

\subsubsection{Core-Collapse Supernovae Mechanism}

Using the pressure produced by the process of nuclear fusion, a star is able to support itself against gravotational collapse. During the proton-proton chain reaction, hydrogen will fuse to produce helium and once temperatures and pressures are high enough, helium fusion will occur. AFter all the helium in the core is used up in the fusion process, the star will contract until the pressure and temperatures get even higher, allowing more massive nuclei to fuse. This reaction will carry on until iron nuclei are produced, this being the element with the highest binding energy, causing the fusion to stop.
\newline
As more and more iron accumulates in the core of the star. the density and temperature of the core will increase, and these higher energy electrons will increase the rate of electron capture on protons that will occur in the iron nuclei. This will cause a reduction in the electron degeneracy pressure, which is further enhanced by the breakdown of the iron nuclei which occurs at higher temperatues when gamma rays interact with them. The degeneracy pressure is no longer greater than the gravitational forces acting inwards, and gravitational core collapse occurs. The point at which in the core the neutrino mean free paths become become approximately the same size as the proto-neutron star is called the "neutrinosphere".  The radius of the neutrinosphere becomes as large as that of the inner core of the star, when the inner core of the star reaches a density of ~$10^{11} gcm^{-3}$, and the electron neutrinos produced from electron capture become unable to escape. Gravitational collapse of the star continues until the inner core reaches nuclear density, at which point a shock wave is produced due to the repulsive force between nuclei. When this shock wave reaches the neutrinosphere, neutrino emission begins, which lasts less than 10 milliseconds. After this shockwave passes, nucleons and electrons fall back onto the proto-neutron star which heats it up. This causes neutrinos of all flavours to be produced via pair production and electron capture. This is called the "accretion phase". Due to this expulsion of neutrinos the shock wave loses energy, but it is revived through matter behind the shock wave being heated by neutrino absorption from the proto neutron star region. After the shock wave is revived, if it has enough energy to blow off the outer layer of matter a supernova occurs. Then, depending on the mass of the PNS, it cools to either become a neutron star or a black hole. If the shock wave energy is not high enough to blow off the outer layer of matter, the accretion phase continues until a black hole is formed.

The energy of the emitted neutrinos depend on their flavour - neutrinos emitted from a deeper layer inside the supernova will be higher in temperature and therefore have a higher energy. For electron neutrinos and electron anti-neutrinos. the dominant interactions are charged current interactions with nucleons. Due to the number of neutrons in a proto-neutron star greatly outnumbering the number of protons, the interaction involving $\nu_{e}$ will be far more efficient than those involving $\nu_{\bar{e}}$, meaning the neutrinosphere for $\nu_{\bar{e}}$ is smaller than for $\nu_{e}$, so they are emitted from the PNS with greater energies. 

Due to there being no charged current interactions involving muon neutrinos in the medium, they involved instead in neutral current reactions, including Bremsstrahlung, neutrino-pair annhilation, and electron-positron pair annhilation. Due to only undergoing these reactions their neutrinosphere is even smaller than that of their electron neutrino and electron anti-neutrino counterparts, therefore being emitted wih even higher energies \cite{nagakura_non-thermal_2021}. Figure \ref{fig:ccsn_nu_flavor_energy} shows the luminosity (top panels) and average energy (bottom panels) for three different neutrino flavours as a function of time for the neutronisation phase (left), accretion phase (centre) and cooling phase (right). Taken from \cite{chakraborty_observing_2014}.

\begin{figure}
    \includegraphics[width=\textwidth]{Figures/ccsn_nu_flavor_energy}
    \caption{Luminosity (top panels) and average energy (bottom panels) for three different neutrino flavours as a function of time for the neutronisation phase (left), accretion phase (centre) and cooling phase (right). Taken from \cite{chakraborty_observing_2014}. }
    \label{fig:ccsn_nu_flavor_energy}
\end{figure}



Kamiokande-II observed neutrinos produced from a supernova, later named 1987A in the Large Magellenic Cloud. These neutrinos were also observed by the Irvine-Michigan-Brookhaven detector and the Baskan Neutrino Observatory. \cite{hirata_observation_1987}.

Neutrinos are the most important signal a supernova can produce, and are more important for finding out when the supernova explosion actually happened - due to neutrinos leaving the supernovae immediately while gamma rays from the the explosion can only leave the supernova when the shock wave reaches the surface of the star.\cite{bethe_supernova_1990}.

The supernova relic neutrinos emitted from all past CCSN are accumulated and form an everpresent background due to the fact that neutrinos only weakly interact with matter. The total flux of this neutrino background has been theoretically predicted and is called the "diffuse supernova neutrino background" (DSNB). When calculating the total SRN flux, the redshift caused by the expansion of the universe needs to be taken into account.

Equation \ref{eq:SRNflux} gives the differential of the supernova relic neutrino flux with respect to the SRN energy at Earth.

\begin{equation}
\mathrm{d} n_{\nu}\left(E_{\nu}\right)=R_{\mathrm{SN}}(z) \frac{\mathrm{d} t}{\mathrm{~d} z} \mathrm{~d} z \frac{\mathrm{d} N_{\nu}\left(E_{\nu}^{\prime}\right)}{\mathrm{d} E_{\nu}^{\prime}}(1+z) \mathrm{d} E_{\nu}
\label{eq:SRNflux}
\end{equation}

Here $E_{\nu}^{\prime}=(1+z) E_{\nu}$ is the energy of neutrinos at a certain redshift $z$, $R_{\mathrm{SN}}(z)$ is the supernova rate at a comoving volume at $z$, $\mathrm{d} N_{\nu} / \mathrm{d} E_{\nu}$ is the number spectrum of neutrinos emitted by one supernova explosion, and $(1+z)^{-3}$ is the expansion of the universe factor. $dt/dz$ is the relationship between redshift and time, given by the Friedmann equation (Equation \ref{eq:friedmann}), where $H_{0}$ is the Hubble constant, $\Omega_{m}$ is the matter density and $\Omega_{\Gamma}$ is the cosmological constant \cite{ando_relic_2004}.

\begin{equation}
\frac{\mathrm{d} z}{\mathrm{~d} t}=-H_{0}(1+z) \sqrt{\Omega_{\mathrm{m}}(1+z)^{3}+\Omega_{\Lambda}}
\label{eq:friedmann}
\end{equation}



\subsection{Analysis Motivation}

As mentioned in prior sections, the main background to SRN search are NCQE interactions of atmospheric neutrinos, where Super-Kamioande's sensitivity to this seacrh is limited by the large uncertainty of this interaction, particularly in the low energy region where the SRN flux is assumed to be large. By measuring the NCQE interaction using T2K beam neutrinos, this uncertainty can be reduced due to the reason that the atmospheric neutrino energy region is similar to that of the T2K beam energy region. The differential cross section of the neutral-current elastic interaction of a neutrino on a free proton or neutron can be written as shown in Equation \ref{NCQExsec}. 

\begin{equation}
\frac{d \sigma^{\nu(\bar{\nu})}}{d q^{2}}=\frac{M^{2} G_{F}^{2}}{8 \pi E_{\nu}^{2}}\left\{A\left(q^{2}\right) \pm B\left(q^{2}\right) \frac{s-u}{M^{2}}+c\left(q^{2}\right) \frac{(s-u)^{2}}{M^{4}}\right\}
\label{NCQExsec}
\end{equation}
    
where $E_{\nu}$ is the initial neutrino or antineutrino energy, $M$ is the mass of the nucleon, $G_{F}$ is the Fermi coupling constant, $s$ and $u$ are the Mandelstam variables, and $q$ is the transfer of four momentum between between incoming (anti)neutrinos and outgoing (anti)neutrinos. Depending on whether it is an incoming neutrino or antineutrino, there is either a plus or a minus between the A and B terms. A,B and C are terms which are made up of form factors specific to neutral current interactions. The ideal goal of this analysis is to complement prior analyses which investigate the NCQE cross-section and analyses which produce neutron multiplicity distributions for NCQE interactions, with the added benefit of comparisons to Run 11 T2K data, which brings with it the first beam events with gadolinium in the far detector. The emphasis on the distributions of neutrons in this thesis and the methods of neutron tagging used are of increasing interest due to the fact that the main purpose behind adding Gadolinium to the detector is to make these neutron captures more visible. 


