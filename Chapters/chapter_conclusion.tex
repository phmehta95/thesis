\chapter{Conclusion}
\epigraph{``Time may be eternal, Captain, but our patience is not. It's time to put an end to your trek through the stars, make room for other more worthy species.''}{Q, Star Trek: TNG S7E26 All Good Things... (1994)}
\label{chp:conc}

The work presented in this thesis has two distinct parts: one is the calibration work centered around the Monte Carlo simulations of the UK Light Injection System optics and the other part is my work with neutron tagging for the Super Kamiokande Gadolinium Upgrade and the use of the neutron tagging efficiency to estimate the DSNB NCQE background error. 
\newline
The UK Light Injection system collimator and diffuser optics were modelled by first measuring the angular distributions of the light intensity. The cumulative distribution profiles of these outputs were sampled inside the Super Kamiokande Detector Simulator (SKDETSIM) software in order to produce Monte Carlo with the same light profiles. These Monte Carlo were verified by producing occupancy plots to visibly compare the beam spots for both types of optics against actual light injection data, and in addition to this unifrom distributions were passed through the inverse CDF functions used in the simulation and these produced values which were very close to the probability distribution functions of the original light profiles. After ensuring that the angular distributions of the light profiles were consistent with the light profiles in data, efforts were made to ensure the distributions of the time of the hits were also consistent. By adjusting the time dispersion of the injected photons in the Monte Carlo, the hit times of the scattered and reflected hits could be shifted, and various amounts of time dispersion were implemented into SKDETSIM using the Box-Muller transform to see what value of time dispersion would alter the track-step of the photon enough to be concurrent with time-of-flight distributions of the data. After a $\chi^{2}$ comparison between Monte Carlo and data timing distributions, a time dispersion value of 10ns was seen to be the best choice.
\newline
To estimate the DSNB NCQE background, neutrino flux is propoagated through event simulation, event reconstruction and neutron tagging to end up with a neutron tagging efficiency. NEUT and SKDETSIM-SKGd were used to model the interaction between the neutrino and ${ }^{16} \mathrm{O}$, and different neutron capture on gadolinium models in SKDETSIM were compared by looking at their BONSAI reconstruction variables. Comparisons of these variables between two neutron tagging algorithms were carried out, in order to ensure that no major differences were found between legacy and new NTag code. NCQE events were selected using defined criteria and comparisons were made between distributions of NCQE events for prior analyses with no gadolinium in the simulation. Distributions of truth neutron capture number, time and the associated number and energy of gamma rays were produced for both the legacy and new NTag codes, as well as the distribution of truth neutrons from events which passed the NCQE selection criteria. With regards to improving the neutron tagging algorithm, primary selection criteria were defined based on hit time and the number of hits in a 14 nsec window. There was a marked improvement in the neutron capture vertex resolution of 15.8\% compared to prior analyses with neutron captures occuring solely on hydrogen. An artifical neural network was used to further refine the algorithm and reject mis-identified neutron candidates by having an input of 12 variables relating to the number, position and isotropy of hits, and the pre and post NN background and signal candidates of these distributions are given along with whether the captures occur on hydrogen or gadolinium. The neutron tagging efficiency in MC at pre-selection and post-selection are calculated and as was expected are 40.16\% for all captures, an improvement on the efficiency in MC before the addition of gadolinium. To validate these efficiencies, comparisons were made with Am/Be + 8BGO calibration data and were shown to be consistent.
\newline
The culmination of this thesis was the calculation of the number of DSNB NCQE events, an important background to DSNB detection. By estimating the number of observed neutrino events of each interaction type for a predicted future POT of $10 \times 10^{21}$, and using the neutron tagging efficiency, the NCQE DSNB background was calculated.
