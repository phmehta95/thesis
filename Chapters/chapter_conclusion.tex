\chapter{Conclusion}
\epigraph{``Time may be eternal, Captain, but our patience is not. It's time to put an end to your trek through the stars, make room for other more worthy species.''}{Q, Star Trek: TNG S7E26 All Good Things... (1994)}
\label{chp:conc}

The work presented in this thesis has two distinct parts: one is the calibration work centered around the Monte Carlo simulations of the UK Light Injection System optics and the other part is my work with neutron tagging for the Super Kamiokande Gadolinium Upgrade and the use of the neutron tagging efficiency to estimate the DSNB NCQE background error. 
\newline
The work in this thesis on the UK Light Injection system involving modelling the collimator and diffuser optics in Monte Carlo is important for the aim of measuring the scattering and absorption coefficients of the water in Super-Kamiokande using the UKLI. Future work regarding the UKLI system implemented in Super-Kamiokande will involve looking at forward scattering using the collimator optics and water quality analysis with the diffuser beam. Along with the modelling of the beam spots in simulation, the varying of the time dispersion of the optics in MC is something that was also successfully demonstrated in this thesis and will be crucial for future comparisons of UKLI MC with data.
\newline
The estimation of the DSNB NCQE background is massively important for DSNB analyses as NCQE interactions are the dominant atmospheric neutrino background below 15 MeV. Measuring the NCQE interaction with the T2K beam improves the situation due to the T2K energy region being similar to that of atmospheric neutrinos. The neutrons associated with these neutrino interactions need to be studied precisely, and due to the addition of gadolinium in Super-Kamiokande, neutron tagging has been demonstrably improved in Monte Carlo simulations of the NCQE interactions, particularly in regards to the neutron tagging efficiency. The neutron tagging methodology for SK-Gd has been explored in great detail, with efficiency validations made using Am/Be + 8BGO data. We can consider the prospects for the NCQE background estimation by using this efficiency value to calculate the error on a predicted future measurement of DSNB NCQE background events for JFY 2028. With the increased neutron tagging efficiency due to the addition of gadolinium, which is also something that will be replicated in the much larger Hyper-Kamiokande experiment and the greater ability to tag neutrons associated with neutrino interactions, a first observation of DSNB signal seems well within reach in the next decade.



