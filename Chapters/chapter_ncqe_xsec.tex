\chapter{DSNB NCQE background calculation}
\label{chp:ncqe_xsec}


\section{Introduction}

As mentioned in Chapter 1, NCQE interactions form the main background to the DSNB signal, and therefore calculating the number of events that form this background concludes the work of this analysis. 

The number of observed neutrino events which are detected at Super-Kamiokande as a function of some observable such as reconstructed neutrino energy $N_{\nu}(E_{\nu})$ is given in Equation \ref{eq:nu_number}.

\begin{equation}
    N_\nu(E_\nu)=\mathcal{C} \int \Phi\left(E_\nu\right) \sigma\left(E_\nu\right) \epsilon d E_\nu
\label{eq:nu_number}
\end{equation}

Here, $\mathcal{C}$ is the target volume, $E_{\nu}$ is the true incoming neutrino energy, $\phi(E_{\nu})$ is the flux of the incoming neutrino, $\sigma(E_{\nu})$ is the cross section for the neutrino interaction and $\epsilon$ is the detection efficiency of the neutrino by the far detector. The number of neutrino events from Monte Carlo for each interaction type is given by the summing the number of NCQE events for neutrinos and anti-neutrinos, and charged-current and NC-other interactions, where the numbers of the neutrino and antineutrino NCQE events are multiplied by scale factors which scale the simulated NCQE cross section prediction.


\section{NCQE prediction scale factors}


The number of events $N_{obs}$ are made up of the sum of the signal ($S$) and background ($B$) events, i.e: $N_{obs} = S + B$. The signal events are described by the summation of the number of neutrino and antineutrino NCQE events multiplied by their scale factors: $S  = f_{\nu-\mathrm{NCQE}} M_{\nu-\mathrm{NCQE}}^{\mathrm{FHC}}+f_{\bar{\nu}-\mathrm{NCQE}} M_{\bar{\nu}-\mathrm{NCQE}}^{\mathrm{FHC}}$ for FHC mode. By having a similar equation for RHC mode, one can produce the Equations in \ref{eq:scale_factor_nu}, where $X$ and $Y$ are the number of background events for FHC and RHC mode respectively, which give the scale factor values for neutrino and antineutrino NCQE interactions. 

\begin{equation}
    \begin{aligned}
    f_{\nu-\mathrm{NCQE}}= & \frac{X M_{\bar{\nu} \mathrm{N} \mathrm{NCQE}}^{\mathrm{RHC}}-Y M_{\bar{\nu}-\mathrm{NCQE}}^{\mathrm{FHC}}}{M_{\nu-\mathrm{NCQE}}^{\mathrm{FHC}} M_{\bar{\nu}-\mathrm{NCQE}}^{\mathrm{RHC}}-M_{\nu-\mathrm{NCQE}}^{\mathrm{RHC}} M_{\bar{\nu}-\mathrm{NCQE}}^{\mathrm{FHC}}} \\
    f_{\bar{\nu}-\mathrm{NCQE}}= & \frac{X M_{\nu-\mathrm{NCQE}}^{\mathrm{RHC}}-Y M_{\nu-\mathrm{NCQE}}^{\mathrm{FHC}}}{M_{\bar{\nu}-\mathrm{NCQE}}^{\mathrm{FHC}} M_{\nu-\mathrm{NCQE}}^{\mathrm{RHC}}-M_{\bar{\nu}-\mathrm{NCQE}}^{\mathrm{RHC}} M_{\nu-\mathrm{NCQE}}^{\mathrm{FHC}}}
    \end{aligned}
\label{eq:scale_factor_nu}
\end{equation}


Table \ref{table:nu_FHC_mc} shows the gives the expected number of neutrino events for each type of interaction including neutral current and charged current interactions for FHC mode (same as mentioned in Chapter 6), and Table \ref{table:nu_RHC_mc} shows the same information for RHC mode. These have been calculated with the 21bv2 flux uncertainty with a total SK POT of $21.4 x 10^{20}$ with Run 11 POT included which was taken between March 9th and April 22nd 2021. This was the T2K run with a concentration of $\mathrm{Gd}_{2}\left(\mathrm{SO}_{4}\right)_{3} \cdot 8 \mathrm{H}_{2} \mathrm{O}$ in SK, which matches the amount in the SKDETSIM Monte Carlo. 


\begin{table}
    $$
    \begin{array}{ccc}
    \hline \text { FHC sample } & \text { MC } \# \boldsymbol{\nu}_{\text {det }} & \text { MC } \# \nu_{\text {det }} \text { fraction (\%) } \\
    \hline \nu-N C Q E & 1199.7 & 75.0 \\
    \bar{\nu}-\text { NCQE } & 34.5 & 2.2 \\
    N C-\text { other } & 288.1 & 19.1 \\
    C C & 17.4 & 3.7\\
    \hline \text { Total } & 1599.2 & 100 \\
    \hline
    \end{array}
    $$
    \caption{FHC MC expectation values for each interaction type with a total SK POT of $10 \times 10^{21}$. }
    \label{table:nu_FHC_mc}
\end{table}


\begin{table}
    $$
    \begin{array}{ccc}
    \hline \text { RHC sample } & \text { MC } \# \boldsymbol{\nu}_{\text {det }} & \text { MC } \# \nu_{\text {det }} \text { fraction (\%) } \\
    \hline \nu-N C Q E & 182.8 & 31.9 \\
    \bar{\nu}-\text { NCQE } & 257.0 & 44.8 \\
    N C-\text { other } & 118.4 & 20.6 \\
    C C & 15.3 & 2.7 \\
    \hline \text { Total } & 573.5 & 100 \\
    \hline
    \end{array}
    $$
    \caption{RHC MC expectation values for each interaction type with a total SK POT of $10 \times 10^{21}$.}
    \label{table:nu_RHC_mc}
\end{table}

Although the analysis in this thesis with regards to the work done in Chapter 6 and 7 used FHC mode events, for the purpose of this cross-section extraction, the number of events for each interaction type in RHC mode have been calculated as well. Using the number of events for each interaction type in Table \ref{table:nu_FHC_mc} and Table \ref{table:nu_RHC_mc} and Equation \ref{eq:scale_factor_nu} and Equation \ref{eq:scale_factor_coeff}, along with the number of observed events (204 in FHC and 97 in RHC, taken from \cite{Abe_2019}), the scale factors are calculated and shown in Equation \ref{eq:scale_factors_value}.

\begin{equation}
    \begin{aligned}
    & f_{\nu-\mathrm{NCQE}}=0.710 \pm 0.001 \text { (stat.) } \\
    & f_{\bar{\nu}-\mathrm{NCQE}}=1.067 \pm 0.004 (\text { stat. })
    \end{aligned}
\label{eq:scale_factors_value}
\end{equation}
    


\section{Flux averaged NCQE cross-section}

NEUT's cross section prediciton for the interaction of neutrinos and antineutrinos with ${ }^{16} \mathrm{O}$ is shown in Equation \ref{eq:NEUT_xsec_prediction}. Here $\langle\sigma_{\nu \text {-NCQE }}^{\text {NEUT }}\rangle$ gives the flux-averaged cross section prediction for neutrinos and $\langle\sigma_{\bar{\nu}-\mathrm{NEUCQ}}^{\mathrm{NEUT}}\rangle$ gives the flux-averaged cross section for antineutrinos. 


\begin{equation}
    \begin{aligned}
    &\left\langle\sigma_{\nu \text {-NCQE }}^{\mathrm{NEUT}}\right\rangle= \frac{\sum_{\nu=\nu_\mu, \nu_e} \int \sigma_{\nu-\mathrm{NCQE}}^{\mathrm{NEUT}}\left(E_\nu\right) \phi_\nu\left(E_\nu\right) d E_\nu}{\sum_{\nu=\nu_\mu, \nu_e} \int \phi_\nu\left(E_\nu\right) d E_\nu}=2.13 \times 10^{-38} \mathrm{~cm}^2 / \text { oxygen } \\
    &\left\langle\sigma_{\bar{\nu} \text {-NCQE }}^{\mathrm{NEUT}}\right\rangle=\frac{\sum_{\nu=\bar{\nu}_\mu, \bar{\nu}_e} \int \sigma_{\bar{\nu} \text {-NCQE }}^{\mathrm{NEUT}}\left(E_\nu\right) \phi_\nu\left(E_\nu\right) d E_\nu}{\sum_{\nu=\bar{\nu}_\mu, \bar{\nu}_e} \int \phi_\nu\left(E_\nu\right) d E_\nu}=0.88 \times 10^{-38} \mathrm{~cm}^2 / \text { oxygen. }
    \end{aligned}
\label{eq:NEUT_xsec_prediction}
\end{equation}

The flux averaged NEUT cross section values in Equation \ref{eq:NEUT_xsec_prediction} is taken from \cite{Abe_2019}, and the integrations are taken up to an energy of 10 GeV. These NEUT cross sections are multiplied by the scale factors in order to obtain the flux-averaged NCQE cross sections, shown in Equation \ref{eq:xsec_value}. The systematic error on the final cross-section values are taken from Table \ref{table:systuncertaintytable} by summing up all the percentages and using the final percentage to calculate the systematic error on the values.


\begin{equation}
    \begin{aligned}
    & \left\langle\sigma_{\nu \text {-NCQE }}\right\rangle=f_{\nu \text {-NCQE }} \cdot\left\langle\sigma_{\nu \text {-NCQE }}^{\mathrm{NEUT}}\right\rangle=1.512 \pm 0.004 \text {(stat.)} \pm 0.191 \text {(syst.)} \times 10^{-38} \mathrm{~cm}^2 / \text { oxygen }, \\
    & \left\langle\sigma_{\bar{\nu} \text {-NCQE }}\right\rangle=f_{\bar{\nu} \text {-NCQE }} \cdot\left\langle\sigma_{\bar{\nu} \text {-NCQE }}^{\mathrm{NEUT}}\right\rangle=0.939 \pm 0.001 \text {(stat.)} \pm 0.119 \text {(syst.)} \times 10^{-38} \mathrm{~cm}^2 / \text { oxygen }
    \end{aligned}
\label{eq:xsec_value}
\end{equation}



