\chapter{DSNB NCQE background calculation}
\label{chp:ncqe_xsec}


\section{Introduction}

As mentioned in Chapter 1, NCQE interactions form the main background to the DSNB signal, and therefore calculating the number of events that form this background concludes the work of this analysis. 

The number of observed neutrino events which are detected at Super-Kamiokande as a function of some observable such as reconstructed neutrino energy $N_{\nu}(E_{\nu})$ is given in Equation \ref{eq:nu_number}.

\begin{equation}
    N_\nu(E_\nu)=\mathcal{C} \int \Phi\left(E_\nu\right) \sigma\left(E_\nu\right) \epsilon d E_\nu
\label{eq:nu_number}
\end{equation}

Here, $\mathcal{C}$ is the target volume, $E_{\nu}$ is the true incoming neutrino energy, $\Phi(E_{\nu})$ is the flux of the incoming neutrino, $\sigma(E_{\nu})$ is the cross section for the neutrino interaction and $\epsilon$ is the detection efficiency of the neutrino by the far detector. The number of neutrino events from Monte Carlo for each interaction type is given by the summing the number of NCQE events for neutrinos and anti-neutrinos, and charged-current and NC-other interactions, where the numbers of the neutrino and antineutrino NCQE events are multiplied by scale factors which scale the simulated NCQE cross section prediction.


\section{NCQE scale factor prediction}


The number of events $N_{obs}$ are made up of the sum of the signal ($S$) and background ($B$) events, i.e: $N_{obs} = S + B$. The signal events are described by the summation of the number of neutrino and antineutrino NCQE events multiplied by their scale factors: $S  = f_{\nu-\mathrm{NCQE}} M_{\nu-\mathrm{NCQE}}^{\mathrm{FHC}}+f_{\bar{\nu}-\mathrm{NCQE}} M_{\bar{\nu}-\mathrm{NCQE}}^{\mathrm{FHC}}$ for FHC mode. By having a similar equation for RHC mode, one can produce the Equations in \ref{eq:scale_factor_nu}, where $X$ and $Y$ are the number of background events for FHC and RHC mode respectively, which give the scale factor values for neutrino and antineutrino NCQE interactions. 

\begin{equation}
    \begin{aligned}
    f_{\nu-\mathrm{NCQE}}= & \frac{X M_{\bar{\nu} \mathrm{N} \mathrm{NCQE}}^{\mathrm{RHC}}-Y M_{\bar{\nu}-\mathrm{NCQE}}^{\mathrm{FHC}}}{M_{\nu-\mathrm{NCQE}}^{\mathrm{FHC}} M_{\bar{\nu}-\mathrm{NCQE}}^{\mathrm{RHC}}-M_{\nu-\mathrm{NCQE}}^{\mathrm{RHC}} M_{\bar{\nu}-\mathrm{NCQE}}^{\mathrm{FHC}}} \\
    f_{\bar{\nu}-\mathrm{NCQE}}= & \frac{X M_{\nu-\mathrm{NCQE}}^{\mathrm{RHC}}-Y M_{\nu-\mathrm{NCQE}}^{\mathrm{FHC}}}{M_{\bar{\nu}-\mathrm{NCQE}}^{\mathrm{FHC}} M_{\nu-\mathrm{NCQE}}^{\mathrm{RHC}}-M_{\bar{\nu}-\mathrm{NCQE}}^{\mathrm{RHC}} M_{\nu-\mathrm{NCQE}}^{\mathrm{FHC}}}
    \end{aligned}
\label{eq:scale_factor_nu}
\end{equation}


Table \ref{table:nu_FHC_mc} shows the gives the expected number of neutrino events for each type of interaction including neutral current and charged current interactions for FHC mode for a predicted future POT of $10 \times 10^{21}$ starting Japanese Fiscal Year 2026 \cite{Abe:2016tii}, and Table \ref{table:nu_RHC_mc} shows the same information for RHC mode. These have been calculated with the 21bv2 flux uncertainty tuning. 


\begin{table}
    $$
    \begin{array}{ccc}
    \hline \text { FHC sample } & \text { MC } \# \boldsymbol{\nu}_{\text {det }} & \text { MC } \# \nu_{\text {det }} \text { fraction (\%) } \\
    \hline \nu-N C Q E & 1199.7 & 75.0 \\
    \bar{\nu}-\text { NCQE } & 34.5 & 2.2 \\
    N C-\text { other } & 288.1 & 19.1 \\
    C C & 17.4 & 3.7\\
    \hline \text { Total } & 1599.2 & 100 \\
    \hline
    \end{array}
    $$
    \caption{FHC MC expectation values for each interaction type with a total SK POT of $10 \times 10^{21}$. }
    \label{table:nu_FHC_mc}
\end{table}


\begin{table}
    $$
    \begin{array}{ccc}
    \hline \text { RHC sample } & \text { MC } \# \boldsymbol{\nu}_{\text {det }} & \text { MC } \# \nu_{\text {det }} \text { fraction (\%) } \\
    \hline \nu-N C Q E & 182.8 & 31.9 \\
    \bar{\nu}-\text { NCQE } & 257.0 & 44.8 \\
    N C-\text { other } & 118.4 & 20.6 \\
    C C & 15.3 & 2.7 \\
    \hline \text { Total } & 573.5 & 100 \\
    \hline
    \end{array}
    $$
    \caption{RHC MC expectation values for each interaction type with a total SK POT of $10 \times 10^{21}$.}
    \label{table:nu_RHC_mc}
\end{table}

Due to this being a future prediction of the background rate using MC and not data, $N_{obs}$ is the total number of events in the MC, meaning that the value of the scale factor is 1.00. However, the uncertainty on the scale factor is not trivial, and Equation \ref{eq:scale_factor_uncertainty} shows how it is calculated.

\begin{equation}
    \delta_{f} = \left({\frac{\sqrt{N_{\text{obs}}}}{N_{\text{obs}}}}\right)^{2} + \left({\frac{\delta_{\epsilon}}{\epsilon}}\right)^{2}
\label{eq:scale_factor_uncertainty}
\end{equation}

The values for the scale factors for FHC and RHC mode and their uncertainties are shown in Equation \ref{eq:scale_factors}.


\begin{equation}
    \begin{aligned}
    & f_{\nu-\mathrm{NCQE}}= 1.000 \pm 0.198 \\
    & f_{\bar{\nu}-\mathrm{NCQE}}=1.000 \pm 0.199
    \label{eq:scale_factors}
    \end{aligned}
\end{equation}


\section{Number of DSNB NCQE background events}

The number of DSNB NCQE background events is given by Equation \ref{eq:background_events}. 

\begin{equation}
N_{B} = \mathcal{C} \epsilon f\int \Phi(E_{\nu}) \sigma(E_{\nu}) d E_{\nu} 
\label{eq:background_events}
\end{equation}

Where $\mathcal{C}$ is the target volume, $\Phi$ is the neutrino flux, $\sigma$ is the neutrino interaction cross section and $\epsilon$ is the neutron tagging efficiency. 



