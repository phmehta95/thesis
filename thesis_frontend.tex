% /////////////////////////////////////////////////////////////////////////////
%
%   Ph.D. Thesis Manuscript -- Pruthvi Mehta
%
% /////////////////////////////////////////////////////////////////////////////


% /////////////////////////////////////////////////////////////////////////////
% PREAMBLE
  %\documentclass
\documentclass[11pt,twoside]{report}
\renewcommand{\appendixname}{}%
  % for title page
  %\title{Neutral-Current Quasi-Elastic events with neutron tagging at Super-Kamiokande}
  \author{Pruthvi Mehta}
  \date{2022}




  %%%%%%%%%%%%%%%%%%%%%%%%%
  %       PACKAGES        %
  %%%%%%%%%%%%%%%%%%%%%%%%%
  % Graphics and Figures
  \usepackage{graphicx}                % Graphics importing library
  \usepackage{float}                   % Placement of floats
  \usepackage[font=small]{caption}     % Set formatting options for captions
  \usepackage{subcaption}              % Like captions, but for sub-figures
  \usepackage{wrapfig}                 % Wrap text around figures
  \usepackage[headheight=14pt,inner=4cm,outer=2.5cm]{geometry} % The minimal margins as specified by Code of Practice.
  \usepackage{setspace}
  \usepackage{datetime}
  \usepackage{fancyhdr}
  \usepackage{tikz}
  \usepackage{floatpag}
  \usepackage{lscape}
  \usepackage[compact]{titlesec}
  \raggedbottom
  \usepackage[titletoc]{appendix}
  \usepackage{placeins}
  \usepackage{epigraph}
  % Tables
  %\usepackage{arydshln}                % dashed version of cline
  \usepackage{array}                   % Do more fancy things with tables
  \newcolumntype{L}{>{$}l<{$}} % math-mode version of "l" column type
  
  \usepackage{multirow}                % Let cells span multiple rows
  %\usepackage{longtable}               % For tables that span multiple pages (remove if not needed)
  \usepackage{booktabs}                % Make tables look better
  \usepackage{makecell}
  \newcommand{\mc}{\multicolumn{1}{c}}
  \usepackage{hhline}

  % Maths
  \usepackage{amsmath}                 % Makes maths look pretty
  \usepackage{amssymb}                 % Something to do with symbols...
  \usepackage{xfrac}                   % Pretty inline fractions
  \usepackage{bbold}
  \usepackage{chemformula}
  \usepackage{gensymb}

  \usepackage{setspace}                % Double spacing for easy reading 
  %\doublespacing{}
  \onehalfspacing

  % Hyperlinks & Colors
  %\usepackage[usenames, dvipsnames, table]{xcolor}

  % Fonts
  \usepackage{anyfontsize}             % Setting of font size by point
  \usepackage{cite}                    % Make citations looks pretty

  % Misc
  \usepackage{lipsum}                  % Lorem ipsum generator, good for imagining writing
  \usepackage[nohyperlinks]{acronym}   % Acronym support
  \usepackage[displaymath, mathlines]{lineno}
  %\usepackage[backend=biber,
  %            style=numeric,
  %            backref=true,
  %            bibencoding=utf8,
  %            url=false,
  %            doi=true,
  %            isbn=false
  
  


  
  
  
  %\AtEveryBibitem{\clearfield{note}\clearfield{addendum}\clearfield{annotation}} 
  %\AtEveryCitekey{\clearfield{note}\clearfield{addendum}\clearfield{annotation}}


  %%%%%%%%%%%%%%%%%%%%%%
  %       FONTS        %
  %%%%%%%%%%%%%%%%%%%%%%
 % \usepackage{fontspec}                       % Use of system fonts
 % \setmainfont{Crimson Text}[
 %   Ligatures=TeX,
 %   SmallCapsFont=Crimson,
 %   SmallCapsFeatures={Letters=SmallCaps}
  %]
  %\setsansfont[Ligatures=TeX]{Helvetica}      % Set sans-serif font

  % Change to 2 to comply with UGS requirements
  %\renewcommand{\baselinestretch}{1.15}

  %\usepackage[hidelinks=true]{hyperref}

  \usepackage{hyperref}
  \hypersetup{colorlinks,
              citecolor=blue,
              filecolor=blue,
              linkcolor=blue,
              urlcolor=blue
  }

  \makeatletter
  %% Insert your title and name.
  \Huge
  \title{\textbf{Neutral-Current Quasi-Elastic events with neutron tagging at Super-Kamiokande}} \let\Title\@title
  \author{Pruthvi Hiren Mehta} \let\Author\@author
  \makeatother


  %/////////////////////////////////////////////////////////////////////////////
  % DOCUMENT

\begin{document}
\pagestyle{empty}

\begin{titlepage}
	\centering
	\vspace*{1cm}
	\includegraphics[width=65mm]{Figures/UoL_Logo.jpg}\par\vspace{1cm}
		\vspace{2cm}
	{\huge \Title\par}
	\vspace{5cm}
	{Thesis submitted in accordance with the requirements of the University of Liverpool for the  degree  of  Doctor  in  Philosophy by  \par}
	\vspace{1cm}
	{\textbf{\Author}\par}
	\vfill

% Bottom of the page
	{\large \monthname\, \the\year\par}
\end{titlepage}
\cleardoublepage



%% In the preface, pages are numbered by roman numerals (i,ii,...) and in the header Author and Title.
%\pagenumbering{roman}
\pagestyle{fancy}

\cleardoublepage
\thispagestyle{empty}

    % Typeset title page.
    %\maketitle


    % Include front matter
\chapter*{Dedication}
\thispagestyle{empty}

%\begin{center}
%    \includegraphics[width=0.1\textwidth]{Figures/dedication_nidhi.png}
%\end{center}


\vspace*{\fill}
\noindent
\hspace*{-\oddsidemargin}%
\begin{center}
\includegraphics[width=0.2\textwidth]{Figures/dedication_nidhi.png}
\end{center}
\vspace*{\fill}
\chapter*{Acknowledgements}
\thispagestyle{empty}
First and foremost thanks must be given to my academic supervisor Professor Neil McCauley, someone whose intelligence, work ethic and kindness have continously been a source of inspiration to me. 

I am also forever indebted to a number of postdoctoral researchers who I was fortunate enough to have aid me with my research; namely Dr Ka Ming Tsui, Dr Pablo Fernandez Menendez, Dr Lauren Anthony, Dr Adrian Pritchard and Dr Samuel Jamuel Jenkins. Thank you all for your continued encouragement and support, and apologies for emailing all of you for help so much (but in my defence ROOT is the worst).

I am also incredibly grateful to members of the neutron tagging working group, in particular Dr Fabio Iacob whose weekly meetings about my analysis were crucial to my work and who is one of the most patient and kind individuals I have ever met (and I wouldn't have survived calculating the systematics without him). I am also indebted to Seungho Han for his help with the new neutron tagging software, and also Nakajima-sensei, Koshio-sensei, Akutsu-sensei and Tairafune-san for their input in the neutron tagging meetings. 

I'd also like to thank current and former members of the Super-Kamiokande and Hyper-Kamiokande calibration groups, Dr Jordan McElwee, Dr Matthew Thiesse, Dr Billy Vinning and Yang-san for UKLI MC implementation help. 
 
A huge thank you to Dr Gedminas Elertas, who has been a source of virtue and intellectual inspiration to me and has made me want to keep learning, not just with things relating to physics but outside of it too. The day-to-day of academic research sometimes ground me down a little internally, but you kept the spirit of scientific curiosity alive inside me, and I can't thank you enough for this. 

Many thanks to my wonderful besties from undergrad, Karel Green and Rachel Couchman, I've known both of you for nearly a decade and I treasure your friendship so so much, you've kept me sane and supported throughout this entire journey. Here's to you, babes!
Also to my schoolfriends Catherine Thomas, Amy Rose-Mansbridge and Catriona Bradley, I've known you for over 15 years and it has been incredible to see you all blossom into amazing, capable and talented young women who I'm blessed to know. Also thanks to Patrick Bates for the calmness and profound wisdom that you so often espouse, you wonderful lovely nerd! 

Thank you to Joseph (my \textit{imzadi}) for your gentle and kind heart, you've always been so incredibly supportive and I'm so happy to know you.  

To my parents, ``thank you'' isn't anywhere near a sufficient enough response to how much you've supported me. Mum - your strength and wisdom is unparalled, and dad, well, you're the best physicist I know. Finally to my sister Nidhi, who this thesis is dedicated to, I love you so much and your kindness and intelligence is something I aspire to everyday. You're amazing! 

\chapter*{Abstract}
\thispagestyle{empty}

The error on the DSNB NCQE background is calculated for an estimated future POT of $10 \times 10^{21}$ starting JFY 2028, and is estimated at $N_{\nu-\mathrm{DSNB}_{NCQE}}= 9.88 \pm 3.62$ events for FHC mode. The neutral current quasi-elastic (NCQE) interactions of neutrinos on ${ }^{16} \mathrm{O}$ generated by the beam neutrinos of the T2K (Tokai-to-Kamioka) experiment produces neutrons in the medium of the far detector Super Kamiokande. This thesis focuses on the detection of these neutrons by a neutron tagging algorithm specifically developed for the addition of 0.026\% $\mathrm{Gd}_{2}\left(\mathrm{SO}_{4}\right)_{3} \cdot 8 \mathrm{H}_{2} \mathrm{O}$ to Super-Kamiokande (the level present in SK from August 2020 - May 2022), modelled in the MC simulation by the ANNRI-Gd model. The neutron tagging algorithm is complemented by the use of an artificial neural network (ANN), which is used on neutron candidates to improve sensitivity to signal events and to reduce backround events. The efficiency of this neutron tagging algorithm is calculated and validated using calibration data from neutrons produced by an Am/Be + 8BGO source.

The calibration work in this thesis centers on developing Monte Carlo for the collimator and diffuser UK Light Injector optics, which were compared against test run data. The time-of-flight (TOF) hit time distributions of these MC were compared against data and the time disperson of the MC was adjusted to approximate that of data.
    % \include{acronyms}

\tableofcontents

\cleardoublepage
    % Include body text
  \chapter{Introduction}
\label{chp:intro}

\section{Introduction}




 %   \part{NEUTRINOS \& SUPERNOVAE}
  \chapter{The Tokai-to-Kamioka experiment}
\label{chp:t2kdetector}

\section{The Tokai-to-Kamioka experiment}
The Tokai-to-Kamioka (T2K) experiment is a long baseline neutrino oscillation experiment based in Japan and its purpose is to study neutrino oscillations: specifically a precision measurement of the neutrino oscillation parameters $\Delta m_{23}^{2}$ and $\sin ^{2} \theta_{23}$ and to improve the measurement of the leptonic CP violating phase $\delta_{CP}$, which is mentioned in Chapter 1. The experiment produces a beam of intense muon neutrinos at J-PARC (Japan Proton Accelerator Research Complex) in Tokai, which is located on the east coast of Japan in Ibaraki Prefecture. The muon neutrino beam travels 295 km west towards the far detector Super-Kamiokande (see Chapter 3.)  The neutrino beam is measured by other detectors such as ND280 and INGRID, before they oscillate and reach Super-Kamiokande which is important with regards to measuring the neutrino oscillation parameters. Both ND280 and Super-Kamiokande are off-axis detectors, meaning that they are located 2.5$\degree$ off axis from the centre of the neutrino beam. This reuslts in the peak of the energy of the muon neutrinos to be 0.6 GeV, maximising the neutrino oscillation on the 295 km baseline is maximised. The spread in muon neutrino energy is reduced, this means that these detectors are far less susceptible to potential backgrounds. This chapter explains production of the muon neutrino beam from the JPARC proton beam line and the near detector complex.

\subsection{Neutrino beam production}

\subsubsection{JPARC proton beam production}


A sequence of three accelerators results in the production of a proton beam from the JPARC accelerator complex, these are the LINAC (linear accelerator), an RCS (rapid cyclic synchrotron) and the main ring synchrotron (MR). A negative hydrogen ion is accelerated to a kinetic energy of 400 MeV, from which a beam of protons is created by converting the negative hydrogen ion beam using charge-stripping foils. This proton beam is then accelerated to a kinetic energy of 3 GeV by the RCS, and about 5\% of the bunches produced from this process are passed to the MR where the proton beam will be accelerated up to 30 GeV. 

\subsubsection{Neutrino beam production}

The neutrino beam is produced using the primary and secondary beamline as shown in Figure \ref{fig:nubeamline}. The primary beamline takes the proton beam from the MR and directs it towards direction Kamioka. The beam is then transferred through a succession of beam monitors which measure facets of the proton beam including the beam profile, intensity and position. The beam monitor which is closest to the graphite target measures the "Protons-On-Target" (POT), a value used to determine the neutrino beam flux. The secondary beamline involves taking the proton bunches and passing them through a target station, the decay volume and the beam dump. After interacting with the target station, the proton bunches are collimated through a 1.7 m graphite rod where the collimated hole the proton bunches pass through are 30 mm in diameter. Beam profile reconstruction occurs in the OTR (Optical Transition Radiation) monitor, made up of titanium alloy foil placed at a 45 degree angle in order to intercept the beam. As the proton beam enters the foil, the visible light that is produced escapes through a collection of mirrors and is then captured by a charge injection device camera, which creates the beam profile. 
\newline
After beam profiling, the proton beam then impacts upon a graphite rod target which is 91.4 cm long and 2.6 cm in diameter - this collision produces secondary hadrons, including pions which are focused by three magnetic horns (shown in Figure \ref{fig:magnetichorns}). These magnetic horns can be used to produce either a muon neutrino or muon antineutrino beam depending on the polarity of the 250kA current they are pulsed with. If a +250 kA current is used, the positive pions and kaons produced can go on to make muon neutrino beams wheareas if a -250 kA current is used negative pions and kaons can decay to create muon antineutrino beams (both are shown in Equation \ref{eq:nubeam}). The +250 kA mode is called Forward Horn Current (FHC) mode and the -250 kA mode is called Reversed Horn Current (RHC) mode. The analysis in this thesis will occur in FHC mode only. 

\begin{figure}
    \includegraphics[width=\textwidth]{Figures/t2k_magnetic_horns.png}
    \caption{Front and side (left and right) view drawings of the three T2K magnetic horns (top, middle bottom).}
        \label{fig:magnetichorns}
\end{figure}

\begin{figure}
    \includegraphics[width=\textwidth]{Figures/nubeamline.png}
    \caption{Schematic of the neutrino beam line}
        \label{fig:nubeamline}
\end{figure}

\begin{equation}
\begin{array}{lll}
\pi^{+} & \longrightarrow \mu^{+}+\nu_{\mu} & \text { FHC } \\
\pi^{-} & \longrightarrow \mu^{-}+\overline{\nu_{\mu}} & \text { RHC }
\end{array}
\label{eq:nubeam}
\end{equation}

A 75 ton volume beam dump made of graphite and iron stops the particles, specifically the protons, secondary hadrons and mesons which have a momentum below 5 GeV/c end up being absorbed by the beam dump. A muon monitor is placed after the beam dump in order to directly measure the the beam intensity and beam direction. Muons can be used to monitor the beam properties because along with the neutrinos these are the main particles produced from the pion decay. After the muon monitor, a nucleon emulsion plate detector measures the flux and momentum of the muons. 

\subsection{Near detectors}

\subsubsection{INGRID detector}

The INGRID (Ineractive Neutrino GRID) detector is a neutrino detector which unlike ND280 is placed on-axis instead of off-axis. This allows it to directly monitor the direction of the neutrino beam and the intensity of the neutrino beam by measuring the interactions of the neutrinos with the alternating iron that make it up. INGRID is also placed 280 m from the graphite target and consists of 14 modules placed in a cross formation, with the centre of the cross placed at the centre of the neutrino beam. The INGRID modules are comprised of nine iron plates alternating with 11 tracking scintillator planes, which are themselves surrounded by scintillator plates the purpose of which is to reject interactions that occur outside the module. A schematic of the INGRID cross is shown in \ref{fig:ingridcross} and a schematic of the modules is shown in \ref{fig:ingridmodule}. 

\begin{figure}
    \includegraphics[width=\textwidth]{Figures/ingridcross.png}
    \caption{Schematic of the INGRID cross taken from \cite{t2kcollaborationT2KExperiment2011}.}
    \label{fig:ingridcross}
\end{figure}
\begin{figure}
    \includegraphics[width=\textwidth]{Figures/ingridmodule.png}
    \caption{Individual INGRID module schematic taken from \cite{t2kcollaborationT2KExperiment2011}.}
    \label{fig:ingridmodule}
\end{figure}

An additional module, called the proton module was added to measure the muons in combination with protons produced by the neutrino beam in INGRID. This module is used to distinguish the quasi-elastic interaction channel in order to compare it with Monte Carlo simulations of the beamline and neutrino interactions. The Proton Module is made of scintillator planes (no alternating iron plates) and is contained by veto planes. The Proton Module was placed in the centre of the INGRID cross at the intersection of the vertical and horizontal modules. Figure \ref{fig:ingridevent} shows what a standard neutrino event looks like inside the INGRID module. A neutrino enters from the left and after interacting with the scintillator cells (shown in green) produces tracks (shown in red), with the relative size of each circle corresponding to the observed signal in that cell. The blue cells show the position of the veto scintillators, while the gray planes show the iron plates.   

The hit efficiency in INGRID is calculated using muons: the muon track is reconstructed, but not using the hit information, and then channels in the scintillator plane  expected to contain hits from the trajectory of the track are checked.
Figure \ref{fig:track_angle} shows the hit efficiency as a function of reconstructed track angle measured by cosmic-ray data 

\begin{figure}
    \includegraphics[width=\textwidth]{Figures/ingridevent.png}
    \caption{INGRID event display showing a typical INGRID event, taken from \cite{t2kcollaborationT2KExperiment2011}.}
    \label{fig:ingridevent}
\end{figure}

\begin{figure}
    \includegraphics[width=\textwidth]{Figures/track_angle_ingrid.png}
    \caption{Hit efficiency plotted against reconstructed track angle}
    \label{fig:track_angle}
\end{figure}


\subsubsection{ND280}

ND280 is a near detector which sits 280 m from the target. It is an off-axis detector, meaning that just like Super-Kamiokande, it is placed 2.5 $\degree$ off-axis from the center of the beam. This stems from the relationship between the energy of the neutrinos produced from the decay of the pions, a relationship shown in Equation \ref{eq:piondecay}. 

\begin{equation}
    E_{\nu}=\frac{m_{\pi}^{2}-m_{\mu}^{2}}{2\left(E_{\pi}-\sqrt{E_{\pi}^{2}-m_{\pi}^{2}} \cos \theta\right)}
\label{eq:piondecay}
\end{equation}

where $E_{\pi}$ is the energy of the parent pion and $\theta$ is the scattering angle between the direction of the outgoing neutrino and the direction of the parent pion's momentum. $m_{pi}$ is the parent pion mass and $m_{\mu}$ is the mass of the outgoing muon. Figure \ref{fig:energyangle} shows the neutrino energy from pion decay plotted against the energy of the parent pion for a range of off-axis angles. 

\begin{figure}
\includegraphics[width=\textwidth]{Figures/energy_angle.PNG}
\caption{Energy of the neutrino plotted against the energy of the parent pion for multiple different off-axis angles}
\label{fig:energyangle}
\end{figure}

As off-axis angle increases, the intensity of the neutrino beam decreases, and therefore picking an off-axis angle of 2.5 $\degree$ at which to place the ND280 detector complex strikes a good balance between keeping a high beam intensity while ensuring a peak energy of 0.6 GeV in order to have the neutrino oscillation be maximised at 295km. The relationship between muon neutrino oscillation probability and muon neutrino energy is shown at the top in Figure \ref{fig:nuprobosc}, and at the bottom the muon neutrino flux 295 km away from the graphite target can be seen for three different off-axis angles.

\begin{figure}
    \includegraphics[width=\textwidth]{Figures/nuprobosc.png}
    \caption{The probability of survival of muon neutrinos (top plot) and neutrino beam flux at the 295km far detector (bottom) Taken from \cite{t2kcollaborationT2KNeutrinoFlux2013}.}
    \label{fig:nuprobosc}
\end{figure}

Figure \ref{fig:ND280_schematic} shows a schematic of the ND280 detector complex. It has five main goals: firstly, to measure the cross sections of muon neutrino interactions, so neutrino-nucleus interaction models can reduce their systematic uncertainties. Secondly, measuring the component of the neutrino beam which is made of electron neutrinos, hence being able to better constrain the background to electron neutrino appearance at the far detector. Thirdly, to predict the event rate at Super-Kamiokande by measuring the energy spectrum of the muon neutrinos produced. It also aims to study non quasi-elastic processes which produce pions below the Super Kamiokande Cherenkov threshold, and to measure the neutral current (NC) pion-nought production rates. 

\begin{figure}
    \includegraphics[width=\textwidth]{Figures/nd280_complex.png}
    \caption{Near detector ND280 schematic taken from \cite{t2kcollaborationT2KExperiment2011}.}
    \label{fig:ND280_schematic}
\end{figure}

N2D280 is made of a neutral pion detector ($P0D$), three Time Projection Chambers (TPCs) and two Fine Grained Detectors (FGDs). These are enclosed within Electromagentic Calorimeters (ECals) and a Side Muon Range Detector (SMRD). These detectors are magnetised using a magnet orgininally used in the UA1 detector at CERN. 

The neutral pion detector is important due to its ability to detect a process that can imitate the signal event of electron neutrino appearance at Super-Kamiokande. Neutral pions are produced during the neutral current interactions on water ($\nu_{\mu} + N \rightarrow \nu_{\mu} + N^{'} +\pi^{0} + X$) and the purpose of $P0D$ is to measure the cross-section of this interaction. The central part of the $P0D$ detector is made of planes of scintillator, brass and water bags which are placed in alternating layers as shown in Figure \ref{fig:p0d}. There are also two electromagnetic calorimeters which remove events entering the detector from the outside. The SMRD (Side Muon Range Detector) is used as a way to measure the momenta of muons which escape the detector complex at a large angle relative to the direction of the beam. There are three time projection chambers placed downstream of the neutral pion detector. These are used for reconstruction of charged particle tracks, particle identification and detrmination of the momentum of the particle. Each individual TPC is comprised of a box filled with a argon based drift gas which is surrounded by an insulating outer box which contains $CO_{2}$. The inward facing walls of the inner box have copper strips with a seperation of precisely 11.5 mm, and along with a central cathode produces a uniform electric field. Therefore, charged particles ionise the drift gas as they pass through the TPC which produces electrons which are pushed away from the central cathode and towards one of the readout planes. These electrons are then amplified by an electric field of 27 kV/cm and the signals are read out by MicroMegas detectors. Figure \ref{fig:TPC_schematic} shows a schematic drawing of a TPC.

\begin{figure}
    \includegraphics[width=\textwidth]{Figures/tpc_schematic.png}
    \caption{Schematic of a TPC taken from \cite{t2kcollaborationT2KExperiment2011}.}
\label{fig:TPC_schematic}
\end{figure}

\begin{figure}
    \includegraphics[width=\textwidth]{Figures/p0d.png}
    \caption{P0D schematic taken from \cite{t2kcollaborationT2KExperiment2011}.}
\label{fig:p0d}
\end{figure}

There are two fine grained detectors (FGDs) in ND280, which have two purposes. Firstly to yield a mass for the neutrino interaction, and secondly, to track the charged particles issuing from the neutrino interaction vertex. Each FGD has dimmensions of 2300 mm x 2400 mm x 365 mm and consist of 1.1 tons of target material. The FGD  contains layers of scintillator bars, with each layer orientated in the x and y directions perpendicular to the neutrino beam which allows for these charged particles to be tracked. The first FGD is comprised of 15 scintillator modules, while the second FDG has 7 scintillator X-Y modules alternating with 6 2.5cm thick water target modules. Comparing the rate at which neutrinos interact with the first FGD and the second FDG, neutrino interaction cross sections on carbon and on water can be measured. Figure \ref{fig:FGD_schematic} shows a view of one of these FGDs.

\begin{figure}
    \includegraphics[width=\textwidth]{Figures/FGD_schematic.png}
    \caption{FGD with the front cover removed: the X-Y scintillator modules are shown in green.}
\label{fig:FGD_schematic}
\end{figure}

The ECal is an electromagnetic calorimeter which surrounds the P0D, the TPCs and the FGDs. It consists of plastic scintillator layers sandwiched between lead absorber sheets. Its purpose is to aid in the full event reconstruction through detecting photons and measuring their energy and direction, along with detecting charged particles and getting information which helps in their identification (seperating out electrons, muons and pions.) As well as this, the ECal helps reconstruct pi-zero events produced by neutrino interactions inside the inner detectors. Figure \ref{fig:ECal_schematic} shows the external view of one ECal module.

\begin{figure}
    \includegraphics[width=\textwidth]{Figures/ECal_schematic.png}
    \caption{Electromagnetic calorimeter module external view}
\label{fig:ECal_schematic}
\end{figure}


  \chapter{The Super Kamiokande Detector}
\label{chp:superk}


\section{Event Reconstruction}

\subsection{Vertex Reconstruction}
For low energy events (events up to 100MeV), Super-Kamiokande currently uses BONSAI (Branch Optimisation Navigating Successive Annealing Interactions) for event reconstruction. Vertex reconstruction for Super-Kamiokande has undergone changes and improvements depending on the phase of the experiment. 
\newline{}
For Phase I of Super-Kamiokande, vertex reconstruction depended on a lattice of test vertices with 4m spacing throughout the detector, with a specific measure of goodness for each test vertex: the test vertex with the highest measure of goodness would have around it a more finely spaced grid, and the process would be repeated. For Phase II of Super-Kamiokande due to the reduced number of PMTs, this approach was no longer as successful as it was in Phase I and as a result the reconstruction perfomance declined, and BONSAI was created as a replacement. Instead of using a fixed grid which was the case with SK-I and SK-II, BONSAI creates test vertices by selecting groups of four PMT hits and seeing where the timing residuals of the PMT hits would be most reduced. After these test vertices have been indentified, a maximum likelihood fit over all the PMT hits in the event is performed, shown in Equation \ref{bonsailikelihood}.

\begin{equation}
    \mathcal{L}(\vec{x}, t_{0})=\sum_{i=1}^{N_{\text {hlt }}} \log (P(t-t_{\text {tof }}-t_{0}))
\label{bonsailikelihood}
\end{equation}

where ($\vec{x}, t_{0}$) is the test vertex, and $(P(t-t_{\text {tof }}-t_{0}))$ is the probablility density function of the timing residual, which for each PMT hit is defined as $(t-t_{\text {tof }}-t_{0})$, where $t_{0}$ is the time of the interaction, $t_{tof}$ is the time of flight from the interaction vertex position to the position of the hit PMT, $t$ is the PMT hit time. The vertex resolution 

\begin{figure}
    \includegraphics[scale=0.4]{Figures/bonsai_pdf_res.png}
\caption{Probability density of the timing residual P$(t-t_{\text {tof }}-t_{0})$, where $t_{0}$ use for the vertex reconstruction maximum likelihood fit. The peaks at 30ns and 100ns are caused by PMT after-pulsing. Figure from \cite[nakanopdf].}
    \label{bonsaivertexpdf}
\end{figure}

\begin{figure}
    \includegraphics[scale=0.4]{Figures/bonsai_vertex_res.png}
\caption{The vertex resolution (the point at which 68\% of the events in the distance distribution between the actual and reconstructed vertex are contained) for the different SK phases. SK-I (Blue), SK-III (Red), SK-IV (Black).  Figure from \cite[nakanopdf].}
    \label{bonsaivertexres}
\end{figure}

\subsection{Direction Reconstruction}

Cherenkov light is emitted in a conical formation as electrons and positrons travel through water, with a Cherenkov angle of $\approx 42\degree$. BONSAI can reconstruct the direction of these particles by using this information along with the reconstructed vertex. This reconstruction occurs using a maximum likelihood function defined in Equation \ref{directionlikelihoodeq}.

\begin{equation}
    \mathcal{L}(\vec{d})=\sum_{i}^{N_{20}} \log (f(\cos\theta_{i}, E))\times\frac{\cos\theta_{i}}{a(\theta_{i})}
    \label{directionlikelihoodeq}
\end{equation}

$f(\cos\theta_{i},E)$ is the expected distribution of the angle between the vector of the direction $\vec{d}$ of the particle, and the observed Cherenkov photon from the position of the reconstructed vertex. The reason there is a spread in this energy distribution is because while the highest value of this distribution occurs at the cosine of the opening Cherenkov angle of $42\degree$, due to the particle travelling through the water being Coulomb scattered multiple times, there is a variation in the angle because of the varying particle energy. $N_{20}$ is the number of hits whose residual hit time is within 20ns of the time of the reconstructed event, which is used in order to reduce the amount dark noise and scattered photons contribute to the direction reconstruction calculation. The  variable $a(\theta_{i})$ is used in the second term in Equation \ref{directionlikelihoodeq}, and it is linked to the angle of incidence of the photon on the PMT $a(\theta_{i})$, and is a correction factor stemming from the acceptance of PMTs and therefore linked to the shape of the PMT and it's acrylic case. 






  \chapter{Super-Kamiokande Detector Calibration}
\label{chp:superkcalib}


In order to achieve optimal event reconstruction for physics analyses, calibration of the Super-Kamiokande detector is crucial. For example, when constructing Monte Carlo simulations of certain processes in the detector, facets of the experiment such as properties of the water, photomultiplier tube response and the inner detector and outer detector electronics are all calibrated so that input parameters for the Monte Carlo simulations can be obtained. This chapter will concern itself with the inner and outer detector calibration, including photomultiplier tube and electronics calibration, PMT gain calibration, quantum efficiency determination and hit timing and charge information calibration. 

\subsection{Inner detector calibration}
\subsubsection{Electronics and photomultiplier tube calibration}

Understanding the timing information from the hit photomultiplier tubes depends on how well the charge from the hit PMT is calculated. To conceive charge calibration, a quantity called photomultiplier tube ''gain" must be calculated. ''Gain" is the conversion factor from the number of photoelectrons produced by the hit PMT and charge, and calibration of this quantity is what interpretation of very high energy events (TeV scale) rely on. Quantum efficiency is another quantity used for the calibration of low energy physics events (such as detection of solar neutrinos), due to them consisting of single photoelectron (single-pe) hits: it is the ratio of the number of the number of photoelectrons emitted by the cathode to the number of photons that are incident on the photomultiplier tube window. Quantum efficiency is particularly useful for low energy events because the number of photons arriving at the photomultiplier tube window is small. Super-Kamiokande calibration converts this measure of quantum efficiency into ''QE" by multiplying the quantum efficiency by the collection efficiency of the photoelectrons onto the first dynode inside the PMT \ref{abeCalibrationSuperKamiokandeDetector2014}. Knowing the gain and QE of each PMT in the detector is important in order to accurately measure the output charge from each individual PMT, which is done by first calculating the relative gain gain difference among all PMTs and then work out the average gain difference over all PMTs in the detector. After this, the variation away from this average gain value can be calculated for each seperate inner detector photomultiplier tube, and the gain value for each can be extracted. 

The relative gain difference is calculated by two measurements using a light source to produce constant-intensity flashes. The first measurement involves using the light source to produce high-intensity flashes so that all photomultiplier tubes in the detctor gets a certain number of photons, and the second measurement has the light source produce low-intensity flashes so that only a few PMTs are hit. teh first measurement provides an average charge value ($Q_{o b s}(i)$) for each inner detector PMT, while the second measurement gives single photoelectron hits, providing a number of times ($N_{o b s}(i)$) that a single PMT gives a charge which is greater than the PMT threshold value. Equation \ref{eq:gaineq} shows how these two values are calculated from the the high and low intensity flash values ($I$), the acceptance of the PMT(i) ($a(i)$), the QE value of the PMT ($\varepsilon_{q e}$) and the PMT gain $G$. 
****fix this make this equation multiline*******
\begin{equation}
    Q_{o b s}(i) \quad \propto \quad I_{high} \times a(i) \times \varepsilon_{q e}(i) \times G(i)
    \newline
    N_{o b s}(i) \quad \propto \quad I_{low} \times a(i) \times \varepsilon_{q e}(i)
\label{eq:gaineq}
\end{equation}

Therefore, by simply dividing these two values of $Q_{o b s}(i)$ and ($N_{o b s}(i)$) the average gain over all PMTs can be calculated.  Figure \ref{fig:relativegain} shows the spread of the relative gain over all the PMTs. 

\begin{figure}
    \includegraphics[width=\textwidth]{Figures/relativegain.png}
\caption{Relative gain of PMTs in Super-Kamiokande}
    \label{fig:relativegain}
\end{figure}


  \chapter{The UK Light Injection System}
\label{chp:ukli}

As mentioned in Chapter 4, the Korean laser system is used to measure the scattering and absorbtion coefficients in Super-Kamiokande. The UK calibration group's efforts have been focussed on improving the data analysis method and improving the accuracy of the water coefficient measurements. To aid this effort a new UK developed Light Injection (UKLI) system was installed into Super-Kamiokande during the refurbishment that ocurred in the summer of 2018. The ultimate goal is to install the light injection system in Hyper-Kamiokande, which was the purpose of its initial development. The UK system has its own set of optics, which unlike the Korean system, involves optics with multiple beam spot diameters, which will be described in detail in this Chapter, along with the electronics involved in the production of the system. Much like the Korean laser system method of measuring the absorption and scattering measurements described in Chapter 4, the measurements from the UK system involves the generation of Monte Carlo specific to the diameter of the beam spots produced from the UKLI, the production methods of which will also be discussed in detail in this chapter. Much like for the Korean system, a monitoring system was introduced to observe the light injection output for the different optics, and was also of interest during the Gadolinium loading period to examine changes due to the addition of gadolinium sulphate in the detector. 

\section{The UK Light Injection System Electronics}

The electronics setup architecture for the UK Light Injection system is made of sixteen light emitting diode boards which are each coupled to three optical fibres, as well as a monitor PMT, an optical power meter and two motherboards as shown in Figure \ref{fig:ukli_system_architecture}. 

\begin{figure}
    \centering
    \includegraphics[width=0.7\textwidth]{Figures/ukli_system_architecture.png}
    \caption{UKLI electronics system architecture, showing optical fibre and motherboard couplings and QBEE connections.}
    \label{fig:ukli_system_architecture}
\end{figure}

The light being pulsed has a wavelength of 435 nm and is produced by light emitting diodes which are controlled by Field Programmable Gate Arrays and uses sixteen LED pulser boards placed on two motherboards (8 LED pulser boards on each) which controls which channel to send a signal to. Fifteen of the sixteen channels deliver light into the detector, while the remaining channel sends light to a seperate monitoring system. There are three optical fibres each LED is coupled to: firstly, a channel connected to the monitor PMT, secondly to an optical fibre which sends light into the Super-Kamiokande detector, and finally to the on-board photodiode monitor. The monitor PMT is a very small 2 inch Hamamatsu PMT which has a peak sensitivity to 400 nm wavelength light. A signal is sent from the light emitting diode to the monitor PMT and this information is sent to one of the QBEE (QTC-based electronics with Ethernet) channels for the detector - the charge recorded by the monitor PMT is meant to be used as a normalisation factor for calculations of the absorption and scattering water parameters. 
\newline

The monitor PMT is contained inside a custom made 3D printed box to make sure there is no external light reaching it. There are nineteen channels which monitor the input of the PMT, and these are kept in place against the PMTs photocathode. Fifteen of these fibres are connected to light emitting diodes which give light to the detector, one channel is coupled to the LED board for monitoring the system and the last three channels are reserve. The channel which is coupled to the LED board for monitoring of the syste is used for calibrating the monitor PMT, where the signal from this channel is inputted into the optical power meter. This fibre and the fifteen fibres that are connected to the LEDs that give light to the detector are linked to a photodiode monitor board (PMD). There are two PMDs per motherboard, therefore four in total, which record the output from the LED channels and can switch off a channel if it is producing light when it is not meant to which stops light leaking into the tank. 

\subsection{UKLI System Optics}

Unlike the Korean laser system mentioned in Chapter 4 that injects light into the detector using an optical fibre which has an opening angle of 4 degrees, the UK Light Injection system contains three different types of light injection optics, with each having a different opening angle: a bare fibre, a collimator and a diffuser. This range of optics can accomadate a larger variety of calibration measurements, and better suit the multiple applications of the light injection system, including it being better more monitoring purposes. 

\subsubsection{The Collimator Optic}

A 2-degree opening half angle is achieved by the collimator optic by using a graded index lens (GRIN) connected to a bare fibre optic cable - this GRIN lens reduces the opening angle of the light coming from the fibre optic. A schematic of the collimator design can be seen in Figure \ref{fig:collimator_schematic}. The GRIN lens is kept in position within the lens mount where a HAFC connector is drilled in to take a lens through the centre of the hole inside it, and there is a 2.35 mm diameter opening in front of the glass window in front of the GRIN lens in order to reduce background light which is not on-axis. All holes drilled into the collimator setup are filled with epoxy to prevent water from damaging the components. 

\begin{figure}
    \centering
    \includegraphics[width=0.7\textwidth]{Figures/collimator_schematic.png}
    \caption{Collimator schematic including the end-cap, lens mount and mounting sleeve structures.}
    \label{fig:collimator_schematic}
\end{figure}

Having such a confined beam allows for a very exact target position on the tank wall, and by decreasing the size of the target beam spot compared to the bare optical fibre, less hit PMTs outside of the target spot are excluded meaning that there are more hits in the water calibration data. Due to the fact that the very collimated beams mean that there is no overlap of beam spots, there can be measurements of the water scattering and absorption coefficients made that are position dependent, allowing for an observation of how the water parameters depend on depth in the tank. 

\subsubsection{The Diffuser Optic}

The diffuser optic is a wide angled beam with a opening half-angle of 40 degrees. It allows for water coefficient measurement calibration and also measurements of PMT gain over time and light attenuation length in water and allows for illumination of several hundred PMTs at once. The diffuser optic is made of a Poly(methyl methacrylate) (PMMA) ball which is a piece of acrylic resin in the shape of a half sphere and it contains PMMA particles suspended in silicone gel, similar to the design of the ''laserball'' used to calibrate the SNO+ detector \cite{Moffat_2005}. As a result of this, the mean free path of light that is injected into the center of the diffuser ball is much shorter than the radius of the ball, so each photon scatters so many times off the PMMA particles at random that the exiting angle of the light is also random. This randomised exit direction of the photons means that the beam timing and intensity would be uniform.  
\newline
The enclosure for the diffuser ball had to be watertight and also able to withstand the pressure of being deployed at the bottom of Super-Kamiokande without the quality of the beam profile being reduced, with the requirement being that it can withstand a maximum pressure of 4 bar. Figure \ref{fig:diffuser_photo} shows a photograph of one of the diffusers used, and one of the diffuser ball enclosures. The enclosure is made of high grade stainless steel and using a chemical and water resistant epoxy resin, all the components were ensured to be watertight. 

\begin{figure}
    \centering
    \includegraphics[width=0.7\textwidth]{Figures/diffuser_photo.png}
    \caption{Photograph of the diffuser by itself (left), empty diffuser enclosure (centre) and diffuser inside enclosure (right).}
    \label{fig:diffuser_photo}
\end{figure}

\subsubsection{Bare Fibre and Optical Plate}

The bare fibre injector are 1mm step index fibres, and are approximately 20 cm in length and are used for validation purposes with the bare fibres in the Korean optical calibration system. These short fibres are screwed into the back end of the optical plate that the collimator and diffuser optics are mounted on.

Using a kinematic mount (shown in Figure \ref{fig:mount}) the optics plate was attached to the Super-Kamiokande PMT structure - due to the position of the Korean system laser injectors there were two different sizes of optics plates required, as two of the barrel injectors are on a lower PMT rail compared with the other three, meaning that the optical plates on the lower PMT rail include a short piece of metal (known as a ''spacer'') which raises the height of the rail an extra 2.5 inches to remove any interference between the PMTs and the structure.

\begin{figure}
    \centering
    \includegraphics[width=0.7\textwidth]{Figures/mount.png}
    \caption{Schematic and technical drawings of the optics plate and mount provided by Warwick University}
    \label{fig:mount}
\end{figure}


\subsubsection{Soak Testing Components}

All light injector components that would come into contact with the ultra-pure water in Super-Kamiokande and the gadolinium sulphate doped water went under strenous water contamination testing. Samples of the ultra-pure water were taken from the detector, as well as a control sample of the same water and also a solution of pure water from Super-Kamiokande doped with one gram of gadolinium sulphate and put in 500 ml bottles. The component samples under examination were left to soak in the bottles which were kept refrigerated at the same temperature as the water in Super-Kamiokande (approximately 13 degrees Celcius.) After three months of soaking and intermittent checking of the components by eye, the transmittence of the component (namely the stainless steel used in the optical plate and the PMMA used in the diffuser) were checked using a spectrometer. Comparisons with control components showed no degradation of transmittance. 

\subsubsection{Results of optics test stands}

In order to test the collimator and diffuser optics, scans of the angular distributions were made using diffuser and collimator test-stands. These scans would show angular distributions of the light intensity, and details of the setup of each test stand are given here. For the diffuser test-stand the angular distribution of the light output was captured using the setup shown in Figure \ref{fig:diffuser_test_stand}. A birds-eye view schematic of the setup is shown in Figure \ref{fig:schematic_dif}.

\begin{figure}
    \centering
    \includegraphics[width=0.7\textwidth]{Figures/diffuser_test_stand.png}
    \caption{Setup of the diffuser test stand provided by Warwick University}
    \label{fig:diffuser_test_stand}
\end{figure}

\begin{figure}
    \centering
    \includegraphics[width=0.7\textwidth]{Figures/schematic_dif.png}
    \caption{Birds-eye view schematic of the diffuser test stand}
    \label{fig:schematic_dif}
\end{figure}

The test stand setup for the diffuser optics consists of a test diffuser ball placed inside a diffuser enclosure, a rotation stage which allows for the movement of the diffuser between -40 and 40 degrees, and a PMT used for pulse intensity measurement set up 250 mm away from the diffuser. An optical fibre couples the diffuser under test to a laser set to a wavelength of 450 nm. 

The setup for the collimator test stand at the University of Warwick is shown in Figure \ref{fig:coll_test_stand}.  The setup for the collimator optic captures the beam cross section by moving a CMOS camera along the beam direction. 

\begin{figure}
    \centering
    \includegraphics[width=0.7\textwidth]{Figures/coll_test_stand.png}
    \caption{Setup of the collimator test stand provided by Warwick University}
    \label{fig:coll_test_stand}
\end{figure}

Figure \ref{fig:diffuser_TF1} shows distributions provided by this test stand data: preliminary TF1 fits made by ROOT to the light profiles for the diffuser.  The x-axis scale shows the polar angle the rotation stand moves through, while the y-axis scale shows the average of the integrated area under all the pulses recorded by the PMT and is not normalised. 

\begin{figure}[!htbp]
    \centering
    
    \caption{Light profiles for the diffuser optics provided by The University of Warwick}\label{fig:diffuser_TF1}
    
    \subfloat[]{\includegraphics[width=0.33\textwidth]{Figures/B1_diffuser_fit.PNG}}\hfill
    \subfloat[]{\includegraphics[width=0.33\textwidth]{Figures/B2_diffuser_fit.PNG}}\hfill
    \subfloat[]{\includegraphics[width=0.33\textwidth]{Figures/B3_diffuser_fit.PNG}}
    
    \subfloat[]{\includegraphics[width=0.33\textwidth]{Figures/B4_diffuser_fit.PNG}}%
    \hspace*{0.005\textwidth}%
    \subfloat[]{\includegraphics[width=0.33\textwidth]{Figures/B5_diffuser_fit.PNG}}
    
\end{figure}

Figure \ref{fig:collimator_TF1} shows TF1 fits made by ROOT to the light profiles for the collimator. The angular distributions shown give the distribution of the polar angle in degrees of light intensity which are relative to the virtual position from which the light cone originates, averaged over all the orientations of the azimuthal angle. Figure \ref{fig:collimator_test_schematic} shows the light cone from the collimator and the direction in which the measurements are taken with the CMOS camera.

\begin{figure}
    \centering
    \includegraphics[width=0.7\textwidth]{Figures/collimator_test_schematic.png}
    \caption{Diagram showing the direction in which measurements of the light intensity were taken}
    \label{fig:collimator_test_schematic}
\end{figure}

\begin{figure}[!htbp]
    \centering
    
    \caption{Light profiles for the collimator optics provided by The University of Warwick}\label{fig:collimator_TF1}
    
    \subfloat[]{\includegraphics[width=0.33\textwidth]{Figures/B1_collimator_fit.PNG}}\hfill
    \subfloat[]{\includegraphics[width=0.33\textwidth]{Figures/B2_collimator_fit.PNG}}\hfill
    \subfloat[]{\includegraphics[width=0.33\textwidth]{Figures/B3_collimator_fit.PNG}}
    
    \subfloat[]{\includegraphics[width=0.33\textwidth]{Figures/B4_collimator_fit.PNG}}%
    \hspace*{0.005\textwidth}%
    \subfloat[]{\includegraphics[width=0.33\textwidth]{Figures/B5_collimator_fit.PNG}}
    
\end{figure}

\subsubsection{UK Calibration data}

In September of 2019 and November of 2019 two sets of test data were taken of the collimator, the diffuser and the bare fibre optic (the B2 bare fibre). For the September 2019 data, all of the data for the B1-B5 optics was taken with 100,000 events, however due to the B3 and B5 collimator from the September 2019 data showing a very weak signal, 150,000 events were taken with the B3 collimator data for and 200,000 events were taken with the November 2019 data. Using an event display developed by the University of Warwick, occupancy plots of the test data sets were produced. Figure \ref{fig:occupancy_coll} shows the occupancy plots for the collimator optic from the November 2019 dataset showing the beam spot inside the unrolled volume of the Super-Kamiokande detector. Similarly, Figure \ref{fig:occupancy_diffuser} shows the occupancy plots for the diffuser optic from the November 2019 dataset. The graph in the bottom right hand corner of the occupancy plots show the corrected time-of-flight plots for the PMT hits from the injector. 

\begin{figure}[!htbp]
    \centering
    
    \caption{Occupancy plot for the collimator optics from the UKLI November 2019 test run} \label{fig:occupancy_coll} 
    
    \subfloat[]{\includegraphics[width=0.49\textwidth]{Figures/B1_occupancy_coll.PNG}} \hfill
    \subfloat[]{\includegraphics[width=0.49\textwidth]{Figures/B2_occupancy_coll.PNG}} \par
    \subfloat[]{\includegraphics[width=0.49\textwidth]{Figures/B3_occupancy_coll.PNG}} \hfill
    \subfloat[]{\includegraphics[width=0.49\textwidth]{Figures/B4_occupancy_coll.PNG}} \par
    \subfloat[]{\includegraphics[width=0.49\textwidth]{Figures/B5_occupancy_coll.PNG}}
    
\end{figure}

\begin{figure}[!htbp]
    \centering
    
    \caption{Occupancy plot for the diffuser optics from the UKLI November 2019 test run} \label{fig:occupancy_diffuser} 
    
    \subfloat[]{\includegraphics[width=0.49\textwidth]{Figures/B1_occupancy_diff.PNG}} \hfill
    \subfloat[]{\includegraphics[width=0.49\textwidth]{Figures/B2_occupancy_diff.PNG}} \par
    \subfloat[]{\includegraphics[width=0.49\textwidth]{Figures/B3_occupancy_diff.PNG}} \hfill
    \subfloat[]{\includegraphics[width=0.49\textwidth]{Figures/B4_occupancy_diff.PNG}} \par
    \subfloat[]{\includegraphics[width=0.49\textwidth]{Figures/B5_occupancy_diff.PNG}}
    
\end{figure}

In addition to test data being taken, there is also an ''autocalib'' system used for long term monitoring of the water parameters in Super-Kamiokande by the Korean laser system. In early 2020 the autocalib scheduler was modified to incorporate data taking by the UKLI system which was very useful for Gadolinium loading calibration purposes (which will be expanded upon in the UKLI monitoring system section) but also in the longer term, it will be useful in the monitoring of daily/weekly water coefficient property measurements, investigation of depth dependence with respect to the water properties and PMT property calibration. Figure \ref{fig:autocalib} shows the schedule for autocalib, and the black dashed lines show the position of the UK barrel collimator and diffusers with respect to the other autocalib data taking streams. The horizontal blue line shows the length of the one autocalib cycle, which is about 4.6 seconds, with each UKLI optic having about 3310 events per day.

\begin{figure}
    \centering
    \includegraphics[width=0.7\textwidth]{Figures/autocalib.png}
    \caption{Schematic showing position of UKLI in autocalib scheduler: the black dashed lines show the UKLI B1-B5 collimator and diffuser optics and the horizontal blue line shows the length of one autocalib cycle.}
    \label{fig:autocalib}
\end{figure}

Figures \ref{fig:occupancy_coll_auto} and \ref{fig:occupancy_diff_auto} shows the occupancy plots for autocalib data taken in July 2020: as can be seen in the text in the upper left hand corner, the number of events in the run is a lot less than the 100,000 events or so taken in the test runs, however they are more than sufficient for monitoring purposes.

\begin{figure}[!htbp]
    \centering
    
    \caption{Occupancy plot for the collimator optics from the UKLI Autocalin July 2020 run} \label{fig:occupancy_coll_auto} 
    
    \subfloat[]{\includegraphics[width=0.49\textwidth]{Figures/B1_occupancy_coll_auto.PNG}} \hfill
    \subfloat[]{\includegraphics[width=0.49\textwidth]{Figures/B2_occupancy_coll_auto.PNG}} \par
    \subfloat[]{\includegraphics[width=0.49\textwidth]{Figures/B3_occupancy_coll_auto.PNG}} \hfill
    \subfloat[]{\includegraphics[width=0.49\textwidth]{Figures/B4_occupancy_coll_auto.PNG}} \par
    \subfloat[]{\includegraphics[width=0.49\textwidth]{Figures/B5_occupancy_coll_auto.PNG}}
    
\end{figure}

\begin{figure}[!htbp]
    \centering
    
    \caption{Occupancy plot for the diffuser optics from the UKLI Autocalib July 2020 run} \label{fig:occupancy_diff_auto} 
    
    \subfloat[]{\includegraphics[width=0.49\textwidth]{Figures/B1_occupancy_diff_auto.PNG}} \hfill
    \subfloat[]{\includegraphics[width=0.49\textwidth]{Figures/B2_occupancy_diff_auto.PNG}} \par
    \subfloat[]{\includegraphics[width=0.49\textwidth]{Figures/B3_occupancy_diff_auto.PNG}} \hfill
    \subfloat[]{\includegraphics[width=0.49\textwidth]{Figures/B4_occupancy_diff_auto.PNG}} \par
    \subfloat[]{\includegraphics[width=0.49\textwidth]{Figures/B5_occupancy_diff_auto.PNG}}
    
\end{figure}


\subsubsection{UKLI MC production}
After making these fits to data the next aim was to use these light profiles as an input to SKDETSIM, the Super Kamiokande Detector Simulator. SKDETSIM uses GEANT3 (GEometry ANd Tracking 3) to simulate what the particles in each event would do inside the detector, and tracks the particle's trajectories and energy loss. Simulating the light injection from the UKLI system in SKDETSIM was done in a similar way to the Korean method of producing Monte Carlo: the same versions of the calibration scripts were used however, small modifications were made to them and to the version of SKDETSIM used to simulate the input photons from the system in the detector. The calibration scripts used for the Korean and UKLI systems both allow for the number of events and the number of injected photons to be set. In order to generate many Monte Carlo files with the absorption, Rayleigh and Mie scattering parameters varied, the ratio of these paramaters ''abs'', ''ray'' and ''mie'', to the value of these parameters currently in the detector simulation can also be set, where a value of ''1.0'' is thr SKDETSIM default value. The top-bottom asymmetry ''TBA'', mentioned in Chapter 4, can also be set, along with the wavelength of the injected light.
\newline
The first step regarding making UKLI based Monte Carlo was changing the position of the injector locations from the positions of the Korean system (shown in Table \ref{table:korean_loc}) to that of the UKLI system (shown in Table \ref{table:UKLI_loc}), due to the fact that the spacing gap rungs between the UK system and the Korean system is +70.7 cm for the B1, B2, B3 barrel injectors and -70.7 cm for the B4 and B5 injectors. The opening angle of the injectors determined by the simulation also had to be changed because the simulation had to accomodate the fact that the injectors for the UKLI system now consisted of three different opening angles for the collimator, diffuser and bare fibre optics, instead of the just the bare fibre optics used in the Korean system. The way the opening angle for the Korean system was set in the calibration scripts was using the "SIGBM" parameter which stands for the sigma of the input injector beam, however an entirely new method of determining the opening angle of the beam was needed to include the information from the light profiles taken from the test stands at Warwick. In order to understand the SIGBM parameter further however, the relationship between opening angle in degrees and SIGBM was plotted by outputting the SIGBM value and the angle to a text file during laser generation Monte Carlo production (shown in Figure \ref{fig:sigbm_angle}.)

\begin{table}[htp]
    $$
\begin{array}{|c|c|c|c|c|c|}  
    \hline \hline{\text {Korean Barrel Injector }} & {\text { x (cm)}} & {\text {y (cm)}} & {\text {z (cm)}}  \tabularnewline
    \hline \text { B1 } & 1490.73 & 768.14 & 1232.25 \\
    \hline \text { B2 } & 1490.73 & 768.14 & 595.95 \\
    \hline \text { B3 } & 1490.73 & 768.14 & -40.35 \\
    \hline \text { B4 } & 1490.73 & 768.14 & -605.95 \\
    \hline \text { B5 } & 1490.73 & 768.14 &  -1242.25 \\
    \hline \hline 
\end{array}
    $$
\caption{Barrel injector positions (x,y,z) of the Korean injectors in cm} 
\label{table:korean_loc}
\end{table}

\begin{table}[htp]
    $$
\begin{array}{|c|c|c|c|c|c|}  
    \hline \hline{\text {UKLI Barrel Injector }} & {\text { x (cm)}} & {\text {y (cm)}} & {\text {z (cm)}}  \tabularnewline
    \hline \text { B1 } & 1490.73 & 768.14 & 1302.95 \\
    \hline \text { B2 } & 1490.73 & 768.14 & 666.2 \\
    \hline \text { B3 } & 1490.73 & 768.14 & -111.05 \\
    \hline \text { B4 } & 1490.73 & 768.14 & -676.65 \\
    \hline \text { B5 } & 1490.73 & 768.14 &  -1313.95 \\
    \hline \hline 
\end{array}
    $$
\caption{Barrel injector positions (x,y,z) of the UKLI injectors in cm} 
\label{table:UKLI_loc}
\end{table}

\begin{figure}
    \centering
    \includegraphics[width=0.7\textwidth]{Figures/sigbm_angle.png}
    \caption{Plot showing opening angle of the bare fibre injector for the Korean system in degrees vs ''SIGBM'' parameter}
    \label{fig:sigbm_angle}
\end{figure}

In order to validate the positions of the targets for the UKLI system, and the relationship between the SIGBM parameter and output angle, producing charge weighted histograms from UKLI test runs is very helpful. It allows us to explore the shape of the beam profile and intensity. Figures \ref{fig:charge_weighted_nov_sept_B1} and \ref{fig:charge_weighted_nov_sept_B4} shows the charge weighted z-profiles for the September and November 2019 datasets for the B1 and B4 collimator injectors, where the blue dashed line shows the expected target position. These are produced by selecting hit PMTs which are greater than 2 m away from the injector (to avoid including PMT hits from backscattered light), and filling the histogram with the z-position of the hit PMT and the number of hits the hit PMT recieves multiplied by the corrected charge from the PMT. The corrected PMT charge is calculated using Equation \ref{eq:gain_correction}, where the gain correction value is taken from official Super-Kamiokande gain tables. 

\begin{equation}
    q/(1 + gain)
\label{eq:gain_correction}
\end{equation}

\begin{figure}
    \centering
    \includegraphics[width=0.7\textwidth]{Figures/charge_weighted_nov_sept_B1.PNG}
    \caption{Charge weighted z profile plots for the B1 collimator UKLI injector optic}
    \label{fig:charge_weighted_nov_sept_B1}
\end{figure}

\begin{figure}
    \centering
    \includegraphics[width=0.7\textwidth]{Figures/charge_weighted_nov_sept_B4.PNG}
    \caption{Charge weighted z profile plots for the B4 collimator UKLI injector optic}
    \label{fig:charge_weighted_nov_sept_B4}
\end{figure}

The B1 and B4 collimator optics give the largest peaks in the charge weighted plots, while the B3 and B5 optics give the weakest peaks, practically indistunguishable from the background pedestal (event with the additional number of events taken during the November data taking runs.) The charge weighted plots were used to validate the relationship between the opening angle of the injector beam in the Monte Carlo production calibration scripts by generating Monte Carlo scripts with varying values of the SIGBM parameter (while keeping the absorption and scattering parameters the same). Fitting the charge weighted profile plots with a Gaussian and using the position of the injector target and the edge of the beam spot, the opening angle of the beam in degrees could be determined.

The width of the beam needed to be defined in order to do this however, and the standard beam radius definition is shown as in Figure \ref{fig:beam_width}. This schematic shows that the width of the beam was defined as using $1/e^2$ for the edge of the beam width, so it is taken as the point where the intensity drops to 0.135 of the Gaussian peak value. While there is good agreement at small opening angles, there is a slight discrepancy between this plot and Figure \ref{fig:sigbm_angle} at larger opening angles, where in Figure \ref{fig:sigbm_angle} the angle in degrees is consistently greater than in Figure \ref{fig:sigbm_angle_validation} for the same value of SIGBM, and this is due to SIGBM being defined as the maximum opening angle, however for small opening angles as for the collimator, the comparison is valid.

\begin{figure}
    \centering
    \includegraphics[width=0.7\textwidth]{Figures/beam_width.PNG}
    \caption{Schematic showing how the width of the MC beam was defined}
    \label{fig:beam_width}
\end{figure}

\begin{figure}
    \centering
    \includegraphics[width=0.7\textwidth]{Figures/sigbm_angle_validation.PNG}
    \caption{Plot of the sigbm parameter vs angle in degrees produced using calculation of the opening angle from the beam width}
    \label{fig:sigbm_angle_validation}
\end{figure}

After using the charge weighted hit plots to understand the SIGBM parameter and how the opening angle is generated by SKDETSIM more thoroughly, the next step involved using the profiles from the Warwick optic test stands (shown in Figures \ref{fig:collimator_TF1} and \ref{fig:diffuser_TF1}) in the production of this opening angle in the detector simulation. This was done by treating the profiles in Figures \ref{fig:collimator_TF1} and \ref{fig:diffuser_TF1} as Probablity Distribution Functions (PDFs) and using inverse transform sampling to make the detector simulation sample at random from it. Inverse transform sampling is a method for generating random numbers from any probability distribution by using the inverse of its cumulative distribution $F^{-1}(x)$. For continuous distributions, such as the results from the collimator and diffuser optics test stands, the algorithm for inverse transform sampling is simple. Firstly, a random variable $U$ is uniformly distributed between [0,1], and secondly the relation $X = F^{-1}_{x}(U)$ would then produce a distribution $X$ following the original probability distribution function, i.e. that of the original PDFs from the optics test stands. 

The first step is to produce the CDFs from the PDFs from the optic test stand profile tests. Figure \ref{fig:collimator_TF1} shows that the original fits to the collimator data did not reach 4 degrees, and as a result the PDFs produced needed to be linearly extrapolated from 3.5 degrees where the measurements cut off to reach 4 degrees. Figure \ref{fig:PDF_CDF_coll} shows the PDFs and CDFs produced from the collimator data, and Figure \ref{fig:PDF_CDF_diff} shows the PDFs and CDFs produced from the diffuser data. The CDFs are normalised with a max of one using min-max scaling.

\begin{figure}[!htbp]
    \centering
    
    \caption{PDFs and corresponding CDFs for the B1 - B5 collimators} \label{fig:PDF_CDF_coll} 
    
    \subfloat[]{\includegraphics[width=0.49\textwidth]{Figures/B1_coll_pdf.png}} \hfill
    \subfloat[]{\includegraphics[width=0.49\textwidth]{Figures/B1_coll_cdf.png}} \par
    \subfloat[]{\includegraphics[width=0.49\textwidth]{Figures/B2_coll_pdf.png}} \hfill
    \subfloat[]{\includegraphics[width=0.49\textwidth]{Figures/B2_coll_cdf.png}} \par
    \subfloat[]{\includegraphics[width=0.49\textwidth]{Figures/B3_coll_pdf.png}} \hfill
    \subfloat[]{\includegraphics[width=0.49\textwidth]{Figures/B3_coll_cdf.png}} \par 
\end{figure}
\begin{figure}[!htbp]
    \ContinuedFloat
    \subfloat[]{\includegraphics[width=0.49\textwidth]{Figures/B4_coll_pdf.png}} \hfill
    \subfloat[]{\includegraphics[width=0.49\textwidth]{Figures/B4_coll_cdf.png}} \par
    \subfloat[]{\includegraphics[width=0.49\textwidth]{Figures/B5_coll_pdf.png}} \hfill
    \subfloat[]{\includegraphics[width=0.49\textwidth]{Figures/B5_coll_cdf.png}} 
    
    
\end{figure}

\begin{figure}[!htbp]
    \centering
    
    \caption{PDFs and corresponding CDFs for the B1 - B5 diffusers} \label{fig:PDF_CDF_diff} 
    
    \subfloat[]{\includegraphics[width=0.49\textwidth]{Figures/B1_diff_pdf.png}} \hfill
    \subfloat[]{\includegraphics[width=0.49\textwidth]{Figures/B1_diff_cdf.png}} \par
    \subfloat[]{\includegraphics[width=0.49\textwidth]{Figures/B2_diff_pdf.png}} \hfill
    \subfloat[]{\includegraphics[width=0.49\textwidth]{Figures/B2_diff_cdf.png}} \par
    \subfloat[]{\includegraphics[width=0.49\textwidth]{Figures/B3_diff_pdf.png}} \hfill
    \subfloat[]{\includegraphics[width=0.49\textwidth]{Figures/B3_diff_cdf.png}} 
\end{figure}
\begin{figure}[!htbp]
    \ContinuedFloat
    \subfloat[]{\includegraphics[width=0.49\textwidth]{Figures/B4_diff_pdf.png}} \hfill
    \subfloat[]{\includegraphics[width=0.49\textwidth]{Figures/B4_diff_cdf.png}} \par
    \subfloat[]{\includegraphics[width=0.49\textwidth]{Figures/B5_diff_pdf.png}} \hfill
    \subfloat[]{\includegraphics[width=0.49\textwidth]{Figures/B5_diff_cdf.png}}     
\end{figure}

After producing the normalised CDFs, the inverse of these CDFs are calculated - Figures \ref{fig:PDF_CDF_inv_coll} shows the comparison of the normalised CDF data for the collimators (top sublot, shown in blue) and the polynomial fit to the CDFs for the collimators (top subplot, shown in red), and the inverse CDF function (bottom subplot shown in green) and the polynomial fit to this inverse CDF function (bottom subplot shown in purple). Figures \ref{fig:PDF_CDF_inv_diff} shows the same plots for the B1-B5 diffusers. 

\begin{figure}[!htbp]
\centering
    
\caption{CDF and inverse CDF fits for the B1 - B5 collimator PDFs} \label{fig:PDF_CDF_inv_coll} 
    
    \subfloat[]{\includegraphics[width=0.49\textwidth]{Figures/B1_inv_coll_cdf.png}} \hfill
    \subfloat[]{\includegraphics[width=0.49\textwidth]{Figures/B2_inv_coll_cdf.png}} \par
    \subfloat[]{\includegraphics[width=0.49\textwidth]{Figures/B3_inv_coll_cdf.png}} \hfill
    \subfloat[]{\includegraphics[width=0.49\textwidth]{Figures/B4_inv_coll_cdf.png}} \par
    \subfloat[]{\includegraphics[width=0.49\textwidth]{Figures/B5_inv_coll_cdf.png}} 

\end{figure}

\begin{figure}[!htbp]
    \centering
        
    \caption{CDF and inverse CDF fits for the B1 - B5 diffuser PDFs} \label{fig:PDF_CDF_inv_diff} 
        
        \subfloat[]{\includegraphics[width=0.49\textwidth]{Figures/B1_inv_diff_cdf.png}} \hfill
        \subfloat[]{\includegraphics[width=0.49\textwidth]{Figures/B2_inv_diff_cdf.png}} \par
        \subfloat[]{\includegraphics[width=0.49\textwidth]{Figures/B3_inv_diff_cdf.png}} \hfill
        \subfloat[]{\includegraphics[width=0.49\textwidth]{Figures/B4_inv_diff_cdf.png}} \par
        \subfloat[]{\includegraphics[width=0.49\textwidth]{Figures/B5_inv_diff_cdf.png}} 
    
\end{figure}

After producing the fits to the inverse cumulative distribution functions, these funtions were inputted into the detector simulation, SKDETSIM. Removing the SIGBM parameter from the calibration scripts meant there needed to be a new way with which to generate the angle at which the photons were produced in the simulation, and this is where the inverse transform sampling occurs for the relevant fit for the selected injector and optic type. Using the same event display used to produce the occupancy plots for the autocalib data and the test run data, occupancy plots of the Monte Carlo were produced. These are shown for the B1 - B5 collimators in Figure \ref{fig:ukli_mc_coll}, and for the B1 - B5 diffusers in Figure \ref{fig:ukli_mc_diff}. These MC were produced with the abs, ray and mie parameters in the calibration scripts set to 1.0, 1.0, and 1.0 (i.e. the current detector simulation values) and the top-bottom asymmetry parameter set to 7.598, which is the most recently tuned value of this parameter. Because the original profiles from the test stands were taken in air, adjustments were made so that the refractive index of the water in the detector was taken into account when implementing the inverse CDFs into the detector simulation. 

\begin{figure}[!htbp]
    \centering
    
    \caption{Monte Carlo simulations of the B1 -B5 collimator injectors}\label{fig:ukli_mc_coll}
    
    \subfloat[]{\includegraphics[width=0.33\textwidth]{Figures/ukli_mc_B1.PNG}}\hfill
    \subfloat[]{\includegraphics[width=0.33\textwidth]{Figures/ukli_mc_B2.PNG}}\hfill
    \subfloat[]{\includegraphics[width=0.33\textwidth]{Figures/ukli_mc_B3.PNG}}
    
    \subfloat[]{\includegraphics[width=0.33\textwidth]{Figures/ukli_mc_B4.PNG}}%
    \hspace*{0.005\textwidth}%
    \subfloat[]{\includegraphics[width=0.33\textwidth]{Figures/ukli_mc_B5.PNG}}
    
\end{figure}

\begin{figure}[!htbp]
    \centering
    
    \caption{Monte Carlo simulations of the B1 -B5 diffuser injectors}\label{fig:ukli_mc_diff}
    
    \subfloat[]{\includegraphics[width=0.33\textwidth]{Figures/ukli_diff_mc_B1.PNG}}\hfill
    \subfloat[]{\includegraphics[width=0.33\textwidth]{Figures/ukli_diff_mc_B2.PNG}}\hfill
    \subfloat[]{\includegraphics[width=0.33\textwidth]{Figures/ukli_diff_mc_B3.PNG}}
    
    \subfloat[]{\includegraphics[width=0.33\textwidth]{Figures/ukli_diff_mc_B4.PNG}}%
    \hspace*{0.005\textwidth}%
    \subfloat[]{\includegraphics[width=0.33\textwidth]{Figures/ukli_diff_mc_B5.PNG}}
    
\end{figure}

In order to validate the diffuser MC inverse cumulative distribution function output, a uniform distribution was run through the equation for the diffuser inverse CDF fits and the original PDF fit for each diffuser was used to fit the output points from the distribution, showing that inverse transform sampling was done correctly for the PDFs. These are shown for each diffuser in Figure \ref{fig:inv_cdf_check}. 

\begin{figure}[!htbp]
    \centering
    
    \caption{Plots of the result of running a randomly sampled numbers from a uniform distibution through the inverse CDFs (in black) for each diffuser, along with the original PDF fits (in red) for the B1 - B5 diffusers.}\label{fig:inv_cdf_check}
    
    \subfloat[]{\includegraphics[width=0.33\textwidth]{Figures/inv_cdf_check_diff_B1.PNG}}\hfill
    \subfloat[]{\includegraphics[width=0.33\textwidth]{Figures/inv_cdf_check_diff_B2.PNG}}\hfill
    \subfloat[]{\includegraphics[width=0.33\textwidth]{Figures/inv_cdf_check_diff_B3.PNG}}
    
    \subfloat[]{\includegraphics[width=0.33\textwidth]{Figures/inv_cdf_check_diff_B4.PNG}}%
    \hspace*{0.005\textwidth}%
    \subfloat[]{\includegraphics[width=0.33\textwidth]{Figures/inv_cdf_check_diff_B5.PNG}}
    
\end{figure}

After producing the UKLI MC, the next step was to use them to attempt to measure the absorption and scattering parameters in a similar way to the Korean system. As mentioned earlier, this was done by producing time of flight corrected hit timing plots for UKLI MC with multiple different values of absorption, and Rayleigh and Mie scattering parameters and comparing them to the TOF corrected plots for UKLI data. Figures \ref{fig:TOF_abs}, \ref{fig:TOF_ray} and \ref{fig:TOF_mie} show the time-of-flight corrected plots for the UKLI MC B1 collimator, produced with values of the 'abs' and 'ray' calibration script parameter between 0.7 and 1.3 to show the affect that varying these parameters have on the time of flight corrected hits. The y-axis shows the number of hits normalised by the total charge in the detector.

\begin{figure}
    \centering
    \includegraphics[width=0.7\textwidth]{Figures/TOF_abs.PNG}
    \caption{Corrected TOF plot with varied absorption, while rayleigh and mie scattering is set to 1.0}
    \label{fig:TOF_abs}
\end{figure}

\begin{figure}
    \centering
    \includegraphics[width=0.7\textwidth]{Figures/TOF_ray.PNG}
    \caption{Corrected TOF plot with varied Rayleigh scattering, while absorption and Mie scattering is set to 1.0}
    \label{fig:TOF_ray}
\end{figure}

\begin{figure}
    \centering
    \includegraphics[width=0.7\textwidth]{Figures/TOF_mie.PNG}
    \caption{Corrected TOF plot with varied Mie scattering, while absorption and Rayleigh scattering is set to 1.0}
    \label{fig:TOF_mie}
\end{figure}

As shown in Figure \ref{fig:TOF_ray}, the TOF correct timing plots are most affected by the Rayleigh scattering parameter, which affects the amount of hits in the scattered hits region, while varying the absorption parameter mostly affects the height of the reflected peak, as the higher the amount of absorption in the tank, the smaller the number of reflected hits. 

After producing time-of-flight corrected plots for the B1 collimator UKLI MC and overlaying it with the Run 82181 November test run data with the abs, ray, mie parameters set to 1.0, and a TBA value of 7.598, it was clear that there were disgreements between the Monte Carlo and data in both the scattered hits region and the reflected hits peak. In order to remedy this, the time dispersion of the reflected hits peak was varied, in order to shift the distribution and better match up the UKLI Monte Carlo and test run data. 

The time dispersion for the injected photons in the laser generation in SKDETSIM is governed by a Gaussian distributed random number generated using a Box-Muller transform, where additional time dispersion added would be the sigma of this Gaussian, and after a random number is passed through it, the output number would be added to the time for each track step of the photon giving the time dispersion.

The Box-Muller transform is a random number sampling method for making pairs of independent, normally distributed random sources from a source of uniformly distributed numbers (from between usually from between 0 and 1). The form of the Box-Muller method implemented to calculate the added time dispersion is the Marsaglia polar method, which works by choosing two independent and uniformly distributed numbers (u,v) between [-1,+1], so that $s = u^{2} + v^{2}$, and if $s=0$, or $s>=1$, another pair of numbers are chosen. Then the standard normal deviate which is given by $$z_{0}=u \cdot \sqrt{\frac{-2 \ln s}{s}}$$ is multiplied by the chosen value of time dispersion in seconds and added to the mean (set to zero) to give the normally dispersed extra track step time for a photon in the distribution. 

Figure \ref{fig:gauss_time_dispersion} shows the effect of implementing this varying time dispersion on the raw hit timing output from the UKLI MC B1 collimator simulation, with the time dispersion shown for 0 ns, 5 ns, 10 ns, 15 ns, 20 ns and 100 ns. 

\begin{figure}
    \centering
    \includegraphics[width=0.7\textwidth]{Figures/gauss_time_dispersion.PNG}
    \caption{Gaussian distributed time dispersion plots, with varying amounts of time dispersion}
    \label{fig:gauss_time_dispersion}
\end{figure}

Along with implementing the Gaussian distributed time dispersion, introducing a double gaussian for the time dispersion was also looked at, due to the height and sharpness of the reflected peak in the data being something that might be better suited to such a fit. Figure \ref{fig:double_gauss_time_dispersion} shows the raw hit timing output from the UKLI MC B1 collimator simulation with a double gaussian time dispersion, for varying time dispersion values. The x-axis on the plots show the hit time in nanoseconds, the y axis is number of events. 

\begin{figure}
    \centering
    \includegraphics[width=0.7\textwidth]{Figures/double_gauss_time_dispersion.PNG}
    \caption{Double gaussian distributed time dispersion plots, with varying amounts of time dispersion}
    \label{fig:double_gauss_time_dispersion}
\end{figure}

Figure \ref{fig:TOF_dispersion_plots} show these time-of-flight corrected plots with both a gaussian and double gaussian time dispersion put in for varying amounts of time dispersion for 5, 10, 15 and 20 ns. 

\begin{figure}[!htbp]
    \centering
    
    \caption{TOF comparison for B1 collimator UKLI MC and Run 82181 test data for both gaussian and double gausian time dispersion for 5 ns, 10ns, 15 ns and 20 ns}\label{fig:TOF_dispersion_plots}
    
    \subfloat[]{\includegraphics[width=0.33\textwidth]{Figures/time_dispersion_TOF_5ns.PNG}}\hfill
    \subfloat[]{\includegraphics[width=0.33\textwidth]{Figures/time_dispersion_TOF_10ns.PNG}}\hfill
    \subfloat[]{\includegraphics[width=0.33\textwidth]{Figures/time_dispersion_TOF_15ns.PNG}}
    \subfloat[]{\includegraphics[width=0.33\textwidth]{Figures/time_dispersion_TOF_20ns.PNG}}%
    
\end{figure}

For the scattered hits region (region between dashed lines), there is now good agreement between the MC and data, with this region giving chi2/ndf values of less than one, and the time dispersion values of 5 ns and 10 ns giving the best agreement. A problem remains regarding the reflected hits peak, where the data continues to have a much higher reflected peak than the Monte Carlo. This points to the amount of charge in the detector for the reflected peak being too large for the Monte Carlo, and further studies will be needed to be done to tune this.


\subsection{UKLI work towards Hyper-Kamiokande}

Since the first implementation of the UKLI MC, there have been several improvements made to the diffuser profiles, with a full $\2pi$ diffuser profile produced from UKLI diffusers placed in a test stand in Sheffield. There were two types of diffusers involved in creating the 2D profiles: the "D5'' diffuser (and SK install spare diffuser) and "HP1'', a prototype for Hyper-Kamiokande. Profile measurements for the diffusers were made in both air and water, and Figure \ref{fig:diffuser_tank_sheff} shows the test stand setup at Sheffield, along with Figure \ref{fig:D5_diffuser} and Figure \ref{fig:HP1_diffuser} which show the D5 and HP1 diffusers respectively.

\begin{figure}[!htbp]
    \centering
    
    \caption{The Sheffield test tank setup along with photos of the D5 and HP1 diffusers}
    
    \subfloat[]{\includegraphics[width=0.33\textwidth]{Figures/D5_diffuser.PNG}}\hfill \label{fig:D5_diffuser}
    \subfloat[]{\includegraphics[width=0.33\textwidth]{Figures/HP1_diffuser.PNG}}\hfill \label{fig:HP1_diffuser}
 
    \hspace*{0.005\textwidth}%
    \subfloat[]{\includegraphics[width=0.33\textwidth]{Figures/diffuser_tank_sheff.PNG}} \label{fig:diffuser_tank_sheff}
    
\end{figure}

In Figure \ref{fig:diffuser_tank_sheff}, not only is there a $\theta$ rotational stand about the vertical axis, there is also $\phi$ rotation about a horzontal axis, allowing for a 2D profile instead of the 1D profiles produced by the Warwick test stand previously. Figure \ref{fig:D5_sheff_2D_air} and Figure \ref{fig:D5_sheff_2D_water} show the 2D profiles for the D5 diffuser produced in air and water, with the $\theta$ and $\phi$ rotations labelled. 

\begin{figure}[!htbp]
    \centering
    
    \caption{Monte Carlo simulations of the B1 -B5 diffuser injectors}
    
    \subfloat[]{\includegraphics[width=0.45\textwidth]{Figures/D5_sheff_2D_air.PNG}}\hfill
    \subfloat[]{\includegraphics[width=0.45\textwidth]{Figures/D5_sheff_2D_water.PNG}}\hfill

\end{figure}

To investigate the relationship between polar angle, azimuthal angle and the amplitude of the light, plots were made for the D5 and HP1 diffusers of the variation of amplitude vs polar angle, for measurements of azimuthal angle in 30\degree intervals. This would give an indication of how uniform the profile is by seeing how closely the values of amplitude for each azimuthal angle are to each other. This is shown in Figure \ref{fig:HP1_D5_azimuthal}, along with the coefficient of quartile variation (CQV) of the amplitude, to show the spread of the data and used instead of the standard deviation as a more robust measure as it will not be affected by outliers.  This data is shown for both the D5 and HP1 diffusers in air and also water, with three distinct regions (shown by the dashed lines) across which the amplitude varies most. 

\begin{figure}[!htbp]
    \centering
    
    \caption{Amplitude and Amplitude CQV plotted against polar angle (in degrees) for different values of azimuthal angle in air and water for the D5 and HP1 diffusers}\label{fig:HP1_D5_azimuthal}
    
    \subfloat[]{\includegraphics[width=0.33\textwidth]{Figures/cqv_D5_air.PNG}}\hfill
    \subfloat[]{\includegraphics[width=0.33\textwidth]{Figures/cqv_D5_water.PNG}}\hfill
    \subfloat[]{\includegraphics[width=0.33\textwidth]{Figures/CQV_HP1_air.PNG}}
    \subfloat[]{\includegraphics[width=0.33\textwidth]{Figures/CQV_hp1_water}}%
    
\end{figure}

The reason for the largest CQV values for the larger values of polar angle is due to smaller amplitude values showing more variation than for larger amplitude values if there an inherent systematic variation stemming from measurement apparatus or dark noise in the tank. 
    %\chapter{Super-Kamiokande Gadolinium Upgrade}\label{chp:superkgdupgrade}


\section{Physics motivation behind Super-Kamiokande Gadolinium Upgrade}

In order to be able to possibly observe the supernova relic neutrino (SRN) flux, also known as diffuse supernova neutrino background (DSNB) flux it was proposed that Gadolinium (Gd) should be added to the the water in Super-Kamiokande. Natural isotopes of gadolinium have large thermal neutron capture cross sections. As a result of this, when neutrons are captured on them there is a cascade of gamma rays that occurs, with an energy totalling ~8 MeV, whereas neutron capture that occurs on hydrogen produces a single 2.2 MeV gamma ray. Two such natural isotopes, Gd-155 and Gd-157 have thermal neutron capture cross sections of 60740 barns and 253700 barns respectively, while the thermal neutron capture cross section of hydrogen is just 0.329 barns. The Super-Kamiokande with Gadolinium experiment, formerly known as GADZOOKS! (Gadolinium Antineutrino Detector Zealously Outperforming Old Kamiokande, Super!) was proposed in 2003, which stated the intention of enriching Water Cherenkov detectors with water soluble gadolinium salt. The ultimate aim is to load a total amount of gadolinium in the form of gadolinium sulphate octahydrate($$
\mathrm{Gd}_{2}\left(\mathrm{SO}_{4}\right)_{3} \cdot 8 \mathrm{H}_{2} \mathrm{O}$$) in Super-Kamiokande which equates to 0.2\% of Gd by mass, which would allow for 90\% neutron capture efficiency. The ability to tag neutrons efficiently in Super-Kamiokande will benefit multiple physics topics, not only for the aforementioned observation of SRN flux, but also for analyses involving atmospheric neutrinos and proton decay. 
\newline

A massive amount of energy is relased when a core-collapse supernova (SN) occurs, about $10^{46}$ J. The vast majority of this energy (99\%) is released in the form of neutrinos, and due to neutrinos interacting with matter only very weakly, these traverse space with barely any attenuation. The interaction through which neutrino detectors such as Super-Kamiokande detect SRN flux is through inverse beta decay (IBD), shown in Equation \ref{eq:IBD_equation}. 

\begin{equation}
    \bar{\nu}_{e}+p \rightarrow n+e^{+}
\label{eq:IBD_equation}
\end{equation}

Due to the large cross section of the interaction, these events constitute about 88\% of the total number of events in the detector \cite{martiEvaluationGadoliniumAction2020}. With efficient neutron tagging in Super-Kamiokande, the backgrounds (charged current interactions and muon spallation) in the search for SRN flux neutrinos would be largely reduced. The neutral current quasielastic (NCQE) background would still remain due to the way the gamma rays arising from neutron capture mimic the signal of the inverse beta decay (IBD) reactions: a schematic of both NCQE and IBD reactions are shown in Figure \ref{fig:NCQE_IBD}. The measurement of the NCQE interactions using T2K beam flux can aid in understanding this background more due to the T2K flux peak being near the atmospheric neutrino flux peak (~600 MeV). 


\begin{figure}[H]
    \includegraphics[width=\textwidth]{Figures/schematic.png}
\caption{Schematic showing the IBD and NCQE interaction mechanisms}
\label{fig:NCQE_IBD}
\end{figure}


Efficient neutron tagging aids information about neutrino energy and neutrino interaction type, and when it comes to studying atmospheric neutrino oscillations, accurate neutrino energy reconstruction is particularly important. Figure \ref{fig:atm_nu_energy} shows the fraction of non-visible energy as a function of the number of tagged neutrons from simulations of atmospheric neutrino interactions at Super-Kamiokande. Here $E_{\nu}$ is the energy of the atmospheric neutrino and $E_{vis}$ is the energy that is reconstructed from charged particles. Due to these neutrinos interacting with nuclei in the detector, more neutrons are produced, and with efficient neutron tagging on gadolinium the neutrino energy can be properly reconstructed. 

\begin{figure}[H]
    \includegraphics[width=\textwidth]{Figures/atm_recon_energy.png}
\caption{MC study of (a) neutron multiplicity production for $\nu$ and ${\bar{\nu}}$, (b) neutral current, charged current and deep and non-deep inelastic scattering, (c) the correction to the visible energy as a function of neutrino multiplicity. Taken from \cite{martiEvaluationGadoliniumAction2020}.}
\label{fig:atm_nu_energy}
\end{figure}


Proton decay searches benefit from the addition of gadolinium to Super-Kamiokande because the main background to proton decay analyses come from atmospheric neutrino interactions, due to Figure \ref{fig:atm_nu_energy} showing that atmospheric neutrinos cause at least one neutron to be produced. 

\section{The EGADS project}

In 2009, prior to the additon of gadolinium in Super-Kamiokande, the EGADS (Evaluating Gadolinium's Action on Detector Systems) project was used to evaluate how the inclusion of gadolinium sulphate octahydrate would affect water quality and detector components inside Super-Kamiokande and their analyses. It was also used to observe how to reduce the visible neutron background from spallation and neutrons from fission chains from the uranium and thorium impurities in the gadolonium sulphate. EGADS is a cylindrical 200 ton stainless steel tank in a cavern near Super-Kamiokande and has its own water purification system and gadolinium sulphate octahydrate dissolving system. 

Observing the impact the addition of gadolinium sulphate octahydrate had on the water quality and components was especially crucial, and after loading 0.2\% gadolinium sulphate into the the EGADS tank in 2013, 240 PMTs were installed into the detector. 224 of these are the 50 cm Super-Kamiokande inner detector PMTs, with 60 of these having the same FRP and acrylic covers as the inner detector PMTs. Much like Super-Kamiokande, the active photocoverage for EGADS is ~40\%, with the inside walls of the detector also being covered in black polyethylene terephthalate sheets. However, unlike Super-Kamiokande, there is no outer detector in EGADS. Figure \ref{fig:EGADS_PMT} shows the PMT map for the EGADS detector along with the PMT types. Along with the PMTs which are identical to the ones inside Super-Kamiokande, EGADS also contains PMTs which are used for research and developement for use inside Hyper-Kamiokande \cite{martiEvaluationGadoliniumAction2020a}.

\begin{figure}[H]
    \includegraphics[width=\textwidth]{Figures/egads_pmt_map.png}
\caption{Map of unrolled cylindrical EGADS detector with PMT types. Taken from \cite{martiEvaluationGadoliniumAction2020a}.}
\label{fig:EGADS_PMT}
\end{figure}

Measurements regarding neutron tagging were also taken using an Americium/Beryllium (Am/Be) source placed inside EGADS. Using an Am/Be neutron source to observe gadolinium's efficacy with respect to tagging neutrons is feasible because the Am/Be source decays as in Equation \ref{eq:ambe_decay}. It produces a prompt 4.4 MeV gamma ray alongside a neutron during it's decay process, and as a result the prompt 4.4 MeV signal can serve as a trigger signal, while the following hundreds of microseconds can be used as a timing window within which to scan for the neutron. Due to its similarity to the neutral current quasieleastic events studied in the analysis in this thesis, it can serve as a helpful control sample and is used in the calculation of the detector response uncertainty in Chapter 7.

\begin{equation}
\begin{array}{c}
\alpha^{9} \mathrm{Be} \longrightarrow^{12} \mathrm{C}^{*} \mathrm{n} \\
{ }^{12} \mathrm{C}^{*} \longrightarrow{ }^{12} \mathrm{C} \gamma(4.4 \mathrm{MeV})
\end{array}
\label{eq:ambe_decay}
\end{equation}

The delayed neutron capture time from an Am/Be source used in EGADS with the gadolinium sulphate concentration of 0.2\% is shown in Figure \ref{fig:EGADS_ambe_capture}. Here, we can see that the neutron capture time from the data is 29$\pm$0.3 $\mu$s and for the Monte Carlo it is 30$\pm$0.8 $\mu$s.

\begin{figure}[H]
    \includegraphics[width=\textwidth]{Figures/egads_ambe.png}
\caption{a) Delayed neutron capture time from prompt event with Am/Be source. (b) Reconstructed energy from gamma rays after neutron capture.}
\label{fig:EGADS_PMT}
\end{figure}

  \chapter{Measurement of Neutral Current Quasielastic
Interactions with Super-Kamiokande Gadolinium Upgrade}
\label{chp:ncqegd}


\section{Bonsai output reconstruction quantities}

Due to this analysis looking specifically at the low energy region, a fitter specific to low energies (called LOWFIT) is used to reconstruct events. 

%%%%%%%%%%REPHRASE BELOW TO LEAD INTO BONSAI OUTPUT QUANTITIES%%%%%%%%%%%%%%%%%%%%%%%%%%%%%%%%%%%%%%%%%%%%%%%%%%%%%%%%%%%%%%%%%%%%%%%
Both MC and data neutrino events undergo a reconstruction phase, where the low-energy fitter BONSAI is applied to the event. This reconstruction is carried out using timing and cable information, however charge information is omitted. The ouput of BONSAI gives information which will be used in the reduction phase of the data and allow for the selection of the NCQE sample. BONSAI is a low-energy fitter used in the few-MeV to tens of MeV energy range, therefore suitable for this analysis. Its directional reconstruction uses the fitted vertex and maximises for direction while it's energy reconstruction of the event is based upon the number of hit PMTs. Equation \ref{bonsai} shows how the vertex reconstruction is formulated, where $w_{i}$ is the Gaussian hit weight of the PMT, $t_{i}$ is the hit time of the PMT and $x_{i}$ is the position of the hit PMT.
\newline
\begin{equation}
\label{bonsai}
g(\vec{v})=\sum_{i=1}^{N} w_{i} e^{.-0.5(t_{i}-|\overrightarrow{x_{i}}-\vec{v}| / c) / \sigma)^{2}}
\end{equation}
The following quantities comprise the BONSAI output, the first two being helpful spectator variables and the latter five constituting parameters which are used in the reduction phase of the analysis, from which the neutrino NCQE sample is determined.\\

\underline{Neutrino vertex direction}\\

This vector points towards the direction which is an average over all the Cherenkov cone axes which are produced, due to there being multiple leptons induced in the interaction.\\



\underline{Neutrino vertex position}\\
The reconstructed location of the neutrino interaction event.


 \underline{Reconstructed energy}\\
 In line with the standard SK low energy analysis definition, this energy is simply the reconstructed energy with the 0.511 MeV electron mass omitted. The range for Erec in this variable is 3.49 MeV to 29.49 MeV - the estimated kinetic energy under the hypothesis that the event is a singular electron.
\newline

 \underline{Dwall}\\
 This variable gives the minimum distance of the neutrino vertex position from the closest wall of the Super-Kamiokande detector.
\newline

 \underline{Effwall}\\
 Thus variable gives the distance between the neutrino vertex posiiton and the closest wall, but moving back from the vertex position along the neutrino vertex direction vector.
\newline

 \underline{Vertex direction and goodness coefficient}\\
 The coefficient $ovaQ$ (defined in Equation \ref{ovaq}) describes the quality of the vertex reconstruction. It consists of two parameters $g^2_{vtx}$ and $g^2_{dir}$ where the former describes the goodness of the vertex which is based on PMT hit timings, and increases the sharper an event is in time. The latter is the directional goodness and measures the azimuthal uniformity in the ring pattern produced by the Cherenkov cone, which decreases the more uniform an event is in space. As a result of this, $ovaQ$ increases the more uniform and sharp in time an event is.

\begin{equation}
    \text { ova } Q=g_{\text {vtx }}^{2}-g_{\text {dir }}^{2}
    \label{ovaq}
\end{equation}

$g_{vtx}$ is calculated using a fit of the PMT timing distribution and using the hit times of the PMT it is defined asin Equation\ref{vertex_goodness}.

\begin{equation}
g_{\mathrm{vtx}}=\frac{\sum_{i} w_{i} \mathrm{e}^{-\frac{1}{2}(\frac{\Delta t_{i}}{\sigma})^{2}}}{\sum_{i} w_{i}} \text { with } w_{i}=-\frac{1}{2}(\frac{\Delta t_{i}}{\omega})^{2}
\label{vertex_goodness}
\end{equation}

Here $\sum_{i} w_{i}$ is the weight given to the i-th hit PMT for the reduction of dark noise, where $\omega$ has a value of 60ns. $\sigma$ has a value of 5ns which is used to test the goodness, and as a result, a sharp timing distribution produces a large vertex goodness. $g_{dir}$ is calculated by looking at how spatially uniform the hit PMTs are around the reconstructed neutrino vertex direction. In order to quantify this uniformity, the Kolmogorov-Smirnov (KS) test is used as in Equation \ref{direction_goodness}.

\begin{equation}
    \mathrm{g}_{\mathrm{dir}}=\frac{\max _{i}\{\angle_{\mathrm{uni}}(i)-\angle_{\mathrm{data}}(i)\}-\min _{i}\{\angle_{\mathrm{uni}}(i)-\angle_{\mathrm{data}}(i)\}}{2 \pi}
\label{direction_goodness}
\end{equation}

where $\angle_{\mathrm{data}}(i)$ is the azimuthal angle of i-th hit real PMT included in the number of hits in 50ns. $\angle_{\mathrm{uni}}(i)=2 \pi i / N_{50}$ is the azimuthal angle of the i-th virtual PMT hit, but only when uniform distribution of the hits is assumed. As the uniformity of the hit pattern increases, the goodness decreases. 
 



 \underline{Cherenkov angle $theta_{C}$}

 For relativistic electrons in water, the value of the Cherenkov opening angle is $\approx 41\degree$, due to the relation: 

 \begin{equation}
\cos \theta_{\mathrm{Cherenkov}}=\frac{1}{n\beta}
\label{cherenkov_angle}
\end{equation}
 
where $\beta = v/c \approx 1$ and $n$ is the refractive index of water, 1.33. However due to other particles in the simulation, such as protons or muons, the Cherenkov cone is expected to be narrower, or if multiple leptons are present, the Cherenkov cones will be less distinct and more spread out, leading to deviations from the 41\degree value. 


\subsection{True neutron tagging information}
\subsection{Primary selection criteria}
\subsection{Secondary selection criteria}
When the neutron vertex is found by this method, 14 variables which describe different aspects of the neutron candidate are calculated. For each of the neutron candidates the vector of these variables are computed and fed into the neural network and this produces an output value which is between 0 and 1. These variables relate to different features regarding categorising hits from neutron capture on Gd or H, including the number of the hits from neutron capture, the isotropy of these hits, the Cherenkov angles of these hits and the position of the neutron vertex in the detector when capture occurs. A description of these variables are given as follows:


\underline{N10nvx}
\newline
This is the number of hits in the 10ns sliding window of the neutron candidate
\newline


\underline{N300S}
\newline
Excluding the number of hits in the 10ns sliding window (N10nvx), this is the number of hits in the extended window of 300ns.
\newline


\underline{NcS}
\newline
This variable is defined as:
\begin{equation}
\label{ncs}
    NcS = N10nvx - Nclushit
\end{equation}

    Where $Nclushit$ is the number of clusterised hits: if hit \textit{i} and \textit{j} are hits on PMTs, then for hit \textit{i} and hit \textit{j} the hit vector $\hat{r}_i$ can be written as:

\begin{equation}
\label{hit}
    \hat{r}_{i}=\frac{\overrightarrow{P M T_{i}}-\overrightarrow{V T X_{n}}}{\left\|\overrightarrow{P M T_{i}}-\overrightarrow{V T X_{n}}\right\|}
\end{equation}

where the angle at the point of the neutron capture vertex between $\hat{r}_{i}$ and $\hat{r}_{j}$ of the PMT hits is defined as:

\begin{equation}
\theta_{i j}=\arccos \left(\hat{p}_{i} \cdot \hat{p}_{j}\right)
\end{equation}

where the hits are defined as clustered if $\theta_{ij}$ is less than 14


\underline{llrca}
\newline
This variable is the log likelihood ratio calculated using triplets of hits from N10nvx that make up a rudimentary Cherenkov cone, from which the opening angle $\theta$ is calculated. Two PDFs ($\theta_{Ci}$) and ($\theta_{Ci}$) are calculated from each $\theta_{Ci}$ where p\_s and p\_b are the probability density functions of $\theta_{C}$ depending on whether the hits come from a true neutron capture on Gd or H or a false neutron capture which makes up the background. The log likelihood ratio variable is computed using Equation {\ref{llrca}}.

\begin{equation}
\label{llrca}
\newline
  llrca =\sum_{i \in\{ { triplets }\}} \log \left(\frac{f_{B}\left(\theta_{C i}\right)}{f_{S}\left(\theta_{C i}\right)}\right)
\end{equation}


\underline{beta-n}
\newline
These variables (where n = 1,2,3,4,5) are defined using Legendre polynomials, shown in Equation \ref{beta}, which gives the isotropy of the Cherenkov hits.

\begin{equation}
\label{beta}
 beta- n=\frac{2}{N 10 {nvx}(N 10 {nvx}-1)} \sum_{i \neq j}  { Legendre }_{n}\left(\cos \theta_{i j}\right)
\end{equation}

where $Legendre_n$ gives the Legendre polynomial of order $n$ and $theta_{ij}$ is the angle between hit PMTs relative to the neutron capture vertex.



\underline{ndwall}
\newline
This parameter, similar to dwall, gives the shortest distance of the neutron capture vertex from the wall of the Super-Kamiokande tank.
\newline



\underline{ntowall}
\newline
This variable (similar to effwall), gives the distance of the neutron capture vertex from the wall, however, unlike ndwall it gives the direction of the neutron capture specifically along the direction of the centre of the hits. The direction ($\overrightarrow{R}$) is given by:

\begin{equation}
\overrightarrow{\operatorname{dir}}=\sum_{i=1}^{N 10 n v x} \hat{p}_{i}
\end{equation}






  \chapter{Systematic uncertainty calculations}
\epigraph{``I made an error in my computations.''\newline
``Oh? This could be an historic occasion.''}{Mr. Spock and Dr Leonard H. ``Bones'' McCoy, Star Trek: TOS S1E19 Tomorrow Is Yesterday (1967)}
\label{chp:syst}

\section{Introduction}

Determining the systematic uncertainties on the neutron tagging efficiency is important, because even though the value has been validated against Americium-Beryllium calibration data there are still sources of systematic uncertainty that need to be taken into account. Due to the neutron tagging algorithm using the reconstructed neutrino interaction vertex in order to find neutron candidates, it means that there is an association between the neutrino interaction vertex and the neutron capture vertex which is introduced into the neutron tagging efficiency. This is utilised when calculating the systematic uncertainties stemming from secondary interactions (SI), shown later in this Chapter. Therefore unertainties involving the neutrino interactions, such as neutrino interaction cross section uncertainties and the uncertainty stemming from neutrino flux are calculated. Uncertainties in how particle interactions occur in the simulation are also calculated such as pion final state and secondary interaction uncertainties, nucleon final state interaction and secondary interaction uncertainties and uncertainties on how pions and muons are captured upon $\ch{^{16}O}$. There are also uncertainties stemming from the detector itself which need to be taken into account, such as the uncertainty in PMT gain, detector response from neutrino interaction uncertainties and uncertainties from gamma production from neutron capture detector response. 

\section{Systematic uncertainty calculation methodology}

Systematic uncertainties are calculated in one of two ways: MC re-weighting or MC regeneration. In the MC-reweighting approach, weights are applied to a quantity and the tagging efficiency of the re-weighted MC is extracted. The general methodology is to have the input of a model, defined by a set of parameters, and to vary them one by one and then calculate the reweighted tagging efficiencies. The set of relative discrepancies $\delta_{i}$ are computed from this set of reweighted tagging efficiencies $T_{i}$ and the nominal tagging efficiency $T_{nom}$ using Equation \ref{eq:tageffdiscrep}.
\newline
\begin{equation}
    \delta_{i}=\frac{T_{i}-T_{\text {nom }}}{T_{\text {nom }}} \quad i \in\{\text { parameters }\}
\label{eq:tageffdiscrep}
\end{equation}
\newline

These relative discrepancies $\delta_{i}$ are used to calculate the one indivdual discrepancy $\delta_{reweighted}$ that would describe the final deviation from the nominal tagging efficiency $T_{nom}$ due to the systematic error. The parameter $\delta_{reweighted}$ describes the model which has been produced through 1$\sigma$ variations of these parameters, therefore the final probability distribution function which describes the deviation from the nominal MC has an assumed Gaussian distribution with the standard deviation being equal to $\delta_{reweighted}$. 
\newline
The other method of estimating a systematic error on the tagging efficiency is Monte Carlo regeneration. This method is used when a single variable is changed, instead of the entire parameter space as with MC re-weighting. Monte Carlo regeneration involves changing a parameter and then re-producing the entire Monte Carlo, and then recalculating the associated tagging efficiency. The deviation of this re-generated Monte Carlo tagging efficiency from the nominal Monte Carlo tagging efficiency is then calculated using Equation \ref{eq:tageffdiscrep}.


The list of systematic errors that effect the tagging efficiency are shown in Table \ref{table:systuncertaintytable}. 

\subsection{Gamma ray detector response}

As mentioned previously, the Americium/Beryllium (Am/Be) source produces a prompt 4.4 MeV gamma ray along with a neutron via Equation \ref{eq:ambe_decay_2}. 

\begin{equation}
    \begin{gathered}
    \alpha{ }^9 \mathrm{Be} \longrightarrow{ }^{12} \mathrm{C}^* \mathrm{n} \\
    { }^{12} \mathrm{C}^* \longrightarrow{ }^{12} \mathrm{C} \gamma(4.4 \mathrm{MeV})
    \end{gathered}
\label{eq:ambe_decay_2}
\end{equation}

In Chapter 6, in order to validate the neutron tagging efficiencies calculated, the tagging efficiency of neutron captures from Am/Be + 8BGO data were also calculated and compared to the Monte Carlo prediction. These efficiencies are shown in Table \ref{table:ambe_tau_chi2} and it is important to understand the errors stemming from the Am/Be + 8BGO neutron capture efficiencies as well. The systematic uncertainties in this Chapter are evaluated by calculating the difference between the nominal tagging efficiency and the tagging efficiency arising from a different process, either a different type of simulation, a different model, or in the case of this uncertainty, a difference in the source of neutrons for neutron capture. Equation \ref{eq:tageffdiscrep} shows how this systematic is quantified; where $\delta_{i}(\pm)$ is the fractional discrepancy arising from variation in tagging efficiency, $T_{i}$ is the neutron tagging efficiency from the Am/Be + 8BGO data, and $T_{nom}$ is the nominal tagging efficiency of the NCQE MC. 

However, due to Table \ref{table:ambe_tau_chi2} showing three neutron capture uncertainties for the Am/Be data dependent on position, it makes sense to calculate the deviation of each of these from the norminal, using Equation \ref{eq:tageffdiscrep} where $i$ denotes the position of the Am/Be source. Equation \ref{eq:total_ambe_deviation} gives the total deviation of the Am/Be tagging efficiency from the nominal, i.e. the total detector response systematic.

\begin{equation}
    {\delta_{AmBe}}^2= \frac{{\delta_{AmBe}^{Centre}}^2 + {\delta_{AmBe}^{+12z}}^2 + {\delta_{AmBe}^{-12x}}^2}{3} 
\label{eq:total_ambe_deviation}
\end{equation}




\subsection{Neutrino beam flux uncertainty}

The uncertainty on neutrino beam fluxes can be evaluated by looking at the dependence of the tagging efficiency on the flux variations. The beam fluxes for the four flavour modes 
$\left(\nu_{e}, \overline{\nu_{e}}, \nu_{\mu}, \overline{\nu_{\mu}}\right)$ have the fractional uncertainties given for each mode, FHC and RHC. The binned uncertainties are shown in Figure \ref{fig:frac_beam_flux_uncertainty}, and the fractional error on the flux (y-axis) is given by the difference between the replica-target tuned flux and the nominal flux divided by the nominal flux \cite{vladisavljevic_constraining_2018}.


Each individual bin for the flux is increased/decreased by its error, the Monte Carlo re-weighting method is then used to extract the taggging efficiency for each flux bin, and Equation \ref{eq:tageffdiscrep} is used to calculate the fractional uncertainty.



Figure \ref{fig:fluxuncertainty} shows the fractional errors calculated from the reweighted Monte Carlo, with the red bars showing the -1$\sigma$ variation and the blue bars showing the +1$\sigma$ variation. Table \ref{table:systuncertaintytable} contains the value for the total uncertainty resulting from the neutrino beam flux, which was calculated using Equation \ref{eq:summingfluxuncertainty}, where the maximum value between the increased and decreased discrepancy is taken and summed over to produce the final neutrino flux beam uncertainty.
Although there are known correlations between the flux parameters, these are not taken into account, because the impact on the neutron tagging efficiencies are extremely small \cite{akutsu_thesis}.
\newline

\begin{figure}[!htb]
    \includegraphics[width=\textwidth]{Figures/frac_beam_flux_uncertainty.png}
    \caption{Fractional uncertainties of beam fluxes, taken from \cite{tn415_fiacob}, for FHC mode. }
    \label{fig:frac_beam_flux_uncertainty}
\end{figure}


\begin{equation}
    \delta_{\nu \text { flux }}=\sum_{i \in\{\text { bins }\}} \max \left[\left|\delta_{i}(+\sigma)\right|,\left|\delta_{i}(-\sigma)\right|\right]
 \label{eq:summingfluxuncertainty}   
\end{equation}

\begin{figure}[!htb]
    \centering
\includegraphics[width=\textwidth]{Figures/flux_uncertainty.png}
\caption{Tagging efficiency fractional uncertainties caused by neutrino beam flux discrepancies associated with neutrino energy in the 0.1 - 10 GeV range. From left to right the sections in this plot are comprised of the beam fluxes elements of $\left(\nu_{e} \overline{\nu_{e}} \nu_{\mu} \overline{\nu_{\mu}}\right)$ respectively.}
\label{fig:fluxuncertainty}
\end{figure}

\subsection{Neutrino cross section uncertainty}

A group of default neutrino cross section values are used to make up the nominal Monte Carlo from which the tagging efficiency is calculated. The values of the parameters that determine the cross sections are shown in Table \ref{table:xsectable}. Each of the parameter values relate to a specific interaction type and are either a normalisation parameter or a paramater which shows a kinematic dependence. \newline

For charged current quasi-elastic interactions, the uncertainty is described by the Fermi momentum of the oxygen nucleus, ($p_{F}^{O}$), the binding energy of the oxygen nucleus,($E_{B}^{O}$) and the axial mass $M_{A}^{C C Q E}$. The axial mass for CCQE interactions relates to the axial form factor which along with vector form factors is proportional to the cross section of the interactions. For neutrino interactions with correlated nucleons (2p2h), an overall normalisation parameter takes the uncertainty of these interactions into account. For $CC$ and $NC1\pi$ interactions, the uncertainty is described by the isospin background, the axial form factor $C_{A 5}^{R E S}$ which just like for CCQE interactions relates to the axial mass $M_{A}^{R E S}$. For neutral current and charged current interactions (both elastic and inelastic) there are normalisation parameters and energy dependent parameters for the uncertainty to take into account. Finally, for charged current interactions with electron neutrinos, the braking radiation from the lepton in the final state is also considered when calculating the uncertainty and is treated using a normalisation parameter.
\newline

The Monte Carlo re-weighting method is used to reweight the nominal Monte Carlo on an event by event basis with each parameter value being increased and decreased by its uncertainty, where the weights applied are produced by T2KReWeight v1r23, and for each reweighted Monte Carlo the equivalent tagging efficiency value is extracted. Equation \ref{eq:tageffdiscrep} shows how the fractional discrepancies are extracted from the nominal and reweighted tagging efficiency values. Just as with the flux parameters, there are known correlations between the cross-section parameters, but due to the impact on the neutron tagging efficency being very small these have been ignored. 

Figure \ref{fig:xsecuncertainty} shows the reweighted Monte Carlo fractional uncertainty plotted for the FHC sample. Since this sample contains a lot of NCother interactions, the uncertainty for this interaction type is greater than for the others, however compared to the rest of the systematics in this Chapter, the neutrino cross section uncertainty is very small.

\begin{table}
    \centering
        $\begin{array}{||cccc||}
        \hline & & & \\
        \text { Parameter } & \text { Interaction } & \text { Type } & \text { Value } \\
        \hline & & & \\
        p_{F}^{O} & \text{CCQE} & { }^{16} \text{O} \text { Fermi momentum } & 225 \pm 31 \text{MeV} / \text{c} \\
        E_{B}^{O} & \text{CCQE} & { }^{16} \text{O} \text { binding energy } & 27 \pm 9 \text{MeV} \\
        M_{A}^{C C Q E} & \text{CCQE} & \text { Axial mass } & 1.2 \pm 0.41 \text{GeV} / \text{c}^{2}\\
        2 p 2 h & 2 \text{p} 2 \text{~h} & \text { Normalization par. } & 1.0 \pm 1.0 \\
        C_{A 5}^{R E S}& \text{CC} \text { and } \text{NC} 1 \pi & \text { Axial form factor } & 1.01 \pm 0.12 \\
        M_{A}^{R E S}  & \text{CC} \text { and NC1 } \pi & \text { Axial mass } & 0.95 \pm 0.15 \text{GeV} / \text{c}^{2} \\
        B G_{A}^{R E S} & \text{CC} \text { and } \text{NC} 1 \pi & \text{I}=1 / 2 \text { continuum background } & 1.3 \pm 0.2 \\
        \text{CC} \text { other } & \text{CC} \text { other } & \text { E-dependent par. } & 0.0 \pm 0.4 \\
        \text{CC} \text { elastic } & \text{CC} \text { elastic } & \text { Normalization par. } & 1.0 \pm 0.3 \\
        \text{NC} \text { other } & \text{NC} \text { other } & \text { E-dependent par. } & 1.0 \pm 0.3 \\
        \text{NC} \text { elastic } & \text{NC} \text {  elastic } & \text { Normalization par. } & 1.0 \pm 0.3 \\
        \text{FS} e^{-} \text {Bremsstrahlung } & \text{CC} \nu_{e} & \text { Normalization par. } & 1.00 \pm 0.03 \\
        \hline
        \end{array}$
        \caption{Neutrino cross section parameters. Taken from \cite{nu_xsec}.}
        \label{table:xsectable}
\end{table}


\begin{figure}[!htb]
\centering
\includegraphics[width=0.8\textwidth]{Figures/xsec_uncertainty.png}
\caption{Tagging efficiency uncertainty caused by the cross-section parameters variations for the FHC mode}
\label{fig:xsecuncertainty}
\end{figure}

\subsection{Pion final state interaction (FSI) and secondary interaction (SI) uncertainties}

Even though this is an analysis concerned with neutral current quasi elastic interactions, pion events are a significant contribution to the background, and as a result it is important to examine the pion interaction uncertainties both for final state interactions and secondary interactions as their trajectories span the detector. 
\newline


The neutrino-nucleus interaction simulator used in this analysis (NEUT) handles pion final state interactions and secondary interactions using a cascade model. This cascade model contains parameters which will have uncertainties on them and these will be tranferred to a possible change in the tagging efficiency.

Depending on the momentum of the pions, different interaction types occur in the model. For pions with a momentum less than 500 MeV, the interactions in place are absorption (ABS), quasi-elastic scattering (QE) and charge exchange (CX).

Absorption occurs when the incident pion is absorbed by the nucleus and no pions remain in the final state. Quasi-elastic (QE) scattering occurs when there is only one pion observed in the final state and it has the same charge as the incident beam. Charge exchange occurs when the charged pion interacts wtht he nucleus and a single $\pi_{0}$ can be seen in the final state.


For pions with a momentum of greater than 500 MeV, a different set of interactions are used. Inelastic interactions (INEL) can now produce hadrons and replace absorption processes, but quasi-elastic scattering (QEH) and charge exchange (CXH) will still occur. The final state interaction parameters and the pion momentum range they are used in can be seen in Table \ref{table:fsiparameters}. Each parameter scales the relevant very small proabability of the charged pion interaction at every stage of the intra-nuclear cascade, aside from the parameter for charge exchange which scales only the fraction of low momentum QE scattering. 


\begin{table}
\centering
\begin{tabular}{||ccc||}
\hline
$\text { Parameter }$ & $\text { Description }$ & $\text{Momentum Region (MeV/c)}$\\
\hline
$f_{\text{ABS}}$ & $\text{Absorption}$ & $<$ 500 \\
$f_{\text{QE}}$ & $\text { Quasi-elastic scatter }$ & $<$ 500 \\
$f_{\text{CX}}$ & $\text { Single charge exchange }$ & $<$ 500 \\
$f_{\text{QEH}}$ & $\text { Quasi-elastic scatter }$ & $>$ 500 \\
$f_{\text{CXH}}$ & $\text { Single charge exchange }$ & $>$ 500 \\
$f_{\text{INEL}}$ & $\text { Hadron }(\text{N}+\text{n}\pi)$ $\text {production}$ & $>$ 500 \\
\hline
\end{tabular}
\caption{Table showing the pion final state interaction parameters in NEUT and the pion momentum range they are used in, taken from \cite{tn_32}.}
\label{table:fsiparameters}
\end{table}



A set of parameter variations which determine a surface in paramater space have been estimated by pion scattering experiments, the values for which are shown in Table \ref{table:fsimodelparameters}. The 1$\sigma$ surface has been explored using the nominal Monte Carlo re-weighting method and the analagous tagging efficiency uncertainty is shown in Equation \ref{eq:tageffdiscrep}, and the uncertainty stemming from the models shown in Table \ref{table:fsimodelparameters} is shown in Figure \ref{fig:fsisiuncertainty}


The pion FSI/SI uncertainty is then calculated using Equation \ref{eq:std_dev_fsisi}, which calculated the standard deviation of the fractional uncertainties for each model.

\begin{equation}
\delta_{\pi f s i / s i}=\sqrt{\frac{\sum_i\left(\delta_i-\bar{\delta}\right)^2}{15}} \quad \bar{\delta}=\frac{1}{16} \sum_i \delta_i
\label{eq:std_dev_fsisi}
\end{equation}


\begin{figure}[!htb]
\centering 
    \includegraphics[width=0.8\textwidth]{Figures/fsisi_uncertainty.png}
\caption{Tagging efficiency fractional uncertainty caused by the variation in the FSI/SI model parameters for the FHC mode.}
\label{fig:fsisiuncertainty}
\end{figure}

\begin{table}
\centering
\begin{tabular}{||ccccccc||}
\hline
$\text { Set }$ & $\text { ABS }$ & $\text { QE }$ & $\text { CX }$ & $\text { INEL }$ & $\text { QEH }$ & $\text { CXH }$ \\
\hline
    $\text { Nominal }$ & 1.1 & 1.0 & 1.0 & 1.0 & 1.8 & 1.8 \\
    & & & & & & \\
    & 0.7 & 0.6 & 0.5 & 1.5 & 1.1 & 2.3 \\
    & 0.7 & 0.6 & 1.6 & 1.5 & 1.1 & 2.3 \\
    $\text { Hadron production Up }$ & 1.6 & 0.7 & 0.4 & 1.5 & 1.1 & 2.3 \\
    & 1.6 & 0.7 & 1.6 & 1.5 & 1.1 & 2.3 \\
    & 0.6 & 1.4 & 0.6 & 1.5 & 1.1 & 2.3 \\
    & 0.7 & 1.3 & 1.6 & 1.5 & 1.1 & 2.3 \\
    & 1.5 & 1.5 & 0.4 & 1.5 & 1.1 & 2.3 \\
    & 1.6 & 1.6 & 1.6 & 1.5 & 1.1 & 2.3 \\
    & & & & & & \\
    & 0.7 & 0.6 & 0.5 & 0.5 & 2.3 & 1.3 \\
    & 0.7 & 0.6 & 1.6 & 0.5 & 2.3 & 1.3 \\
    & 1.6 & 0.7 & 0.4 & 0.5 & 2.3 & 1.3 \\
    $\text { Hadron production Down }$ & 1.6 & 0.7 & 1.6 & 0.5 & 2.3 & 1.3 \\
    & 0.6 & 1.4 & 0.6 & 0.5 & 2.3 & 1.3 \\
    & 0.7 & 1.3 & 1.6 & 0.5 & 2.3 & 1.3 \\
    & 1.5 & 1.5 & 0.4 & 0.5 & 2.3 & 1.3 \\
    & 1.6 & 1.6 & 1.6 & 0.5 & 2.3 & 1.3\\
\hline
\end{tabular}
\caption{Pion FSI/SI model parameter nominal value and variations grouped according to inelastic hadron production value, taken from \cite{tn_32}.}
\label{table:fsimodelparameters}
\end{table}

\subsection{Nucleon final state interactions}

Uncertainties regarding the nucleon final state interactions can change the number of nucleons knocked out of $\ch{^{16}O}$, therefore how the tagging efficiency is changed due to the variation in nucleon final state interactions needs to be investigated. This uncertainty is extracted using GENIE, a Monte Carlo event generator which contains the INTRANUKE (hA) intranuclear transport model \cite{andreopoulos2010genie}. The uncertainties in the in the total scattering probability for hadrons inside the target nuclei ($x_{m f p}^{N}$) and the uncertainties in the likelihood of each hadron rescattering method: elastic ($x_{e l}^{N}$), inelastic ($x_{i n e l}^{N}$), charge exchange ($x_{c e x}^{N}$), pion production ($x_{\pi}^{N}$) and absorption ($x_{a b s}^{N}$) are taken into account. The fractional uncertainties for these modes for pions is shown in Table \ref{table:nucleonfsiuncertainties}. 

\begin{table}
\begin{tabular}{||ccc||}
\hline
$\text {Abbreviation}$ & $\text { Description of uncertainty }$  & $\text{Fractional uncertainty}$ \\
\hline
$ x_{m f p}^{N}$ & $\text { Nucleon mean free path (total rescattering probability) }$ & $\pm 20 \%$ \\
$x_{c e x}^{N}$ & $\text { Nucleon charge exchange probability }$ & $\pm 50 \%$ \\
$x_{e l}^{N}$ & $\text { Nucleon elastic reaction probability }$ & $\pm 30 \%$ \\
$x_{i n e l}^{N}$ & $\text { Nucleon inelastic reaction probability }$ & $\pm 40 \%$ \\
$x_{a b s}^{N}$ & $\text { Nucleon absorption probability }$ & $\pm 20 \%$ \\
$x_{\pi}^{N}$ & $\text { Nucleon pion-production probability }$ & $\pm 20 \%$\\
\hline
\end{tabular}
\caption{Nucleon final state interaction parameters of the hA model executed inside GENIE.} 
\label{table:nucleonfsiuncertainties}
\end{table}

A nominal GENIE Monte Carlo sample is generated (different from the previously used NEUT Monte Carlo) and this shifted using the re-weighting method to a varied GENIE Monte Carlo by individually increasing and decreasing the parameters in Table \ref{table:nucleonfsiuncertainties} by its error. For each shifted Monte Carlo produced, the fractional uncertainty can be written as in Equation \ref{eq:tageffdiscrep}.

The tagging efficiency fractional uncertainties are displayed in Figure \ref{fig:nucleonfsiuncertainty}, showing which parameter from Table \ref{table:nucleonfsiuncertainties} each uncertainty has arisen from.

Because the parameters in Table \ref{table:nucleonfsiuncertainties} are independent, the total uncertainty on the nucleon FSI is calculated by summing the fractional uncertainties on the parameters in quadrature, and the maximum difference betwwen each $\pm \sigma$ pair, as shown in Equation \ref{eq:totalnucleonfsi}.

\begin{equation}
    \delta_{\text{Nucleon FSI}}=\sqrt{\sum_i \max \left[\delta_i^2(+\sigma), \delta_i^2(-\sigma)\right]}
\label{eq:totalnucleonfsi}
\end{equation}

\begin{figure}[!htb]
\centering 
    \includegraphics[width=0.7\textwidth]{Figures/nucleonfsi_uncertainty.png}
\caption{Tagging efficiency fractional uncertainties caused by the nucleon final state interaction model parameter variation for the FHC mode}
\label{fig:nucleonfsiuncertainty}
\end{figure}

\subsection{Muon and pion capture on Oxygen-16}

Neutrons are produced from negative muon capture on $\ch{^{16}O}$ as show in Equation \ref{eq:muoncap}.

\begin{equation}
        \mu^{-}\;\ch{^{16}{O}}\;\longrightarrow\;\nu_{\mu}\;n\;\ch{^{15}{N}}
\label{eq:muoncap}
\end{equation}

Direct neutrons are produced from pion capture on $\ch{^{16}O}$, but also a number of evaporation neutrons that leave the nucleus. For the capture of muons and pions on $\ch{^{16}O}$, the energy spectra of the neutrons produced have been measured: for muons the spectra can range up to 15 MeV, while in the case of pions the spectra can reach up to 100 MeV.


GEANT4 simulates the capture processes for muons and pions, but there are alternate models that can be used: for example, the Chiral Invariant Phase Space (CHIPS) model for muon captures (based on non pertubative QCD) and two different routines for pion capture, one which is based on CHIPS and one based on intra-nuclear cascade.

Because any change in the model can alter the energy spectra of the neutrons, these alternative functions can be used to estimate the fractional uncertainties for the tagging efficiency. This is done by using the MC regeneration method, where the nominal Monte Carlo is regenerated by replacing the default GEANT4 routines with the alternative models. For the alterantive muon capture model and the two alternative pion capture models, the fractional discrepancies are shown in Equation \ref{eq:tageffdiscrep}.

Figure \ref{fig:mupicap_uncertainty} shows the fractional uncertainties caused by muon and pion capture on oxygen for each model.

\begin{figure}[!htb]
\centering
    \includegraphics[width=0.7\textwidth]{Figures/mupicap_uncertainty.png}
\caption{Fractional uncertainties in the tagging efficiency caused by muon and pion capture on the muon capture CHIPS model, pion capture CHIPS model and the pion capture Bertini model.}
\label{fig:mupicap_uncertainty}
\end{figure}

The final total muon and pion capture systematic error is summarised in Equation \ref{eq:total_mupicap_error}. Due to the systematic errors $\delta_{muon C H I P S}$, $\delta_{pion C H I P S}$ and $\delta_{pion B e r t}$, (which are the errors due to the the muon capture CHIPS model, pion capture CHIPS model and pion capture Bertini model respectively) being independent of one another they can be added in quadrature.

\begin{equation}
    \delta_{\text {muon and pion capture }}=\sqrt{{\delta_{muon C H I P S}}^2 + {\delta_{pion C H I P S}}^2 + {\delta_{pion B e r t}}^2 }
\label{eq:total_mupicap_error}
\end{equation}


\subsection{Nucleon SI}
Uncertainties in how Nucleon SI interactions are modelled can affect the tagging efficiency, due to how these uncertainties can affect the final number of nucleons present in the simulation and how far they travel in the detector medium. There are two Monte Carlo code systems used in order to determine how nucleons travel in the simulation, MICAP and HETC \cite{micap_hetc}. These come as part of GCALOR, used to determine the energy and direction of incident hadrons, leptons and photons \cite{1998gcalor}. MICAP (Monte-Carlo-Ionization-Chamber-Analysis-Program), which simulates interactions based on calculated cross section and angular distributions for secondary particles, and is called for neutrons with a kinetic energy below 20 MeV \cite{Zeitnitz:1994bs}. HETC on the other hand, is the High-Energy-Transport-Code, and is responsible for transporting charged hadrons above 20 MeV (up to an energy of 10 GeV) through the detector medium \cite{gabrielhetc}. 

\subsection{MICAP uncertainty calculation}

A library of cross-section data called ENDF \cite{endf} (Evaluated Nuclear Data Files) are used by MICAP to determine what processes the neutrons go through in the detector medium, and their respective cross sections. There are two versions of libraries which are used in evaluating the uncertainty in the tagging efficiency, version B release V (released in 1978) and version B release VIII, released in 2018 \cite{endf8}, \cite{endf82}. There is very little difference between the total neutron on hydrogen cross sections between the two versions of the code, however, in the energy range of 0.1 MeV to 20 MeV, there are differences of up to 40\% in the cross-sections of neutron on oxygen, as shown in Figures \ref{fig:neutron_cross_section_H} and \ref{fig:neutron_cross_section_O}. Both an elastic and inelastic part comprise the total cross-section of a nucleus, and these can effect the way neutrons travel in the simulated detector medium and also the way secondary particles from these interactions are distributed. The way these inelastic processes are simulated depend on the nuclei involved: hydrogen nuclei capture the neutrons in the range ($10^{-11}$, $10^{-7}$) MeV, while neutron captures on oxygen mainly happen on the 4 MeV to 20 MeV energy region. The ENDF/B-VIII library extends the neutron capture energy range for hydrogen to the range ($10^{-11}$, $10^{-4}$) MeV and for captures on oxygen there were differences between ENDF/B-V and ENDF/B-VIII in the 0-10 MeV energy range \cite{akutsu_thesis}.  To calculate the way uncertainties arise from the way MICAP simulates neutron captures, the ENDF-V library is replaced by ENDF-VIII, and the tagging efficiency using this library ($T_{VIII}$) is evaluated from regenerated MC. Equation \ref{eq:MICAP_uncertainty} shows the fractional uncertainty $\delta_{MICAP}$ regarding MICAP.  

\begin{equation}
    \delta_{m c p}=\frac{T_{V I I I}-T_{n o m}}{T_{n o m}}
\label{eq:MICAP_uncertainty}
\end{equation}

\begin{figure}[!htb]
    \centering
    \includegraphics[width=\textwidth]{Figures/neutron_cross_section_endf_H.PNG}
\caption{Left:Fraction of inelastic cross-section for hydrogen for both the ENDF-V and ENDF-VIII libraries. Rght: Difference of the fraction of the inelastic cross-section between the two libraries. Taken from \cite{tn415_fiacob}.}
\label{fig:neutron_cross_section_H}
\end{figure}

\begin{figure}[!htb]
    \centering
    \includegraphics[width=\textwidth]{Figures/neutron_cross_section_endf_O.PNG}
\caption{Left:Fraction of inelastic cross-section for $\ch{^{16}O}$ for both the ENDF-V and ENDF-VIII libraries. Rght: Difference of the fraction of the inelastic cross-section between the two libraries. Taken from \cite{tn415_fiacob}.}
\label{fig:neutron_cross_section_O}
\end{figure}

A difference in the inelastic component can change the neutron travel distance, so looking at how changing the Nucleon SI changes the distance between the prompt vertex and the neutron capture vertex is important: this is shown in Figure \ref{fig:ENDF8_syst_error}.

\begin{figure}[!htb]
\centering
    \includegraphics[width=0.7\textwidth]{Figures/endf8_tageff_plot.PNG}
\caption{Tagging efficiency dependence of the true distance between the neutrino and neutron capture vertices between the ENDF V library (used in the nominal MC, shown in cyan) and the ENDF/B VIII library shown in red.}
\label{fig:ENDF8_syst_error}
\end{figure}

\subsection{HETC uncertainty calculation}

Due to there being no experimental data for cross-section calculations on nucleon-oxygen scattering in the energy range at which T2K functions, experimental data of proton-carbon scattering is used to assign error on the cross sections. In the proton-carbon scattering analysis, NEUT was used to evaluate the theoretical cross sections of carbon \cite{hayato_neut}, and this uses cross sections calculated using the Bertini model \cite{bertini_hetc}, which is also used in HETC \cite{hetc_paper}. The comparison of these calculated cross sections to other theoretical calculations as well as to data, showed that the total cross sections calculated by Bertini need to be varied by $\pm$ 30\% in order to be consistent with them. As a result of this, the Monte Carlo is regenerated twice where the free nucleon-nucleon cross sections are scaled by $\pm$ 30\%. Figure \ref{fig:pp_nn_xsec} shows the proton-proton(neutron-neutron) and proton-neutron cross-sections with the shaded area around the line representing the values when the cross sections are scaled by $\pm$ 30\%. Equation \ref{eq:tageffdiscrep} gives the fractional uncertainty stemming from re-generating the MC with this $\pm$ 30\% scaling, where $\delta_{i}(\pm)$ is the error and $T_{i}$ is the tagging efficiency of the MC with this scaling.


\begin{figure}[!htb]
\centering
    \includegraphics[width=\textwidth]{Figures/pp_nn_xsec.PNG}
\caption{Left: proton-proton (neutron-neutron) cross sections, where the shaded area covers the values which are scaled by $\pm$ 30\%. Right: proton-neutron cross sections, where the shaded area covers the values which are scaled by $\pm$ 30\%. Taken from \cite{tn415_fiacob}.}
\label{fig:pp_nn_xsec}
\end{figure}



\begin{equation}
    \delta_{HETC }(\pm)=\frac{T_{HETC}(\pm)-T_{nom }}{T_{nom }}
\label{eq:HETC_uncertainty}
\end{equation}

Figure \ref{fig:HETC_taggeff_syst} shows the dependence of tagging efficiency on the distance between the prompt vertex and neutron capture vertex for the nominal MC and the scaled cross section by $\pm$ 30\%. 



\begin{figure}[!htb]
\centering
    \includegraphics[width=0.7\textwidth]{Figures/hetc_han_syst.PNG}
\caption{Tagging efficiency dependence of the true distance between the prompt vertex and the neutron capture vertex for the nominal MC (cyan) and the $\pm$ 30\% cross section cases.}
\label{fig:HETC_taggeff_syst}
\end{figure}

The fractional discrepancies regarding the HETC codes are not inherently symmetric which is why the positive and negative uncertainties stemming from the Nucleon SI are calculated individually as shown in Equation \ref{eq:nucleon_si_final_error}, and these values are presented in Table \ref{table:systuncertaintytable}.

\begin{equation}
    \begin{aligned}
    & \delta_{\text{Nucleon FSI}}^{+}=\left|\delta_{\text{MICAP}}\right|+\left|\delta_{\text {HETC}}(+\sigma)\right| \\
    & \delta_{\text{Nucleon FSI}}^{-}=\left|\delta_{\text {MICAP}}\right|+\left|\delta_{\text {HETC}}(-\sigma)\right|
    \end{aligned}
    \label{eq:nucleon_si_final_error}
\end{equation}


\subsection{Uncertainty due to PMT gain simulation}

The change in PMT gain over time in the Super-K detector provides a systematic uncertainty for the simulations in this analysis. In SKDETSIM-SKGd the PMT gain drift is modelled by scaling the charge recieved by the PMT according to Equation \ref{eq:PMT_drift_gain}, where $G_{0}$ is the amount of average PMT gain value from June 2017. 

\begin{equation}
    Q \longrightarrow Q \times\left(1+\frac{G(t)-G_0}{G_0}\right)
\label{eq:PMT_drift_gain}
\end{equation}

In addition to the gain changing over time, so does the number of PMT hits due to the gain, and this has to be adjusted by a correction factor of $\alpha$ which has a value of 1.6, shown in Equation \ref{eq:PMT_drift_gain_scaling}. This value is estimated by comparing calibration data and simulations \cite{linyan_thesis}. For $\alpha$ = 1, Equation \ref{eq:PMT_drift_gain_scaling} reduces to Equation \ref{eq:PMT_drift_gain}.


\begin{equation}
    Q \longrightarrow Q \times\left(1+\alpha\frac{G(t)-G_0}{G_0}\right)
\label{eq:PMT_drift_gain_scaling}
\end{equation}

The discrepancies in tagging efficiency for this systematic uncertainty are produced by looking at the fractional uncertainty in tagging efficiencies between $\alpha$ = 1.6 and $\alpha$ = 1, according to Equation \ref{eq:PMT_gain_drift_uncertainty}.

\begin{equation}
    \delta_i=\frac{T_i^{\alpha=1.6}-T_i^{\alpha=1}}{T_i^{\alpha=1}} \quad i \in\{\text { Regeneration points }\}
\label{eq:PMT_gain_drift_uncertainty}
\end{equation}


\subsection{Detector response for neutrino events}
 
The Bonsai fitter output produces parameters that encapsulate the detector repsonse to neutrino events: namely the parameters $E_{rec}$, dwall, effwall, ovaQ, and $\theta_C$. During the selection of NCQE events, cuts are made on these variables in order to give the most plentiful NCQE sample. However, uncertainties on these parameters can modify the amount of NCQE events and affect the position vector of the reconstructed neutrino vertex, and therefore the NTag algorithm, which could have an impact on the tagging efficiency. The errors on these variables are shown in Table \ref{table:det_resp_nu_errors}, and are also provided in TN-374 \cite{tn_374}.

\begin{table}
    $$
\begin{array}{ll}
\hline \text { Reduction parameter }  & \text { Uncertainty } \\
\hline E_{\text {rec }}  & \pm 5 \% \\
\text { ovaQ } & \pm 1.5 \% \\
\theta_C & \pm 2 \text { degree } \\
\text{Effwall} & \pm 50 \text {cm} \\
\text{DWall} & \pm 5 \text{cm}  \\
\hline
\end{array}
$$
\caption{Uncertainties on Bonsai output parameters (taken from \cite{tn_374}).}
\label{table:det_resp_nu_errors}
\end{table}

The nominal Monte Carlo is re-weighted to produce two new Monte Carlo ($\pm$) for each of the Bonsai fitter output parameters, $i$, and the respective tagging efficiencies $T_{i}(\pm\sigma$) are calculated. The fractional uncertainties are given using Equation \ref{eq:tageffdiscrep}, where $i$ represents the Bonai reduction parameter being varied.



Figure \ref{fig:nu_det_syst_error} shows the fractional uncertainties for each parameter. Due to these variables being considered independent of one another, their uncertainties can be summed in quadrature where as usual the maximum is taken between the $\delta(\pm\sigma)$ pairs, shown in Equation \ref{eq:total_nu_det_syst}.

\begin{equation}
    \delta_{\text {$\nu$ response }}=\sqrt{\sum_i \max \left[\delta_i^2(+\sigma), \delta_i^2(-\sigma)\right]}
\label{eq:total_nu_det_syst}
\end{equation}

\begin{figure}[!htb]
    \centering
    \includegraphics[width=\textwidth]{Figures/nu_det_response.png}
    \caption{Variation in tagging efficiency produced by the error in neutrino detector response parameters (Bonsai output variables). Red: -1$\sigma$ discrepancy. Blue: +1$\sigma$ discrepancy. }
    \label{fig:nu_det_syst_error}
\end{figure}

\newpage


\subsection{Final systematic error}

\begin{table}[htb!]
\centering
    $\begin{array}{lll}
    \hline \text { Systematic error} & \text{Value (\%)} \\
    \hline 
    \text { PMT gain simulation } & \pm 3.00 \\ 
    \nu \text { cross section } & \pm 0.30 \\ 
    \nu \text { beam flux } & \pm 0.60 \\ 
    \pi \text { FSI/SI } & \pm 0.10  \\ 
    \text { Nucleon FSI } & \pm 0.20  \\ 
    \text { Nucleon SI } & { }_{-5.40}^{+5.40} \\ 
    \mu \text{and} \pi \text { capture on }{ }^{16} \mathrm{O} & \pm 0.29\\ 
    \text{Det. resp. for $\nu$} & \pm 1.36\\
    \text { Det. resp. for gamma rays }& \pm 1.36 \\ 
    \end{array}$
\caption{Final systematic uncertainties for the tagging efficiency} 
\label{table:systuncertaintytable}
\end{table}


In general the systematic errors associated with the tagging efficiency are as expected: the neutrino cross section and beam flux errors provide only a small contribution to the total systematic error on the tagging efficiency because they have very little impact on the neutron tagging efficiency, and the uncertainty on neutron propagation in Super-Kamiokande. Compared to previous analyses which studied neutron tagging in water at Super-K (\cite{tn415_fiacob}, \cite{akutsu_thesis}), the pion FSI/SI error is small. This could be due to the fact that post the addition of $\mathrm{Gd}_{2}\left(\mathrm{SO}_{4}\right)_{3} \cdot 8 \mathrm{H}_{2} \mathrm{O}$ into the simulation, there are more processes occuring within SKDETSIM due to the additional presence of oxygen, gadolinium and sulphate ions. As a result, the impact that this uncertainty has on the total systematic uncertainty is reduced compared to analyses where these additional nuclei are not present. Similarly, the Nucleon SI uncertainty is also small compared to (\cite{tn415_fiacob}, \cite{akutsu_thesis}), because of the additional interactions occuring on the extra nuclei in the simulation, and therefore the nucleon FSI uncertainty that can alter the number of protons and neutrons knocked out of ${ }^{16} \mathrm{O}$ is lessened. The nucleon SI error is the biggest amongst the uncertainties listed in Table \ref{table:systuncertaintytable}, which is appropriate and in keeping with the results in \cite{tn415_fiacob}, \cite{akutsu_thesis}, as variations in the inelastic cross section for ${ }^{16} \mathrm{O}$ and the uncertainties on the cross sections calculated by the Bertini model which are scaled by $\pm$ 30\% will have a direct impact on the distance between the neutrino interaction vertex and the neutron capture vertex, and therefore a direct impact on the neutron tagging efficiency. The muon and pion capture uncertainties on ${ }^{16} \mathrm{O}$ have smaller values than \cite{akutsu_thesis}, one possible reason for this is the presence of extra gadolinium and sulphate ions in the simulation so that less muons will capture on ${ }^{16} \mathrm{O}$, and as a result the impact on the associated energy spectra of neutrons will be less, so the affect on the neutron tagging efficiency will be less. The neutrino detector response uncertainty is in keeping with the uncertainty evaluated in \cite{tn415_fiacob}, which is understandable as the same errors on the NCQE reduction parameters are used. The detector response from gamma ray uncertainty evaluated by comparisons with Am/Be + 8 BGO data is smaller compared to either of the results from \cite{tn415_fiacob} or \cite{akutsu_thesis} - this makes sense due to the greater compatibility of the NCQE MC neutron tagging efficiency value with the value obtained by comparisons to the Am/Be + BGO data. The uncertainty of the detector response for the neutrinos being the same as the detector response for gamma ray uncertainty is a coincendence ($\pm 1.36$) as these are calculated in two wholly different ways, the former by varying the reduction cut parameters and the latter by comparisons with Am/Be + 8BGO data. The second largest systematic uncertainty in Table \ref{table:systuncertaintytable} comes from the uncertainty in the PMT gain. This value is larger than for \cite{tn415_fiacob} or \cite{akutsu_thesis}, however PMT gain increases over time and as the gain increases so does the neutron tagging efficiency discrepancy due to PMT gain.
  \chapter{DSNB NCQE background calculation}
\label{chp:ncqe_xsec}


\section{Introduction}

As mentioned in Chapter 1, NCQE interactions form the main background to the DSNB signal, and therefore calculating the number of events that form this background concludes the work of this analysis. 

The number of observed neutrino events which are detected at Super-Kamiokande as a function of some observable such as reconstructed neutrino energy $N_{\nu}(E_{\nu})$ is given in Equation \ref{eq:nu_number}.

\begin{equation}
    N_\nu(E_\nu)=\mathcal{C} \int \Phi\left(E_\nu\right) \sigma\left(E_\nu\right) \epsilon d E_\nu
\label{eq:nu_number}
\end{equation}

Here, $\mathcal{C}$ is the target volume, $E_{\nu}$ is the true incoming neutrino energy, $\phi(E_{\nu})$ is the flux of the incoming neutrino, $\sigma(E_{\nu})$ is the cross section for the neutrino interaction and $\epsilon$ is the detection efficiency of the neutrino by the far detector. The number of neutrino events from Monte Carlo for each interaction type is given by the summing the number of NCQE events for neutrinos and anti-neutrinos, and charged-current and NC-other interactions, where the numbers of the neutrino and antineutrino NCQE events are multiplied by scale factors which scale the simulated NCQE cross section prediction.


\section{NCQE prediction scale factors}



 \begin{equation}
    \begin{aligned}
    & N_{\mathrm{obs}}^{\mathrm{FHC}}-M_{\mathrm{NC}-\text { other }}^{\mathrm{FHC}}-M_{\mathrm{CC}}^{\mathrm{FHC}}=f_{\nu-\mathrm{NCQE}} M_{\nu-\mathrm{NCQE}}^{\mathrm{FHC}}+f_{\bar{\nu}-\mathrm{NCQE}} M_{\bar{\nu}-\mathrm{NCQE}}^{\mathrm{FHC}} \\
    & N_{\mathrm{obs}}^{\mathrm{RHC}}-M_{\mathrm{NC}-\mathrm{other}}^{\mathrm{RHC}}-M_{\mathrm{CC}}^{\mathrm{RHC}}=f_{\nu-\mathrm{NCQE}} M_{\nu-\mathrm{NCQE}}^{\mathrm{RHC}}+f_{\bar{\nu}-\mathrm{NCQE}} M_{\bar{\nu}-\mathrm{NCQE}}^{\mathrm{RHC}}
    \end{aligned}
\label{eq:scale_factor_mode}
\end{equation}


Equation \ref{eq:scale_factor_mode} gives the scale factors for neutrino and anti-neutrino events for FHC and RHC modes and by rearranging and transforming these two equations, the scale factors for the number of neutrino and antineutrino NCQE events can be extracted instead, as shown in Equation \ref{eq:scale_factor_nu}. 

\begin{equation}
    \begin{aligned}
    f_{\nu-\mathrm{NCQE}}= & \frac{X M_{\bar{\nu} \mathrm{N} \mathrm{NCQE}}^{\mathrm{RHC}}-Y M_{\bar{\nu}-\mathrm{NCQE}}^{\mathrm{FHC}}}{M_{\nu-\mathrm{NCQE}}^{\mathrm{FHC}} M_{\bar{\nu}-\mathrm{NCQE}}^{\mathrm{RHC}}-M_{\nu-\mathrm{NCQE}}^{\mathrm{RHC}} M_{\bar{\nu}-\mathrm{NCQE}}^{\mathrm{FHC}}} \\
    f_{\bar{\nu}-\mathrm{NCQE}}= & \frac{X M_{\nu-\mathrm{NCQE}}^{\mathrm{RHC}}-Y M_{\nu-\mathrm{NCQE}}^{\mathrm{FHC}}}{M_{\bar{\nu}-\mathrm{NCQE}}^{\mathrm{FHC}} M_{\nu-\mathrm{NCQE}}^{\mathrm{RHC}}-M_{\bar{\nu}-\mathrm{NCQE}}^{\mathrm{RHC}} M_{\nu-\mathrm{NCQE}}^{\mathrm{FHC}}}
    \end{aligned}
\label{eq:scale_factor_nu}
\end{equation}

where the coefficients $X$ and $Y$ are given in Equation \ref{eq:scale_factor_coeff}.


\begin{equation}
    \begin{aligned}
    X & =N_{\mathrm{obs}}^{\mathrm{FHC}}-M_{\mathrm{NC}-o t h e r}^{\mathrm{FHC}}-M_{\mathrm{CC}}^{\mathrm{FHC}} \\
    Y & =N_{\mathrm{obs}}^{\mathrm{RHC}}-M_{\mathrm{NC}-o t h e r}^{\mathrm{RHC}}-M_{\mathrm{CC}}^{\mathrm{RHC}}
    \end{aligned}
\label{eq:scale_factor_coeff}
\end{equation}


Table \ref{table:nu_FHC_mc} shows the gives the expected number of neutrino events for each type of interaction including neutral current and charged current interactions for FHC mode (same as mentioned in Chapter 6), and Table \ref{table:nu_RHC_mc} shows the same information for RHC mode. These have been calculated with the 21bv2 flux uncertainty with a total SK POT of $21.4 x 10^{20}$ with Run 11 POT included which was taken between March 9th and April 22nd 2021. This was the T2K run with a concentration of $\mathrm{Gd}_{2}\left(\mathrm{SO}_{4}\right)_{3} \cdot 8 \mathrm{H}_{2} \mathrm{O}$ in SK, which matches the amount in the SKDETSIM Monte Carlo. 


\begin{table}
    $$
    \begin{array}{ccc}
    \hline \text { FHC sample } & \text { MC } \# \boldsymbol{\nu}_{\text {det }} & \text { MC } \# \nu_{\text {det }} \text { fraction (\%) } \\
    \hline \nu-N C Q E & 1199.7 & 75.0 \\
    \bar{\nu}-\text { NCQE } & 34.5 & 2.2 \\
    N C-\text { other } & 288.1 & 19.1 \\
    C C & 17.4 & 3.7\\
    \hline \text { Total } & 1599.2 & 100 \\
    \hline
    \end{array}
    $$
    \caption{FHC MC expectation values for each interaction type with a total SK POT of $10 \times 10^{21}$. }
    \label{table:nu_FHC_mc}
\end{table}


\begin{table}
    $$
    \begin{array}{ccc}
    \hline \text { RHC sample } & \text { MC } \# \boldsymbol{\nu}_{\text {det }} & \text { MC } \# \nu_{\text {det }} \text { fraction (\%) } \\
    \hline \nu-N C Q E & 182.8 & 31.9 \\
    \bar{\nu}-\text { NCQE } & 257.0 & 44.8 \\
    N C-\text { other } & 118.4 & 20.6 \\
    C C & 15.3 & 2.7 \\
    \hline \text { Total } & 573.5 & 100 \\
    \hline
    \end{array}
    $$
    \caption{RHC MC expectation values for each interaction type with a total SK POT of $10 \times 10^{21}$.}
    \label{table:nu_RHC_mc}
\end{table}

Although the analysis in this thesis with regards to the work done in Chapter 6 and 7 used FHC mode events, for the purpose of this cross-section extraction, the number of events for each interaction type in RHC mode have been calculated as well. Using the number of events for each interaction type in Table \ref{table:nu_FHC_mc} and Table \ref{table:nu_RHC_mc} and Equation \ref{eq:scale_factor_nu} and Equation \ref{eq:scale_factor_coeff}, along with the number of observed events (204 in FHC and 97 in RHC, taken from \cite{Abe_2019}), the scale factors are calculated and shown in Equation \ref{eq:scale_factors_value}.

\begin{equation}
    \begin{aligned}
    & f_{\nu-\mathrm{NCQE}}=0.710 \pm 0.001 \text { (stat.) } \\
    & f_{\bar{\nu}-\mathrm{NCQE}}=1.067 \pm 0.004 (\text { stat. })
    \end{aligned}
\label{eq:scale_factors_value}
\end{equation}
    


\section{Flux averaged NCQE cross-section}

NEUT's cross section prediciton for the interaction of neutrinos and antineutrinos with ${ }^{16} \mathrm{O}$ is shown in Equation \ref{eq:NEUT_xsec_prediction}. Here $\langle\sigma_{\nu \text {-NCQE }}^{\text {NEUT }}\rangle$ gives the flux-averaged cross section prediction for neutrinos and $\langle\sigma_{\bar{\nu}-\mathrm{NEUCQ}}^{\mathrm{NEUT}}\rangle$ gives the flux-averaged cross section for antineutrinos. 


\begin{equation}
    \begin{aligned}
    &\left\langle\sigma_{\nu \text {-NCQE }}^{\mathrm{NEUT}}\right\rangle= \frac{\sum_{\nu=\nu_\mu, \nu_e} \int \sigma_{\nu-\mathrm{NCQE}}^{\mathrm{NEUT}}\left(E_\nu\right) \phi_\nu\left(E_\nu\right) d E_\nu}{\sum_{\nu=\nu_\mu, \nu_e} \int \phi_\nu\left(E_\nu\right) d E_\nu}=2.13 \times 10^{-38} \mathrm{~cm}^2 / \text { oxygen } \\
    &\left\langle\sigma_{\bar{\nu} \text {-NCQE }}^{\mathrm{NEUT}}\right\rangle=\frac{\sum_{\nu=\bar{\nu}_\mu, \bar{\nu}_e} \int \sigma_{\bar{\nu} \text {-NCQE }}^{\mathrm{NEUT}}\left(E_\nu\right) \phi_\nu\left(E_\nu\right) d E_\nu}{\sum_{\nu=\bar{\nu}_\mu, \bar{\nu}_e} \int \phi_\nu\left(E_\nu\right) d E_\nu}=0.88 \times 10^{-38} \mathrm{~cm}^2 / \text { oxygen. }
    \end{aligned}
\label{eq:NEUT_xsec_prediction}
\end{equation}

The flux averaged NEUT cross section values in Equation \ref{eq:NEUT_xsec_prediction} is taken from \cite{Abe_2019}, and the integrations are taken up to an energy of 10 GeV. These NEUT cross sections are multiplied by the scale factors in order to obtain the flux-averaged NCQE cross sections, shown in Equation \ref{eq:xsec_value}. The systematic error on the final cross-section values are taken from Table \ref{table:systuncertaintytable} by summing up all the percentages and using the final percentage to calculate the systematic error on the values.


\begin{equation}
    \begin{aligned}
    & \left\langle\sigma_{\nu \text {-NCQE }}\right\rangle=f_{\nu \text {-NCQE }} \cdot\left\langle\sigma_{\nu \text {-NCQE }}^{\mathrm{NEUT}}\right\rangle=1.512 \pm 0.004 \text {(stat.)} \pm 0.191 \text {(syst.)} \times 10^{-38} \mathrm{~cm}^2 / \text { oxygen }, \\
    & \left\langle\sigma_{\bar{\nu} \text {-NCQE }}\right\rangle=f_{\bar{\nu} \text {-NCQE }} \cdot\left\langle\sigma_{\bar{\nu} \text {-NCQE }}^{\mathrm{NEUT}}\right\rangle=0.939 \pm 0.001 \text {(stat.)} \pm 0.119 \text {(syst.)} \times 10^{-38} \mathrm{~cm}^2 / \text { oxygen }
    \end{aligned}
\label{eq:xsec_value}
\end{equation}




  \chapter{Conclusion}
\label{chp:conc}

The work presented in this thesis has two distinct parts: one is the calibration work centered around the Monte Carlo simulations of the UK Light Injection System optics and the other part is my work with neutron tagging for the Super Kamiokande Gadolinum Upgrade and the use of neutron tagging to estimate the DSNB NCQE background. 
\newline
The UK Light Injection system collimator and diffuser optics were modelled by first measuring the angular distributions of the light intensity. The cumulative distribution profiles of these outputs were sampled inside the Super Kamiokande Detector Simulator (SKDETSIM) software in order to produce Monte Carlo with the same light profiles. These Monte Carlo were verified by producing occupancy plots to visibly compare the beam spots for both types of optics against actual light injection data, and in addition to this unifrom distributions were passed through the inverse CDF functions used in the simulation and these produced values which were very close to the probability distribution functions of the original light profiles. After ensuring that the angular distributions of the light profiles were consistent with the light profiles in data, efforts were made to ensure the distributions of the time of the hits were also consistent. By adjusting the time dispersion of the injected photons in the Monte Carlo, the hit times of the scattered and reflected hits could be shifted, and various amounts of time dispersion were implemented into SKDETSIM using the Box-Muller transform to see what value of time dispersion would alter the track-step of the photon enough to be concurrent with time-of-flight distributions of the data. After a $\chi^{2}$ comparison between Monte Carlo and data timing distributions, a time dispersion value of 10ns was seen to be the best choice.
\newline
To estimate the DSNB NCQE background, neutrino flux is propoagated through event simulation, event reconstruction and neutron tagging to end up with a neutron tagging efficiency. NEUT and SKDETSIM-SKGd were used to model the interaction between the neutrino and ${ }^{16} \mathrm{O}$, and different neutron capture on gadolinium models in SKDETSIM were compared by looking at their BONSAI reconstruction variables. Comparisons of these variables between two neutron tagging algorithms were carried out, in order to ensure that no major differences were found between legacy and new NTag code. NCQE events were selected using defined criteria and comparisons were made between distributions of NCQE events for prior analyses with no gadolinium in the simulation. Distributions of truth neutron capture number, time and the associated number and energy of gamma rays were produced for both the legacy and new NTag codes, as well as the distribution of truth neutrons from events which passed the NCQE selection criteria. With regards to improving the neutron tagging algorithm, primary selection criteria were defined based on hit time and the number of hits in a 14 nsec window. There was a marked improvement in the neutron capture vertex resolution of 15.8\% compared to prior analyses with neutron captures occuring solely on hydrogen. An artifical neural network was used to further refine the algorithm and reject mis-identified neutron candidates by having an input of 12 variables relating to the number, position and isotropy of hits, and the pre and post NN background and signal candidates of these distributions are given along with whether the captures occur on hydrogen or gadolinium. The neutron tagging efficiency in MC at pre-selection and post-selection are calculated and as was expected are 40.16\% for all captures, an improvement on the efficiency in MC before the addition of gadolinium. To validate these efficiencies, comparisons were made with Am/Be + 8BGO calibration data and were shown to be consistent.
\newline
The culmination of this thesis was the calculation of the number of DSNB NCQE events, an important background to DSNB detection. By estimating the number of observed neutrino events of each interaction type for a predicted future POT of $10 \times 10^{21}$, and using the neutron tagging efficiency, the NCQE DSNB background was calculated.

 

    % Include bibliography
  \bibliographystyle{unsrt}  
  \bibliography{thesislibrary2}

% include appendix

\begin{appendices}
Figure \ref{fig:PDF_CDF_coll} shows the complete PDFs and CDFs for the B1 - B5 collimator data taken from the test stand at Warwick and Figure \ref{fig:PDF_CDF_diff} shows the complete PDFs and CDFs for the B1 - B5 diffusers. Figure \ref{fig:PDF_CDF_inv_coll} show the inverse CDF fits for the collimator, and Figure \ref{fig:PDF_CDF_inv_diff} shows the inverse CDF fits for the diffuser. 

\begin{figure}[!htbp]
    \centering
    
    \caption{PDFs and corresponding CDFs for the B1 - B5 collimators} 
    \label{fig:PDF_CDF_coll}
    \subfloat[B1 collimator PDF]{\includegraphics[width=0.49\textwidth]{Figures/B1_coll_pdf.png}}\hfill
    \subfloat[B1 collimator CDF]{\includegraphics[width=0.49\textwidth]{Figures/B1_coll_cdf.png}} \par
    \subfloat[B2 collimator PDF]{\includegraphics[width=0.49\textwidth]{Figures/B2_coll_pdf.png}} \hfill
    \subfloat[B2 collimator CDF]{\includegraphics[width=0.49\textwidth]{Figures/B2_coll_cdf.png}} \par
    \subfloat[B3 collimator PDF]{\includegraphics[width=0.49\textwidth]{Figures/B3_coll_pdf.png}} \hfill
    \subfloat[B3 collimator CDF]{\includegraphics[width=0.49\textwidth]{Figures/B3_coll_cdf.png}} \par 
\end{figure}
\begin{figure}[!htbp]
    \ContinuedFloat
    \subfloat[B4 collimator PDF]{\includegraphics[width=0.49\textwidth]{Figures/B4_coll_pdf.png}} \hfill
    \subfloat[B4 collimator CDF]{\includegraphics[width=0.49\textwidth]{Figures/B4_coll_cdf.png}} \par
    \subfloat[B5 collimator PDF]{\includegraphics[width=0.49\textwidth]{Figures/B5_coll_pdf.png}} \hfill
    \subfloat[B5 collimator CDF]{\includegraphics[width=0.49\textwidth]{Figures/B5_coll_cdf.png} } 
    
    
\end{figure}

\begin{figure}[!htbp]
    \centering
    
    \caption{PDFs and corresponding CDFs for the B1 - B5 diffusers} 
    \label{fig:PDF_CDF_diff}
    \subfloat[B1 diffuser PDF]{\includegraphics[width=0.49\textwidth]{Figures/B1_diff_pdf.png}} \hfill
    \subfloat[B1 diffuser CDF]{\includegraphics[width=0.49\textwidth]{Figures/B1_diff_cdf.png}} \par
    \subfloat[B2 diffuser PDF]{\includegraphics[width=0.49\textwidth]{Figures/B2_diff_pdf.png}} \hfill
    \subfloat[B2 diffuser CDF]{\includegraphics[width=0.49\textwidth]{Figures/B2_diff_cdf.png}} \par
    \subfloat[B3 diffuser PDF]{\includegraphics[width=0.49\textwidth]{Figures/B3_diff_pdf.png}} \hfill
    \subfloat[B3 diffuser CDF]{\includegraphics[width=0.49\textwidth]{Figures/B3_diff_cdf.png}} 
\end{figure}
\begin{figure}[!htbp]
    \ContinuedFloat
    \subfloat[B4 diffuser PDF]{\includegraphics[width=0.49\textwidth]{Figures/B4_diff_pdf.png}} \hfill
    \subfloat[B4 diffuser CDF]{\includegraphics[width=0.49\textwidth]{Figures/B4_diff_cdf.png}} \par
    \subfloat[B5 diffuser PDF]{\includegraphics[width=0.49\textwidth]{Figures/B5_diff_pdf.png}} \hfill
    \subfloat[B5 diffuser CDF]{\includegraphics[width=0.49\textwidth]{Figures/B5_diff_cdf.png}}      
\end{figure}

\begin{figure}[!htbp]
    \centering
        
    \caption{CDF and inverse CDF fits for the B1 - B5 collimator PDFs}  
    \label{fig:PDF_CDF_inv_coll}
        \subfloat[B1 inverse collimator CDF]{\includegraphics[width=0.49\textwidth]{Figures/B1_inv_coll_cdf.png}} \hfill
        \subfloat[B2 inverse collimator CDF]{\includegraphics[width=0.49\textwidth]{Figures/B2_inv_coll_cdf.png}} \par
        \subfloat[B3 inverse collimator CDF]{\includegraphics[width=0.49\textwidth]{Figures/B3_inv_coll_cdf.png}} \hfill
        \subfloat[B4 inverse collimator CDF]{\includegraphics[width=0.49\textwidth]{Figures/B4_inv_coll_cdf.png}} \par
        \subfloat[B5 inverse collimator CDF]{\includegraphics[width=0.49\textwidth]{Figures/B5_inv_coll_cdf.png}} 
    
    \end{figure}

    \begin{figure}[!htbp]
        \centering
            
        \caption{CDF and inverse CDF fits for the B1 - B5 diffuser PDFs}  
        \label{fig:PDF_CDF_inv_diff}   
            \subfloat[B1 inverse diffuser CDF]{\includegraphics[width=0.49\textwidth]{Figures/B1_inv_diff_cdf.png}} \hfill
            \subfloat[B2 inverse diffuser CDF]{\includegraphics[width=0.49\textwidth]{Figures/B2_inv_diff_cdf.png}} \par
            \subfloat[B3 inverse diffuser CDF]{\includegraphics[width=0.49\textwidth]{Figures/B3_inv_diff_cdf.png}} \hfill
            \subfloat[B4 inverse diffuser CDF]{\includegraphics[width=0.49\textwidth]{Figures/B4_inv_diff_cdf.png}} \par
            \subfloat[B5 inverse diffuser CDF]{\includegraphics[width=0.49\textwidth]{Figures/B5_inv_diff_cdf.png}} 
        
    \end{figure}
\end{appendices}


\end{document}
