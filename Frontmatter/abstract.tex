\chapter*{Abstract}
\thispagestyle{empty}

The error on the DSNB NCQE background is calculated for an estimated future POT of $10 \times 10^{21}$ starting JFY 2028, and is estimated at $N_{\nu-\mathrm{DSNB}_{NCQE}}= 9.88 \pm 3.62$ events for FHC mode. The neutral current quasi-elastic (NCQE) interactions of neutrinos on ${ }^{16} \mathrm{O}$ generated by the beam neutrinos of the T2K (Tokai-to-Kamioka) experiment produces neutrons in the medium of the far detector Super Kamiokande. This thesis focuses on the detection of these neutrons by a neutron tagging algorithm specifically developed for the addition of 0.026\% $\mathrm{Gd}_{2}\left(\mathrm{SO}_{4}\right)_{3} \cdot 8 \mathrm{H}_{2} \mathrm{O}$ to Super-Kamiokande (the level present in SK from August 2020 - May 2022), modelled in the MC simulation by the ANNRI-Gd model. The neutron tagging algorithm is complemented by the use of an artificial neural network (ANN), which is used on neutron candidates to improve sensitivity to signal events and to reduce backround events. The efficiency of this neutron tagging algorithm is calculated and validated using calibration data from neutrons produced by an Am/Be + 8BGO source.

The calibration work in this thesis centers on developing Monte Carlo for the collimator and diffuser UK Light Injector optics, which were compared against test run data. The time-of-flight (TOF) hit time distributions of these MC were compared against data and the time disperson of the MC was adjusted to approximate that of data.