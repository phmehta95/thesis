\chapter*{Statement of Originality}
\thispagestyle{empty}

This thesis and the work presented in it stem from my own original and independent research. However, contributions from other researchers working on the Tokai-to-Kamioka and Super-Kamiokande experiments are also presented and referenced. My NCQE analysis was carried out with input from the neutron tagging working group, and my calibration work was carried out with input from the Super-Kamiokande and Hyper-Kamiokande calibration working groups.

Chapter 1 gives an overview of neutrino physics, including the solar neutrino problem, neutrino oscillation and the diffuse supernova neutrino background, and also an overview of neutrino-nucleus interactions. All relevant books and articles used to gather this information have been cited.

Chapter 2, 3 and 4 give an overview of the T2K and Super-K experiments relevant to this thesis, including design, data aquisition methods, information on the SK-Gd upgrade and information on calibration methods carried out at Super-K. These three chapters summarise prior analyses carried out such as those performed by the EGADS experiment and by the calibration group at Super-K, and relevant publications have been cited. 

Chapter 5 presents my contribution to the Super-Kamiokande calibration group and implementation of the UKLI MC. The measurements of the light profiles of the UKLI optics from the test stands were taken by Dr Sammy Valder (University of Warwick), but their implementation into SKDETSIM was done by myself along with the testing of the output and the variation of the time dispersion. The code for the event display for the occupancy plots was provided by Dr Billy Vinning (University of Warwick) but was modified to receive MC output by myself. The base code for the $\chi^{2}$ comparison for time-of-flight corrected timing distributions was developed by Dr Adrian Pritchard (formerly University of Liverpool) but rewritten to accomadate my SK UKLI MC. 

Chapter 6 presents the NCQE analysis of my thesis, with an emphasis on the development of the neutron tagging algorithm. The legacy neutron tagging algorithm was developed from neutron tagging on hydrogen code (developed by Dr Ryosuke Akutsu (TRIUMF)) by myself to accomodate for the addition of $\mathrm{Gd}_{2}\left(\mathrm{SO}_{4}\right)_{3} \cdot 8 \mathrm{H}_{2} \mathrm{O}$ to Super-K. The new NTag algorithm was developed by Seungho Han (ICRR) and was modified by myself to work for the NCQE analysis, and the NCQE selection code was rewritten from scratch by myself to work with the new NTag algorithm. Due to many factors, there is no SK phase VI data for comparison with the NCQE MC plots in this thesis. This is due to the NIWG model and flux tunings not being updated for this analysis during the completion of this chapter and also due to the collaboration's ongoing investigations into discrepancies regarding neutron capture models between SKDETSIM-SKGd (used in this analysis) and SKG4 (a C++/Geant4 based Super-K detector simulator). 

Chapter 7 presents the systematic errors for my analysis, and uses the T2K beam working group for the neutrino beam fluxes and the T2K neutrino interaction working group (NIWG) for the neutrino interaction cross sections. Although these inputs were taken from external sources, the method of implementation had to be modified so the errors could be propagated with SKDETSIM-SKGd, as opposed to for SKDETSIM (used in prior NCQE ntag analyses), and this was implemented for each error by myself. 

Chapter 8 presents the DSNB NCQE background measurement for a future POT of $10 \times 10^{21}$. The total number of NCQE events and the scale factor and DSNB NCQE background measurement error were calculated by myself, however the number of DSNB background events were taken from plots produced by the DSNB working group and presented in the December 2022 Super-Kamiokande Collaboration meeting, and a relevant link and citation have been given. 