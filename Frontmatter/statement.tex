\chapter*{Statement of Originality}
\thispagestyle{empty}

This thesis and the work presented in it stem from my own original and independent research. However, contributions from other researchers working on the Tokai-to-Kamioka and Super-Kamiokande are also presented and referenced as analyses and the code/software used to develop that analysis are always collaboratively produced and improvements on these analyses are discussed in working group meetings with multiple PhD students, postdoctoral researchers and professors present. My NCQE analysis was carried out with input from the neutron tagging working group, and my calibration work was carried out with input from the Super-Kamiokande and Hyper-Kamiokande calibration working groups.

Chapter 1 gives an overview of neutrino physics, including the solar neutrino problem, neutrino oscillation and the diffuse supernova neutrino background, and also an overview of neutrino-nucleus interactions. All relevant books and articles used to gather this information has been cited.

Chapter 2 and 3 and 4 give an overview of the T2K and Super-K experiments relevant to this thesis, including design, data aquisition methods, information on the SK-Gd upgrade and information on calibration methods carried out at Super-K. These three chapters summarise prior analyses carried out such as those performed by the EGADS experiment and by the calibration group at Super-K, and relevant publications have been cited. 

Chapter 5 presents my contribution to the Super-Kamiokande calibration group and implementation of the UKLI MC. The measurements of the light profiles of the UKLI optics from the test stands were taken at the University of Warwick by Dr Sammy Valder, but their implementation into SKDETSIM was done by myself along with the testing of the output and the variation of the time dispersion. The code for 